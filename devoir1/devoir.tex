\documentclass[11pt]{article}
\usepackage[french]{babel}
\usepackage[utf8]{inputenc}
\usepackage[T1]{fontenc}
\usepackage{standalone}
\usepackage{geometry}
\usepackage{gphys}
\usepackage{siunitx}
\usepackage{amsmath}
\usepackage{amsfonts}
\usepackage{mathtools}
\usepackage{booktabs}
\usepackage{float}
\usepackage{pgfplots}
\usepackage{tikz}
\usepackage{minted}
\usepackage{relsize}
\usepackage{nicefrac}
\usepackage{multicol}

\usetikzlibrary{decorations.pathreplacing}
\usetikzlibrary{arrows}

%%%%%%%%%%%%%%%%%%
% Configurations %
%%%%%%%%%%%%%%%%%%

\geometry{
	total={210mm,297mm},
	left=34mm,
	right=34mm,
	top=30mm,
	bottom=30mm,
}

\sisetup{locale = FR}

\makeatletter
\let\@afterindenttrue\@afterindentfalse
\makeatother

%%%%%%%%%%
% Macros %
%%%%%%%%%%

\newcommand\comb{%
    \ensuremath{%
        \mathcal{C}%
    }%
}

\newcommand\perm{%
    \ensuremath{%
        \mathcal{P}%
    }%
}

\renewcommand\P{%
    \ensuremath{%
        \mathbb{P}%
    }%
}

\makeatletter
\newcommand*{\centerfloat}{%
    \parindent \z@%
    \leftskip \z@ \@plus 1fil \@minus \textwidth%
    \rightskip\leftskip%
    \parfillskip \z@skip}%
\makeatother

%%%%%%%%%%%%%
% Documents %
%%%%%%%%%%%%%

\begin{document}
\def\Sigle{MTH2302A}
\def\Cours{Probabilités et Statistique}
\def\Devoir{Devoir 1}
\AjouterAuteur{Gabriel-Andrew}{Pollo-Guilbert}{1837776}
\AjouterProfesseur{}{Simon}{Demontigny}
\def\logoscale{1}
\pagetitre

\section{Deux par deux}
Dans ce problème, on alloue des sièges à $B$ voyageurs solitaires en
premier et ensuite à $A$ paires. On s'intérèsse aux cas où les $D$
paires ne peuvent pas être ensemble.

L'avion est constitué de $C$ rangées contenant chacune 2 sections de
3 sièges. La figure \ref{fig:avion} représente la répartition des sièges
dans l'avion.
\begin{figure}[H]
    \centering
    \documentclass[11pt]{article}
\usepackage[french]{babel}
\usepackage[utf8]{inputenc}
\usepackage[T1]{fontenc}
\usepackage{standalone}
\usepackage{tikz}

\begin{document}
\def\h{15mm}
\def\w{ 5mm}

\def\T{}\def\M{}\def\B{}

\newcommand\triplet[2]{%
    \begin{scope}[xshift=#1,yshift=#2]%
        \draw (-\w/2,-\h/2) -- (-\w/2,\h/2);%
        \draw ( \w/2,-\h/2) -- ( \w/2,\h/2);%
        %
        \foreach\y in {0,1,2,3} {%
            \draw (-\w/2,\y*\h/3-\h/2) -- (\w/2,\y*\h/3-\h/2);%
        }%
        %
        \node at (0, \h/3) {\T};%
        \node at (0, 0)    {\M};%
        \node at (0,-\h/3) {\B};%
        %
        \def\T{}\def\M{}\def\B{}%
    \end{scope}%
}

\begin{tikzpicture}
\foreach\x in {1,2,3,4,5} {
    \triplet{\x cm}{ 1 cm}
    \triplet{\x cm}{-1 cm}
    \node at (\x cm, 0) {\x};
}

\triplet{7 cm}{ 1 cm}
\triplet{7 cm}{-1 cm}
\node at (7 cm, 0) {$C$};
\node at (6 cm, 1 cm) {$\cdots$};
\node at (6 cm,-1 cm) {$\cdots$};
    
\end{tikzpicture}
\end{document}




    \caption{répartition des sièges dans l'avion}
    \label{fig:avion}
\end{figure}

On cherche à placer les voyageurs seuls de manière à ce qu'il reste exactement
$A-D$ sections pouvant contenir des paires. Pour chaque section, il existe 5
manières d'empêcher une paire d'être ensemble. La figure \ref{fig:avion_solo}
montre ces 5 cas.

\begin{figure}[H]
    \centering
    \input{figure/avion_solo}
    \caption{5 cas pouvant empêcher une paire d'être ensemble}
    \label{fig:avion_solo}
\end{figure}

Si l'on commence à la première section, on peut choisir une des 5 manières de
bloquer la formation d'une paire. À la deuxième section, on choisie à nouveau
une des 5 manières. À la troisième aussi et ainsi de suite. Par contre, il faut
s'assurer qu'il reste des voyageurs à placer et aussi de laisser $A-D$ sections
pour les paires.

Pour visualiser le problème, on peut dessiner un arbre pour chaque étape possible.
Pour chaque feuille, on dénote l'état $(c,n,s)$ où $c$ est le nombre de sections
manquantes à bloquer, $n$ est le nombre restant de voyageurs seuls et $s$ le nombre
de sections à laisser. À chaque feuille, il y a une branche possible pour chaque cas
pouvant bloquer une paire. La figure \ref{fig:avion_arbre_cns} représente une telle
feuille.

\begin{figure}[H]
    \centerfloat
    \documentclass[11pt]{article}
\usepackage[french]{babel}
\usepackage[utf8]{inputenc}
\usepackage[T1]{fontenc}
\usepackage{standalone}
\usepackage{tikz}

\begin{document}
\begin{tikzpicture}[
    scale=0.8,
    every node/.style={},level distance=40pt,
    level 1/.style={sibling distance=40mm}]
\node {$(c,n,s)$}
    child {
        node {$(c-1,n-3,s)$}
    }
    child {
        node {$(c-1,n-2,s)$}
    }
    child {
        node {$(c-1,n-2,s)$}
    }
    child {
        node {$(c-1,n-2,s)$}
    }
    child {
        node {$(c-1,n-1,s)$}
    }
;
\end{tikzpicture}
\end{document}




    \caption{feuille d'un arbre $(c,n,s)$}
    \label{fig:avion_arbre_cns}
\end{figure}

En effet, le nombre de sections restantes à bloquer diminue à chauque étape. De plus,
le nombre restant de voyageurs diminue en fonction du cas pris. On remarque que trois
branches se répète, c'est-à-dire celles où deux voyageurs sont utilisés pour bloquer
une paire. Pour chaque feuille, on peut définir une fonction de récurrence comme suit
\begin{equation*}
    f(c,n,s)=f(c-1,n-1,s)+3f(c-1,n-2,s)+f(c-1,n-3,s).
\end{equation*}

Pour définir la fin de la récurrence, il faut regarder un exemple. Soit 5 voyageurs
qui doivent être placer sur 3 sections afin de permettre une paire libre. La figure
\ref{fig:avion_paires} montre quelques cas possibles lorsque le paramètre $c=0$ dans
la récurrence, c'est-à-dire qu'il ne reste plus de sections à bloquer. Les X dénotent
deux voyageurs seuls tandis que les $P$ la paire qui doit être ensemble.

\begin{figure}[H]
    \centerfloat
    \documentclass[11pt]{article}
\usepackage[french]{babel}
\usepackage[utf8]{inputenc}
\usepackage[T1]{fontenc}
\usepackage{standalone}
\usepackage{tikz}

\begin{document}
\def\h{15mm}
\def\w{ 5mm}

\def\T{}\def\M{}\def\B{}

\newcommand\triplet[2]{%
    \begin{scope}[xshift=#1,yshift=#2]%
        \draw (-\w/2,-\h/2) -- (-\w/2,\h/2);%
        \draw ( \w/2,-\h/2) -- ( \w/2,\h/2);%
        %
        \foreach\y in {0,1,2,3} {%
            \draw (-\w/2,\y*\h/3-\h/2) -- (\w/2,\y*\h/3-\h/2);%
        }%
        %
        \node at (0, \h/3) {\T};%
        \node at (0, 0)    {\M};%
        \node at (0,-\h/3) {\B};%
        %
        \def\T{}\def\M{}\def\B{}%
    \end{scope}%
}

\begin{tikzpicture}
\begin{scope}
    \def\T{}\def\M{X}\def\B{}
    \triplet{0.0cm}{0cm}

    \def\T{X}\def\M{X}\def\B{}
    \triplet{0.6cm}{0cm}
    
    \def\T{P}\def\M{P}\def\B{}
    \triplet{1.2cm}{0cm}    
\end{scope}

\begin{scope}[xshift=3cm]
    \def\T{X}\def\M{X}\def\B{X}
    \triplet{0.0cm}{0cm}

    \def\T{X}\def\M{X}\def\B{X}
    \triplet{0.6cm}{0cm}
    
    \def\T{P}\def\M{P}\def\B{}
    \triplet{1.2cm}{0cm}    
\end{scope}

\begin{scope}[xshift=6cm]
    \def\T{X}\def\M{X}\def\B{X}
    \triplet{0.0cm}{0cm}

    \def\T{X}\def\M{X}\def\B{}
    \triplet{0.6cm}{0cm}
    
    \def\T{P}\def\M{P}\def\B{}
    \triplet{1.2cm}{0cm}    
\end{scope}

\begin{scope}[xshift=9cm]
    \def\T{X}\def\M{X}\def\B{X}
    \triplet{0.0cm}{0cm}

    \def\T{}\def\M{X}\def\B{}
    \triplet{0.6cm}{0cm}
    
    \def\T{P}\def\M{P}\def\B{}
    \triplet{1.2cm}{0cm}    
\end{scope}

\end{tikzpicture}
\end{document}




    \caption{exemples de cas finals}
    \label{fig:avion_paires}
\end{figure}

Lorsque $c=0$, il faut finir d'allouer des sièges aux voyageurs restants. On ne peut
pas leur donner les sièges restants des cas bloquants, car ce sont des possibilités
calculées à priori par une autre branche. Par conséquent, on peut seulement allouer
les sièges à coté des paires ensembles.

Dans certains cas, la branche est impossible car il n'y a pas assez de sièges
disponibles pour les voyageurs ou il y plus de sièges alloués que de voyageurs seuls.
Ces conditions se remarque si $n<0$ ou $n>s$, car il y a un siège restant pour chaque
paire ensemble.

Dans certains cas, il ne reste plus de voyageurs sans siège, alors la branche est
valide. Lorsqu'il y reste des voyageurs et qu'il y a assez de sièges libres à coté
des paires ensembles, il faut regarder un autre exemple.

Soit 3 voyageurs qui doivent être placés sur 4 sections afin de permettre trois
paires d'être ensemble. La figure \ref{fig:avion_paires_more} montre des exemples
respectant ces contraintes, avec R dénotant les voyageurs restants après avoir
remplie les sections bloquées.

\begin{figure}[H]
    \centerfloat
    \input{figure/avion_paires_more}
    \caption{exemples de cas finals}
    \label{fig:avion_paires_more}
\end{figure}

Dans ces cas, il y a 2 voyageurs restants. Ils peuvent être à coté de n'importe
quelle paire. Par conséquent, il faut choisir 2 paires parmis les 3. De plus, chaque
voyageur restant peut être à droite ou à gauche de la paire. Par conséquent, le
nombre de possibilités de d'allouer des sièges aux voyageurs restants est donné
par $2^2\comb_3^2$.

En général, on peut définir le système d'équations à récurrence suivant,
\begin{equation*}
    \left\{
        \arraycolsep=1.4pt\def\arraystretch{1}
        \begin{array}{rcl}
            f(0,0,s)&=&1\\
            f(0,n,s)&=&\left\{
                \begin{matrix}
                    2^n\comb_s^n &\text{ si } 0\leq n\leq s\\
                    0            &\text{sinon}\\
                \end{matrix}
            \right.\\
            f(c,n,s)&=&f(c-1,n-1,s)+3f(c-1,n-2,s)+f(c-1,n-3,s)
        \end{array}
    \right.
\end{equation*}
pour calculer le nombre de possibilités à partir de n'importe quelle feuille.

Dans notre cas, il faut placer 12 voyageurs solos afin de laisser 28 paires
ensembles et 2 paires séparées dans les 18 rangées (36 sections). Il y a donc
$36-28=8$ sections bloquées. Le nombre de possibilités est donné par $f(8,12,28)$.
Ceci dit, le calcul prend juste en compte les 8 premières rangées comme étant
bloquées. Hors, il faut choisir 8 rangées bloquées parmis les 36 disponibles.
On ajuste le calcul de sorte que le nombre de possibilités est donné par
$\comb_{36}^8f(8,12,28)$.

Si on ne prend pas compte de la méthode de répartition des sièges, le nombre total
de possibilités est donné en choisisant 12 sièges parmis les 108 disponibles. Par
conséquent, la probabilité de $A$, c'est-à-dire d'avoir exactement 2 paires
séparées, est donnée par
\begin{equation*}
    \P(A)
    =\frac{\comb_{36}^8f(8,12,28)}{\comb_{12}^{108}}
    =\frac{\SI{8199405016}{}}{\SI{520752510551}{}}
    \approx\SI{1.574}{\percent}.
\end{equation*}

À la page suivant ce trouve une implémentation de $f(c,n,s)$ en Haskell. Le temps
d'exécution est instantané pour les paramètres du problème. De plus, un simulateur
écrit en C est fournit afin de comparer la réponse du calcul. Il est probablement
possible d'optimiser le calcul en éliminant les branches impossibles dès qu'elles
arrivent, mais le calcul fut assez rapide qu'il en n'était pas nécessaire.

\pagebreak
\newgeometry{left=20mm,right=6mm,top=20mm}
\inputminted[linenos]{haskell}{avion/src/Main.hs}
\vspace{1cm}
\begin{multicols}{2}
    \inputminted[linenos,fontsize=\footnotesize]{C}{avion/main.c}
\end{multicols}
\restoregeometry


\pagebreak
\section{Deux têtes valent mieux qu’une}
Soit une équipe dont un membre à le numéro le plus élevé $m$. Cela implique qu'il
y a au moins 1 membre pour chaque numéro positif inférieur à $m$. La prochaine
personne à ce joindre au groupe peut être inviter par n'importe quelle personne
ayant un numéro dans $\{0,1,\dots,m\}$. Par conséquent, le nouveau membre
se vera attribuer un numéro dans $\{1,2,\dots,m+1\}$. La figure \ref{fig:arbre_comb}
montre les formations possibles d'une équipe à 5 personnes.

\begin{figure}[H]
    \centering
    \input{figure/arbre_comb}
    \caption{formations possibles pour un groupe à 5 personnes}
    \label{fig:arbre_comb}
\end{figure}

On remarque que le numéro de la dernière personne est sans importance dans
le processus. Seulement le plus grand numéro dans l'équipe détermine le nombre
de choix à une feuille de l'arbre. On peut représenter chaque feuille par un
couple $(n,m)$ où $n$ est le nombre restant de membre à ajouter et $m$ le numéro
maximum actuel dans le groupe. La figure \ref{fig:arbre_comb_state} représente
un arbre en utilisant cette notation pour les feuilles. Les branches sont
numérotés en fonction du membre inviter à une étape.

\begin{figure}[H]
    \centering
    \documentclass[11pt]{standalone}
\usepackage[french]{babel}
\usepackage[utf8]{inputenc}
\usepackage[T1]{fontenc}
\usepackage{standalone}
\usepackage{tikz}
\usepackage{relsize}

\begin{document}
\begin{tikzpicture}[
    every node/.style={scale=0.8,font=\relsize{0.5}},
    level distance=30pt,
    level 2/.style={sibling distance=70mm},
    level 3/.style={sibling distance=29mm},
    level 4/.style={sibling distance= 8mm}]
\tikzset{branche/.style={inner sep=1,outer sep=1,fill=white}}
    
\node {$(4,0)$}
    child {
        node {$(3,1)$}
        child {
            node {$(2,1)$} 
            child {
                node {$(1,1)$}
                child {
                    node {$(0,1)$}
                    edge from parent node[branche] {\small 1}
                }
                child {
                    node {$(0,2)$}
                    edge from parent node[branche] {\small 2}
                }
                edge from parent node[branche] {\small 1}
            }
            child {
                node {$(1,2)$}
                child {
                    node {$(0,2)$}
                    edge from parent node[branche] {\small 1}
                }
                child {
                    node {$(0,2)$}
                    edge from parent node[branche] {\small 2}
                }
                child {
                    node {$(0,3)$}
                    edge from parent node[branche] {\small 3}
                }
                edge from parent node[branche] {\small 2}
            }
            edge from parent node[branche] {\small 1}
        }
        child {
            node {$(2,2)$}
            child {
                node {$(1,2)$}
                child {
                    node {$(0,2)$}
                    edge from parent node[branche] {\small 1}
                }
                child {
                    node {$(0,2)$}
                    edge from parent node[branche] {\small 2}
                }
                child {
                    node {$(0,3)$}
                    edge from parent node[branche] {\small 3}
                }
                edge from parent node[branche] {\small 1}
            }
            child {
                node {$(1,2)$}
                child {
                    node {$(0,2)$}
                    edge from parent node[branche] {\small 1}
                }
                child {
                    node {$(0,2)$}
                    edge from parent node[branche] {\small 2}
                }
                child {
                    node {$(0,3)$}
                    edge from parent node[branche] {\small 3}
                }
                edge from parent node[branche] {\small 2}
            }
            child {
                node {$(1,3)$}
                child {
                    node {$(0,3)$}
                    edge from parent node[branche] {\small 1}
                }
                child {
                    node {$(0,3)$}
                    edge from parent node[branche] {\small 2}
                }
                child {
                    node {$(0,3)$}
                    edge from parent node[branche] {\small 3}
                }
                child {
                    node {$(0,4)$}
                    edge from parent node[branche] {\small 4}
                }
                edge from parent node[branche] {\small 3}
            }
            edge from parent node[branche] {\small 2}
        }
        edge from parent node[branche] {\small 1}
    };
\end{tikzpicture}
\end{document}




    \caption{représentation $(n,m)$ de l'abre de la figure \ref{fig:arbre_comb}}
    \label{fig:arbre_comb_state}
\end{figure}

\subsection{Formations possibles}
En examinant l'arbre de la figure \ref{fig:arbre_comb_state}, on remarque que
chaque feuille contient $m$ sous-feuilles identiques auxquelles le prochain
membre obtient un numéro déjà existant dans le groupe. De plus, une feuille a
aussi toujours une sous-feuille où le prochain membre obtient un numéro supérieur
à ceux dans tout le groupe. La figure \ref{fig:arbre_feuille} résume n'importe
quelle feuille $(n,m)$.

\begin{figure}[H]
    \centering
    \documentclass[11pt]{standalone}
\usepackage[french]{babel}
\usepackage[utf8]{inputenc}
\usepackage[T1]{fontenc}
\usepackage{standalone}
\usepackage{tikz}
\usepackage{relsize}

\usetikzlibrary{decorations.pathreplacing}

\begin{document}
\begin{tikzpicture}[
    every node/.style={scale=0.8,font=\relsize{0.5}},
    level distance=30pt,
    level 1/.style={sibling distance=20mm}]
\tikzset{branche/.style={inner sep=1,outer sep=1,fill=white}}
    
\node {$(n,m)$}
    child {node (S) {$(n-1,m)$}}
    child {node     {$(n-1,m)$}}
    child {node     {$\cdots$}}
    child {node (E) {$(n-1,m)$}}
    child {node     {$(n-1,m+1)$}};
\draw[decorate,decoration={brace,amplitude=10pt,raise=5pt}]
    (E.east) -- (S.west)
    node[midway,yshift=-1cm] {$m$ fois};
\end{tikzpicture}
\end{document}




    \caption{possibilités du prochain numéro à une feuille $(n,m)$}
    \label{fig:arbre_feuille}
\end{figure}

Pour chaque feuille, on définie une fonction récursive $f_1$ nous donnant le
nombre de formations possibles sous celle-ci tel que
\begin{equation*}
    f_1(n,m)=f_1(n-1,m+1)+mf_1(n-1,m).
\end{equation*}

La récursion termine lorsqu'il ne reste plus personne à ajouter, soit
\begin{equation*}
    f_1(0,m)=1,
\end{equation*}
et elle commence par le fondateur ayant le numéro 0, c'est-à-dire
\begin{equation*}
    f_1(n,0)=f_1(n-1,1),
\end{equation*}
car il ne peut qu'inviter une personne en lui attribuant le numéro 1. Par
conséquent, le système d'équations suivant calcule le nombre de formations
possibles, soit
\begin{equation}\label{eq:f1}
    \arraycolsep=1.4pt\def\arraystretch{1}
    \left\{
        \begin{array}{lcl}
            f_1(n,m)&=&f_1(n-1,m+1)+mf_1(n-1,m)\\
            f_1(n,0)&=&f_1(n-1,1)\\
            f_1(0,m)&=&1\\
        \end{array}
    \right.
\end{equation}

\subsection{Formations possibles ayant exactement $F$ numéro 1}
On s'intéresse au nombre de formations ayant exactement $F$ personnes
invitées par le fondateur. La figure \ref{fig:arbre_comb_1} montre
les formations d'une équipe contenant 5 membres et exactement 3
personnes invitées par le fondateur. Les feuilles en rouge dénote
une formation ne respectant pas cette condition.

\begin{figure}[H]
    \centering
    \input{figure/arbre_comb_1}
    \caption{formations de 5 membres contenant exactement 3 numéro 1}
    \label{fig:arbre_comb_1}
\end{figure}

Pour calculer le nombre de formations possibles, on utilise une
récursion similaire à celle présentée en \eqref{eq:f1}. Par contre, il
faut ajouter un nouveau paramètre à chaque feuille de l'arbre. On dénote
chaque feuille $(n,m,c)$ où $c$ est le compte actuel de personnes invitées
par le fondateur. La figure \ref{fig:arbre_comb_1_state} représente un
arbre utilisant cette notation.

\begin{figure}[H]
    \centerfloat
    \input{figure/arbre_comb_1_state}
    \caption{représentation $(n,m,c)$ de l'abre de la figure \ref{fig:arbre_comb_1}}
    \label{fig:arbre_comb_1_state}
\end{figure}

En examinant l'arbre de la figure \ref{fig:arbre_comb_1_state}, on remarque
qu'il est très similaire à celui de la figure \ref{fig:arbre_comb_state}. La
seule différence est que $c$ est incrémenter une fois par feuille, c'est-à-dire
lorsque le fondateur invite une personne. La figure \ref{fig:arbre_feuille_1}
représente n'importe quelle feuille $(n,m,c)$.

\begin{figure}[H]
    \centering
    \documentclass[11pt]{standalone}
\usepackage[french]{babel}
\usepackage[utf8]{inputenc}
\usepackage[T1]{fontenc}
\usepackage{standalone}
\usepackage{tikz}
\usepackage{relsize}

\usetikzlibrary{decorations.pathreplacing}

\begin{document}
\begin{tikzpicture}[
    every node/.style={scale=0.8,font=\relsize{0.5}},
    level distance=30pt,
    level 1/.style={sibling distance=25mm}]
\tikzset{branche/.style={inner sep=1,outer sep=1,fill=white}}
    
\node {$(n,m,c)$}
    child {node     {$(n-1,m,c+1)$}}
    child {node (S) {$(n-1,m,c)$}}
    child {node     {$\cdots$}}
    child {node (E) {$(n-1,m,c)$}}
    child {node     {$(n-1,m+1,c)$}};
\draw[decorate,decoration={brace,amplitude=10pt,raise=5pt}]
    (E.east) -- (S.west)
    node[midway,yshift=-1cm] {$(m-1)$ fois};
\end{tikzpicture}
\end{document}




    \caption{possibilités du prochain numéro à une feuille $(n,m,c)$}
    \label{fig:arbre_feuille_1}
\end{figure}

Pour chaque feuille, on définie une fonction récursive $f_2$ nous donnant le
nombre de formations possibles sous celle-ci tel que
\begin{equation*}
    f_2(n,m,c)=f_2(n-1,m,c+1)+f_2(n-1,m+1,c)+(m-1)f_2(n-1,m,c)
\end{equation*}

La récursion termine lorsqu'il ne reste plus personne à ajouter et elle est
est valide si $c=F$, soit
\begin{equation*}
    f_2(0,m,F)=1,
\end{equation*}
sinon
\begin{equation*}
    f_2(0,m,c)=0,
\end{equation*}
et elle commence par le fondateur ayant le numéro 0, c'est-à-dire
\begin{equation*}
    f_2(n,0,c)=f_2(n-1,1,1),
\end{equation*}
car il ne peut qu'inviter une personne en lui attribuant le numéro 1. Par
conséquent, le système d'équations suivant calcule le nombre de formations 
possibles, soit
\begin{equation}\label{eq:f2}
    \arraycolsep=1.4pt\def\arraystretch{1}
    \left\{
        \begin{array}{lcl}
            f_2(n,m,c)&=&f_2(n-1,m,c+1)+f_2(n-1,m+1,c)+(m-1)f_2(n-1,m,c)\\
            f_2(n,0,c)&=&f_2(n-1,1,1)\\
            f_2(0,m,c)&=&0\\
            f_2(0,m,F)&=&1\\
        \end{array}
    \right.
\end{equation}

\subsection{Formations possibles d'aucun numéro supérieur à $G$}
Pour trouver les formations ne contenant pas de numéro supérieur à $G$,
il suffit de les filtrer à la fin du calcul. Étant que l'on suit déjà
le plus grand numéro dans la représentation $(n,m)$, il suffit de
redéfinir la fin de la récursion par
\begin{equation*}
    f_3(0,m)=\left\{
        \begin{matrix}
            1&\text{si}&m\leq G\\
            0&\text{si}&m>    G\\
        \end{matrix}
    \right.,
\end{equation*}
de sorte à obtenir la fonction $f_3$ définie par le système d'équations
\begin{equation}\label{eq:f3}
    \arraycolsep=1.4pt\def\arraystretch{1}
    \left\{
        \begin{array}{lcl}
            f_3(n,m)&=&f_3(n-1,m+1)+mf_3(n-1,m)\\
            f_3(n,0)&=&f_3(n-1,1)\\
            f_3(0,m)&=&\left\{
                \begin{matrix}
                    1&\text{si}&m\leq G\\
                    0&\text{si}&m>    G\\
                \end{matrix}
            \right.,\\
        \end{array}
    \right.
\end{equation}

La figure \ref{fig:arbre_comb_s} montre le cas d'une équipe de 5 personnes
ne pouvant pas contenir des numéros supérieurs à 2.

\begin{figure}[H]
    \centering
    \documentclass[11pt]{standalone}
\usepackage[french]{babel}
\usepackage[utf8]{inputenc}
\usepackage[T1]{fontenc}
\usepackage{standalone}
\usepackage{tikz}

\begin{document}
\begin{tikzpicture}[
    scale=0.8,
    every node/.style={circle,draw},level distance=30pt,
        level 2/.style={sibling distance=85mm},
        level 3/.style={sibling distance=35mm},
        level 4/.style={sibling distance=10mm}]
\node {0}
    child {
        node {1}
        child {
            node {1}
            child {
                node {1}
                child {
                    node {1}
                }
                child {
                    node {2}
                }
            }
            child {
                node {2}
                child {
                    node {1}
                }
                child {
                    node {2}
                }
                child {
                    node[draw=red,text=red] {3}
                }
            }
        }
        child {
            node {2}
            child {
                node {1}
                child {
                    node {1}
                }
                child {
                    node {2}
                }
                child {
                    node[draw=red,text=red] {3}
                }
            }
            child {
                node {2}
                child {
                    node {1}
                }
                child {
                    node {2}
                }
                child {
                    node[draw=red,text=red] {3}
                }
            }
            child {
                node {3}
                child {
                    node[draw=red,text=red] {1}
                }
                child {
                    node[draw=red,text=red] {2}
                }
                child {
                    node[draw=red,text=red] {3}
                }
                child {
                    node[draw=red,text=red] {4}
                }
            }
        }
    };
\end{tikzpicture}
\end{document}




    \caption{formations ne contenant pas des numéros supérieurs à 2}
    \label{fig:arbre_comb_s}
\end{figure}

\subsection{Formations avec les deux contraintes}
En utilisant la même logique qu'au paravant, il suffit d'ajouter la condition
établie en \eqref{eq:f3} dans \eqref{eq:f2} afin d'obtenir la fonction $f_5$
définie par le système d'équations
\begin{equation}\label{eq:f2}
    \arraycolsep=1.4pt\def\arraystretch{1}
    \left\{
        \begin{array}{lcl}
            f_4(n,m,c)&=&f_4(n-1,m,c+1)+f_4(n-1,m+1,c)+(m-1)f_4(n-1,m,c)\\
            f_4(n,0,c)&=&f_4(n-1,1,1)\\
            f_4(0,m,c)&=&0\\
            f_4(0,m,F)&=&\left\{
                \begin{matrix}
                    1&\text{si}&m\leq G\\
                    0&\text{si}&m>    G\\
                \end{matrix}
            \right.,\\
        \end{array}
    \right.
\end{equation}

\subsection{Solution}
Soit des groupes de $E=17$ personnes et les événements
\begin{table}[H]
    \centering
    \begin{tabular}{l}
        A: le fondateur a invité exactement $F=3$ personnes\\
        B: il n'y a aucun numéro supérieur à $G=7$\\
    \end{tabular}
\end{table}

Il y a un total de $f_1(16,0)$ formations possibles. Parmis celles-ci, il y a
$f_2(16,0,0)$ formations dont le fondateur a invité 3 personnes. Finalement, il
y a $f_3(16,0)$ formations dans lequel il n'y a aucun numéro supérieur à 7.
\footnote{On utilise $n=16$ au lieu de $n=E=17$, car le fondateur est compté
parmis les membres de l'équipe dès le départ}

La probabilité qu'une équipe formée contient 3 membres invités par le fondateur est
\begin{equation*}
    \mathbb{P}(A)
    =\frac{f_2(16,0,0)}{f_1(16,0)}
    =\frac{\SI{2902665885}{}}{\SI{10480142147}{}}
\end{equation*}
et celle qu'il n'y ait aucune numéro supérieur à 7 est
\begin{equation*}
    \mathbb{P}(B)
    =\frac{f_3(16,0)}{f_1(16,0)}
    =\frac{\SI{7291973067}{}}{\SI{10480142147}{}}.
\end{equation*}

La probabilité qu'on ne retrouve aucun étudiant avec un numéro supérieure à 7
sachant que le fondateur a invité exactement 3 personnes est donnée par
\begin{equation*}
    \mathbb{P}(B|A)
    =\frac{\mathbb{P}(A\cap B)}{\mathbb{P}(A)},
\end{equation*}
où la probabilité d'avoir les deux contraintes est donné par
\begin{equation*}
    \mathbb{P}(A\cap B)
    =\frac{f_4(16,0,0)}{f_1(16,0)}
    =\frac{\SI{2060946720}{}}{\SI{10480142147}{}},
\end{equation*}
de sorte à obtenir
\begin{equation*}
    \mathbb{P}(B|A)
    =\frac{\SI{2060946720}{}}{\SI{2902665885}{}}
    \approx\SI{0.71}{\percent}.
\end{equation*}

La prochaine page contient le listage du code ayant effectué les calculs de
$f_1$, $f_2$, $f_3$ et $f_4$ en Haskell. Le temps d'exécution fut pratiquement
instantanné pour les paramètres de ce problème.

\pagebreak
\inputminted[linenos]{haskell}{groupe/src/Main.hs}

\pagebreak
\section{Au confluent de deux rivières}
Soit les événements
\begin{table}[H]
    \centering
    \begin{tabular}{l}
        A: la zone 1 n'est pas inondée\\
        B: la zone 2 n'est pas inondée\\
        C: la zone 3 n'est pas inondée\\
        D: au moins une zone est inondée\\
    \end{tabular}
\end{table}

On peut définir la probabilité d'avoir au moins une zone inondée avec
son complément, c'est-à-dire qu'il n'y ait aucune zone d'inondée, soit
\begin{equation*}
    \mathbb{P}(D)=1-\mathbb{P}(D^c)=1-\mathbb{P}(A\cap B\cap C).
\end{equation*}

Hors, la probabilité que la zone 3 ne soit pas inondée est équivalente
à la probabilité que son niveau d'eau $h_T$ soit inférieure à $H$, soit
\begin{equation}\label{eq:int_ft}
    \mathbb{P}(C)=\mathbb{P}(h_T\leq H)=F_T(H)=\int_{-\infty}^Hf_T(t)\d{t},
\end{equation}
où $F_T$ est la fonction de répartition de $h_T(h)$ et $f_T(h)$ sa fonction
de densité de probabilité. De plus, le niveau d'eau à la zone 3 dépend
du niveau d'eau $h_1$ à la zone 1, $h_2$ à la zone 2 et d'une valeur
$h_3$ selon
\begin{equation*}
    h_T=Mh_1+Nh_2+h_3.
\end{equation*}

Les valeurs $h_1$, $h_2$ et $h_3$ sont des variables aléatoires uniformes
telles que $h_1\in\mathbb{D}_1=[I,J]$, $h_2\in\mathbb{D}_2=[K,L]$ et
$h_3\in\mathbb{D}_3=[O,P]$. On peut définir
\begin{equation*}\label{eq:h345}
    h_T=h_4+h_5+h_3,
\end{equation*}
où $h_4$ et $h_5$ sont aussi des variables aléatoires uniformes telles que
$h_4\in\mathbb{D}_4=[M\cdot I,M\cdot J]$ et $h_5\in\mathbb{D}_5=
[N\cdot K,N\cdot L]$.

On cherche la fonction de densité de probabilité $f_T(h)$. On suppose que
$h_T=h$, $h_4=x$ et $h_5=y$. L'équation \eqref{eq:h345} est valide si est
seulement si $h_3=h_T-x-y$. La probabilité d'avoir $h_T=h$ est équivalente
à la probabilité d'avoir ces trois conditions, c'est-à-dire
\begin{equation*}
    \mathbb{P}(h_T=h)=
    \mathbb{P}(
        \left\{h_4=x\right\}\cap
        \left\{h_5=y\right\}\cap
        \left\{h_3=h-x-y\right\}).
\end{equation*}
Puisque les variables $h_3$, $h_4$ et $h_5$ prennent des valeurs 
indépendamments l'un de l'autre, il en résulte qu'on peut écrire
\begin{equation}\label{eq:hth}
    \begin{split}
        \mathbb{P}(h_T=h)&=
        \mathbb{P}(h_4=x)
        \mathbb{P}(h_5=y)
        \mathbb{P}(h_3=h-x-y)\\&=
        f_4(x)f_5(y)f_3(h-x-y),
    \end{split}
\end{equation}
où $f_3(h)$, $f_4(h)$ et $f_5(h)$ sont respectivement les fonctions de
densité de probabiltié de $h_3$, $h_4$ et $h_5$ définies par
\begin{equation*}
    f_3(h)=\left\{
        \begin{matrix}
            \nicefrac{1}{(P-O)}&\text{si } h\in\mathbb{D}_3\\
            0                  &\text{sinon}\\
        \end{matrix}
    \right.,
\end{equation*}
\begin{equation*}
    f_4(h)=\left\{
        \begin{matrix}
            \nicefrac{1}{M(J-I)}&\text{si } h\in\mathbb{D}_4\\
            0                   &\text{sinon}\\
        \end{matrix}
    \right.
\end{equation*}
et
\begin{equation*}
    f_5(h)=\left\{
        \begin{matrix}
            \nicefrac{1}{N(L-K)}&\text{si } h\in\mathbb{D}_5\\
            0                   &\text{sinon}\\
        \end{matrix}
    \right..
\end{equation*}

Hors, l'équation \eqref{eq:hth} est valide pour tout $x,y\in\mathbb{R}$ de
sorte que la probabilité d'obtenir $h_T=h$ est la somme
\begin{equation*}
    \mathbb{P}(h_T=h)=\sum_{\forall x}\sum_{\forall y}f_4(x)f_5(y)f_3(h-x-y),
\end{equation*}
ou plus mathématiquement correcte sous la forme d'intégrale,
\begin{equation*}
    \mathbb{P}(h_T=h)=
    \int_{-\infty}^\infty\int_{-\infty}^\infty
        f_4(x)f_5(y)f_3(h-x-y)\d{x}\d{y},
\end{equation*}
car $x$ et $y$ sont continues.

On sait que $f_4(x)=0$ si $x\in\notin\mathbb{D}_4$ et $f_5(y)=0$ si
$y\in\notin\mathbb{D}_5$ de sorte qu'on peut écrire
\begin{equation*}
    \begin{split}
        \mathbb{P}(h_T=h)&=
        \int_{\mathbb{D}_5}\int_{\mathbb{D}_4}
            f_4(x)f_5(y)f_3(h-x-y)\d{x}\d{y}\\&=
        \int_{\mathbb{D}_5}\int_{\mathbb{D}_4}
            \left(\frac{1}{M(J-I)}\right)
            \left(\frac{1}{N(L-K)}\right)
            f_3(h-x-y)\d{x}\d{y}\\&=
        \left(\frac{1}{M(J-I)}\right)
        \left(\frac{1}{N(L-K)}\right)
        \int_{\mathbb{D}_5}\int_{\mathbb{D}_4}
            f_3(h-x-y)\d{x}\d{y}.
    \end{split}
\end{equation*}

Même avec les paramètres biens définis, il est difficile d'intégrer cette
fonction en raison de la définition par partie de $f_3$. Une intégration
numérique est idéale. La figure \ref{fig:graphique_ft} montre le graphique
de $f_T$ calculée numériquement.

\begin{figure}[H]
    \def\prefixf{}
    \centering
    \documentclass[11pt]{article}
\usepackage[french]{babel}
\usepackage[utf8]{inputenc}
\usepackage[T1]{fontenc}
\usepackage{standalone}
\usepackage{tikz}
\usepackage{pgfplots}

\def\prefixf{../}

\begin{document}
\begin{tikzpicture}[scale=0.9,trim axis left]
    \begin{axis}[
        xlabel=$h$,
        ylabel=$f_T(h)$,
        scaled ticks=false,
        tick label style={/pgf/number format/fixed},
        xmin=2,xmax=4,
        width=9cm,
        height=5cm]
    \addplot[black] table 
        [x=h,y=f,col sep=comma,mark=none]
        {\prefixf integrate/data.csv};
    \end{axis}
\end{tikzpicture}
\end{document}




    \caption{graphique de $f_T(h)$ avec les paramètres du problème}
    \label{fig:graphique_ft}
\end{figure}

L'aire sous sa courbe fut aussi approximée jusqu'à une précision de
\SI{1.000000}{}, confirmant validité de la démarche. De plus, on remarque
$f_T(h)=0$ lorsque $h_3$, $h_4$ et $h_5$ sont inférieurs à leur valeur
minimum ou lorsqu'ils sont supérieurs à leur maximum.

Par définition, $f_T(h)=\mathbb{P}(h_t=h)$ de sorte qu'il est maintenant
possible d'évaluer $F_T(h)$ avec \eqref{eq:int_ft}, c'est-à-dire la
probabilité que la zone 3 ne soit pas inondée. Hors, on cherche la
probabilité que les trois zones ne sont pas inondées.

Par conséquent, la hauteur de la rivière à chaque zone ne doit pas
dépasser $H$, c'est-à-dire $h_1\in\mathbb{H}_1=[I,H]$,
$h_2\in\mathbb{H}_1=[K,H]$ et $h_T\leq H$. On obtient ainsi que
$h_4\in\mathbb{H}_1=[M\cdot I,M\cdot H]$ et $h_4\in\mathbb{H}_1=
[N\cdot K,N\cdot H]$ de sorte que la probabilité que $h_T=h$ et que
les deux autres zones ne sont pas inondées peut se calculer avec 
\begin{gather*}\mathclap{
    \mathbb{P}(
        \left\{h_T=h\right\}\cap
        \left\{h_1\leq H\right\}\cap
        \left\{h_2\leq H\right\})=
    \left(\frac{1}{M(J-I)}\right)
    \left(\frac{1}{N(L-K)}\right)
    \int_{\mathbb{H}_5}\int_{\mathbb{H}_4}
        f_3(h-x-y)\d{x}\d{y}.}
\end{gather*}

En d'autres termes, la dernière équation calcule la probabilité que
$h_T=h$ tout en ignorant les cas où les deux autres rivières sont inondées.
On s'intérèsse lorsque $h_T\leq H$. En intégrant, on obtient
\begin{equation*}
    \mathbb{P}(A\cap B\cap C)=
    \left(\frac{1}{M(J-I)}\right)
    \left(\frac{1}{N(L-K)}\right)
    \int_{-\infty}^H
    \int_{\mathbb{H}_5}\int_{\mathbb{H}_4}
        f_3(h-x-y)\d{x}\d{y}\d{h}.
\end{equation*}

On peut facilement calculer les valeurs possibles de $h_T$ en utilisant
les cas minimums et maximus de $h_4$, $h_5$ et $h_3$ de sorte que
\begin{equation*}
    \mathbb{P}(A\cap B\cap C)=
    \left(\frac{1}{M(J-I)}\right)
    \left(\frac{1}{N(L-K)}\right)
    \int_{h_\text{min}}^H\int_{\mathbb{H}_5}\int_{\mathbb{H}_4}
        f_3(h-x-y)\d{x}\d{y}\d{h},
\end{equation*}
où $h_\text{min}=M\cdot I+N\cdot K+O$.

Par intégration numérique, on obtient que $\mathbb{P}(A\cap B\cap C)\approx
\SI{0.432835}{}$ de sorte que la probabilité qu'au moins une zone soit
inondée est $1-0.432835\approx\SI{57}{\percent}$. La prochaine page contient
le code effectuant le calcul ainsi qu'une simulation numérique du problème.
Les deux résultats semblent être en accord ensemble.

\pagebreak
\newgeometry{left=25mm,right=12mm,top=20mm}
\begin{multicols}{2}
    \inputminted[linenos,fontsize=\footnotesize]{C}{integrate/main.c}
\end{multicols}
\restoregeometry

\section{Chasse à l'agent double}
Pour chaque personne dans l'équipe, on définie des événements pour chaque
classe possible, soit
\begin{table}[H]
    \centering
    \begin{tabular}{l}
        $A$: la personne est une recrue\\
        $B$: la personne est un agent régulier\\
        $C$: la personne est un agent senior\\
    \end{tabular}
\end{table}\noindent
avec
$\P(A)=\nicefrac{20}{67}$,
$\P(B)=\nicefrac{22}{67}$ et
$\P(C)=\nicefrac{25}{67}$,
les probabilités qu'une personne fasse partie d'un groupe.

Dans les 67 agents, il y a un agent double. Soit l'événement
\begin{table}[H]
    \centering
    \begin{tabular}{l}
        $D$: la personne est l'agent double\\
    \end{tabular}
\end{table}\noindent
de sorte que $\P(D)=\nicefrac{1}{67}$. Le directeur estime que
$\P(A|D)=0.51$,
$\P(B|D)=0.32$ et
$\P(C|D)=0.17$.

Pour l'interrogation, on définie les événements
\begin{table}[H]
    \centering
    \begin{tabular}{l}
        $I$: la personne est interrogée\\
        $T$: l'interrogation révèle un agent double\\
        $L$: l'interrogation révèle un agent loyal\\
    \end{tabular}
\end{table}\noindent
avec $\P(T^c)\neq\P(L)$, car on peut voir ces événements comme deux
tests à part à l'intérieur de l'interrogation. Les probabilités
d'identifier correctement les agents sont données par
$\P(D|T\cap I\cap A)=0.96$,
$\P(D|T\cap I\cap B)=0.48$ et
$\P(D|T\cap I\cap C)=0.24$,
tandis que celles d'identifier correctement un agent lotal sont
$\P(D^c|L\cap I\cap A)=0.84$,
$\P(D^c|L\cap I\cap B)=0.92$ et
$\P(D^c|L\cap I\cap C)=0.96$.

On sait que l'interrogation a révélée un seul agent double dans parmi
les 20 recrues. La probabilité que l'interrogation en révèle qu'un seul
agent parmi ceux-ci est donnée par
\begin{equation*}
    \P\left[
        (L_1\cap I_1\cap A_1)\cap
        (L_2\cap I_2\cap A_2)\cap\cdots\cap
        (L_{17}\cap I_{17}\cap A_{17})\cap
        (T_{18}\cap I_{18}\cap A_{18})
    \right],
\end{equation*}
où les indices dénotes un agent interrogé. Puisqu'il sont tous interrogés
indépendaments, on a
\begin{equation*}
    \P(L_1\cap I_1\cap A_1)
    \P(L_2\cap I_2\cap A_2)\cdots
    \P(L_{17}\cap I_{17}\cap A_{17})
    \P(T_{18}\cap I_{18}\cap A_{18}),
\end{equation*}
ou encore
\begin{equation*}
    \P(Z_A)=
    \P(L_i\cap I_i\cap A_i)^{17}
    \P(T_i\cap I_i\cap A_i),
\end{equation*}
où $Z_A$ est l'événement tel que l'interrogatoire a révélé un agent double
parmi les recrues. Par le même raisonement, on a aussi
\begin{equation*}
    \P(Z_B)=
    \P(L_i\cap I_i\cap B_i)^{19}
    \P(T_i\cap I_i\cap B_i)
\end{equation*}
et
\begin{equation*}
    \P(Z_B)=
    \P(L_i\cap I_i\cap C_i)^{21}
    \P(T_i\cap I_i\cap C_i),
\end{equation*}
pour les agents réguliers et seniors.

On cherche les probabilités que soit la recrue, soir l'agent régulier ou
l'agent senior soit l'agent double. Par conséquent, on cherche $\P(D|Z_A)$,
$\P(D|Z_B)$ et $\P(D|Z_C)$. Dans le cas où aucun de ces agents n'est le vrai
agent double, il faut calculer le complément des trois cas, soit $1-\P(D|Z_A)-
\P(D|Z_B)-\P(D|Z_C)$.

\pagebreak
\section*{Annexe}
\begin{table}[H]
    \centering
	\caption{table des valeurs}
    \documentclass[11pt]{article}
\usepackage[french]{babel}
\usepackage[utf8]{inputenc}
\usepackage[T1]{fontenc}
\usepackage{siunitx}
\usepackage{booktabs}

\begin{document}
\begin{tabular}{cS[table-format=5.3,table-auto-round]ccS[table-format=5.3,table-auto-round]}
    \toprule
        lettre & {nombre} && lettre & {nombre} \\
    \midrule
        A	&	0.71000004	&&	L	&	3.37\\
        B	&	0.71500003	&&	M	&	2.77\\
        C	&	2.95	&&	N	&	0.75\\
        D	&	1.59	&&	O	&	1.875\\
        E	&	0.77	&&	P	&	43.0\\
        F	&	1.78	&&	Q	&	4.67\\
        G	&	1.535	&&	R	&	5.4\\
        H	&	59.0	&&	S	&	1.87\\
        I	&	1.16	&&	T	&	2.6399999\\
        J	&	0.265	&&	U	&	0.22999999\\
        K	&	1.76	&&		&	\\
    \bottomrule
\end{tabular}
\end{document}

\end{table}
\end{document}


