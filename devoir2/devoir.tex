\documentclass[11pt]{article}
\usepackage[french]{babel}
\usepackage[utf8]{inputenc}
\usepackage[T1]{fontenc}
\usepackage{standalone}
\usepackage{geometry}
\usepackage{gphys}
\usepackage{siunitx}
\usepackage{amsmath}
\usepackage{amsthm}
\usepackage{titlesec}
\usepackage{amsfonts}
\usepackage{mathtools}
\usepackage{booktabs}
\usepackage{float}
\usepackage{tikz}
\usepackage{relsize}
\usepackage{nicefrac}
\usepackage{multicol}

\usetikzlibrary{decorations.pathreplacing}
\usetikzlibrary{arrows}
\usetikzlibrary{calc}
\usetikzlibrary{babel}

%%%%%%%%%%%%%%%%%%
% Configurations %
%%%%%%%%%%%%%%%%%%

\geometry{
	total={210mm,297mm},
	left=34mm,
	right=34mm,
	top=30mm,
	bottom=30mm,
}

\sisetup{locale = FR}

\makeatletter
\let\@afterindenttrue\@afterindentfalse
\makeatother

% on met les sous-sections dans la marge
\titleformat{\subsection}[leftmargin]{\bfseries\filleft}{}{0pt}{}
\titlespacing{\subsection}{4pc}{1.5ex plus .1ex minus .2ex}{1pc}

%%%%%%%%%%
% Macros %
%%%%%%%%%%

\newtheorem{theoreme}{Théorème}

\newcommand\comb{%
    \ensuremath{%
        \mathcal{C}%
    }%
}

\newcommand\perm{%
    \ensuremath{%
        \mathcal{P}%
    }%
}

\renewcommand\P[1]{%
	\ensuremath{%
		\mathbb{P}\left[#1\right]%
	}%
}%

\newcommand\Pg[2]{%
	\ensuremath{%
		\mathbb{P}\left[#1\middle\rvert#2\right]%
	}%
}%

\newcommand\Esp[1]{%
	\ensuremath{%
		\mathrm{E}\left[#1\right]%
	}%
}%

\newcommand\Var[1]{%
	\ensuremath{%
		\mathrm{Var}\left[#1\right]%
	}%
}%

\newcommand\Espg[2]{%
	\ensuremath{%
		\mathrm{E}\left[#1\middle\rvert#2\right]%
	}%
}%

\newcommand\Norm[2]{%
	\ensuremath{%
		\mathcal{N}\left(#1,#2\right)%
	}%
}%

\newcommand\Bin[2]{%
	\ensuremath{%
		\mathcal{B}\left(#1,#2\right)%
	}%
}%

\newcommand\Uni[2]{%
	\ensuremath{%
		\mathcal{U}\left(#1,#2\right)%
	}%
}%

\newcommand\Geo[1]{%
	\ensuremath{%
		\mathcal{G}\left(#1\right)%
	}%
}%

\newcommand\Poi[1]{%
	\ensuremath{%
		\mathrm{Poi}\left(#1\right)%
	}%
}%

\newcommand\Max[2]{%
	\ensuremath{%
		\mathrm{max}\left(#1,#2\right)%
	}%
}%

\newcommand\e{%
    \ensuremath{%
        \mathrm{e}%
    }%
}

%%%%%%%%%%%%%
% Documents %
%%%%%%%%%%%%%

\begin{document}
\def\Sigle{MTH2302A}
\def\Cours{Probabilités et Statistique}
\def\Devoir{Devoir 2}
\AjouterAuteur{Gabriel-Andrew}{Pollo-Guilbert}{1837776}
\AjouterProfesseur{}{Simon}{Demontigny}
\def\logoscale{1}
\pagetitre

\section*{Question 1}
\subsection{a)}
Soit $T$ la durée de vie d'un composant $k$ avec comme fonction de densité
de probabilités
\begin{equation*}
    f_T(t)=\left\{
        \begin{matrix}
            A^2t\e^{-At} & \text{si} & t\geq 0,\\
            0            & \text{sinon},
        \end{matrix}
    \right.
\end{equation*}
où $A$ est une constante positive.

On obtient la fonction de répartition en intégrant. Pour $x\geq 0$, on
obtient par intégration par partie :
\begin{equation*}
    F_T(t)
    =\int_0^tf(s)\d{s}
    =A^2\int_0^ts\e^{-As}\d{s}
    =1-(At+1)\e^{-At},
\end{equation*}
de sorte à obtenir la fonction de fiabilité d'un composant, soit
\begin{equation*}
    R_k(t)
    =1-F_T(t)
    =\left\{
        \begin{matrix}
            (At+1)\e^{-At} & \text{si} & t\geq 0\\
            1              & \text{sinon}
        \end{matrix}
    \right.,
\end{equation*}
car $F_T(t)=0$ pour $x<0$.

La figure \ref{fig:sys_serie} montre un système contenant $m$ de
ces composants en série.
\begin{figure}[H]
    \centering
    \documentclass[11pt]{article}
\usepackage[french]{babel}
\usepackage[utf8]{inputenc}
\usepackage[T1]{fontenc}
\usepackage{standalone}
\usepackage{tikz}

\usetikzlibrary{babel}

\begin{document}
\begin{tikzpicture}[
    every node/.style={
        draw,
        minimum width=2cm,
        minimum height=1cm
    }
]

\node            (C1) at (0,0) {$1$};
\node            (C2) at (3,0) {$2$};
\node[draw=none] (CD) at (6,0) {$\cdots$};
\node            (CM) at (9,0) {$m$};

\draw (-2,0) -- (C1) -- (C2) -- (CD) -- (CM) -- (11,0);

\end{tikzpicture}
\end{document}




    \caption{système en série à $m$ composants}
    \label{fig:sys_serie}
\end{figure}

Soit $X$ la durée de vie du système en entier dont la fonction de fiabilité
est donnée par
\begin{equation*}
    R(x)
    =\prod_{k=1}^mR_k(x)
    =\bigg[R_k(x)\bigg]^m
    =\bigg[(Ax+1)\e^{-Ax}\bigg]^m
    =(Ax+1)^m\e^{-Amx},
\end{equation*}
pour $x\geq 0$, car les composants sont indépendant.

La durée de vie moyenne de ce système est l'espérance de $X$, soit
\begin{equation*}
    \Esp{X}
    =\int_{-\infty}^\infty R(t)\d{x}
    =\int_0^\infty(Ax+1)^m\e^{-Amx}\d{x}
\end{equation*}
En posant $u=Ax+1\Leftrightarrow u-1=Ax$ et $\d{u}=A\d{x}$, on obtient
\begin{equation*}
    \Esp{X}
    =\frac{1}{A}\int_0^\infty u^m\e^{-(u-1)m}\d{u}
    =\frac{1}{A}\int_0^\infty u^m\e^{-um+m}\d{u}
    =\frac{\e^{m}}{A}\int_0^\infty u^m\e^{-um}\d{u}
\end{equation*}

Hors, l'ingénieur estime que la durée de vie moyenne du système est $B$
de sorte qu'on obtient
\begin{equation*}
    B=\frac{\e^{m}}{A}\cdot\frac{m!}{m^{m+1}}\Leftrightarrow
    AB=\frac{m!}{m^{m+1}}\e^{m},
\end{equation*}
avec le théorème \ref{th:integrale} présenté en annexe.

Dans notre situation, $A=0.710$ et $B=715$. Puisqu'on cherche une valeur
$m$ entière et positive, il es possible de la déterminer en regardant
simplement un graphique. La figure \ref{fig:graph_solve} montre l'équation.

\begin{figure}[H]
    \centering
    \caption{Graphique de l'équation $AB=\dfrac{m!}{m^{m+1}}\e^{m}$}
    %% Creator: Matplotlib, PGF backend
%%
%% To include the figure in your LaTeX document, write
%%   \input{<filename>.pgf}
%%
%% Make sure the required packages are loaded in your preamble
%%   \usepackage{pgf}
%%
%% Figures using additional raster images can only be included by \input if
%% they are in the same directory as the main LaTeX file. For loading figures
%% from other directories you can use the `import` package
%%   \usepackage{import}
%% and then include the figures with
%%   \import{<path to file>}{<filename>.pgf}
%%
%% Matplotlib used the following preamble
%%   \usepackage{fontspec}
%%   \setmainfont{DejaVu Serif}
%%   \setmonofont{DejaVu Sans Mono}
%%
\begingroup%
\makeatletter%
\begin{pgfpicture}%
\pgfpathrectangle{\pgfpointorigin}{\pgfqpoint{4.143065in}{3.245112in}}%
\pgfusepath{use as bounding box, clip}%
\begin{pgfscope}%
\pgfsetbuttcap%
\pgfsetmiterjoin%
\pgfsetlinewidth{0.000000pt}%
\definecolor{currentstroke}{rgb}{0.000000,0.000000,0.000000}%
\pgfsetstrokecolor{currentstroke}%
\pgfsetstrokeopacity{0.000000}%
\pgfsetdash{}{0pt}%
\pgfpathmoveto{\pgfqpoint{0.000000in}{0.000000in}}%
\pgfpathlineto{\pgfqpoint{4.143065in}{0.000000in}}%
\pgfpathlineto{\pgfqpoint{4.143065in}{3.245112in}}%
\pgfpathlineto{\pgfqpoint{0.000000in}{3.245112in}}%
\pgfpathclose%
\pgfusepath{}%
\end{pgfscope}%
\begin{pgfscope}%
\pgfsetbuttcap%
\pgfsetmiterjoin%
\pgfsetlinewidth{0.000000pt}%
\definecolor{currentstroke}{rgb}{0.000000,0.000000,0.000000}%
\pgfsetstrokecolor{currentstroke}%
\pgfsetstrokeopacity{0.000000}%
\pgfsetdash{}{0pt}%
\pgfpathmoveto{\pgfqpoint{0.495024in}{0.499875in}}%
\pgfpathlineto{\pgfqpoint{3.967024in}{0.499875in}}%
\pgfpathlineto{\pgfqpoint{3.967024in}{3.087075in}}%
\pgfpathlineto{\pgfqpoint{0.495024in}{3.087075in}}%
\pgfpathclose%
\pgfusepath{}%
\end{pgfscope}%
\begin{pgfscope}%
\pgfsetbuttcap%
\pgfsetroundjoin%
\definecolor{currentfill}{rgb}{0.000000,0.000000,0.000000}%
\pgfsetfillcolor{currentfill}%
\pgfsetlinewidth{0.501875pt}%
\definecolor{currentstroke}{rgb}{0.000000,0.000000,0.000000}%
\pgfsetstrokecolor{currentstroke}%
\pgfsetdash{}{0pt}%
\pgfsys@defobject{currentmarker}{\pgfqpoint{0.000000in}{0.000000in}}{\pgfqpoint{0.000000in}{0.048611in}}{%
\pgfpathmoveto{\pgfqpoint{0.000000in}{0.000000in}}%
\pgfpathlineto{\pgfqpoint{0.000000in}{0.048611in}}%
\pgfusepath{stroke,fill}%
}%
\begin{pgfscope}%
\pgfsys@transformshift{0.495024in}{0.499875in}%
\pgfsys@useobject{currentmarker}{}%
\end{pgfscope}%
\end{pgfscope}%
\begin{pgfscope}%
\pgftext[x=0.495024in,y=0.451264in,,top]{\rmfamily\fontsize{11.000000}{13.200000}\selectfont \(\displaystyle 20\)}%
\end{pgfscope}%
\begin{pgfscope}%
\pgfsetbuttcap%
\pgfsetroundjoin%
\definecolor{currentfill}{rgb}{0.000000,0.000000,0.000000}%
\pgfsetfillcolor{currentfill}%
\pgfsetlinewidth{0.501875pt}%
\definecolor{currentstroke}{rgb}{0.000000,0.000000,0.000000}%
\pgfsetstrokecolor{currentstroke}%
\pgfsetdash{}{0pt}%
\pgfsys@defobject{currentmarker}{\pgfqpoint{0.000000in}{0.000000in}}{\pgfqpoint{0.000000in}{0.048611in}}{%
\pgfpathmoveto{\pgfqpoint{0.000000in}{0.000000in}}%
\pgfpathlineto{\pgfqpoint{0.000000in}{0.048611in}}%
\pgfusepath{stroke,fill}%
}%
\begin{pgfscope}%
\pgfsys@transformshift{0.842224in}{0.499875in}%
\pgfsys@useobject{currentmarker}{}%
\end{pgfscope}%
\end{pgfscope}%
\begin{pgfscope}%
\pgftext[x=0.842224in,y=0.451264in,,top]{\rmfamily\fontsize{11.000000}{13.200000}\selectfont \(\displaystyle 21\)}%
\end{pgfscope}%
\begin{pgfscope}%
\pgfsetbuttcap%
\pgfsetroundjoin%
\definecolor{currentfill}{rgb}{0.000000,0.000000,0.000000}%
\pgfsetfillcolor{currentfill}%
\pgfsetlinewidth{0.501875pt}%
\definecolor{currentstroke}{rgb}{0.000000,0.000000,0.000000}%
\pgfsetstrokecolor{currentstroke}%
\pgfsetdash{}{0pt}%
\pgfsys@defobject{currentmarker}{\pgfqpoint{0.000000in}{0.000000in}}{\pgfqpoint{0.000000in}{0.048611in}}{%
\pgfpathmoveto{\pgfqpoint{0.000000in}{0.000000in}}%
\pgfpathlineto{\pgfqpoint{0.000000in}{0.048611in}}%
\pgfusepath{stroke,fill}%
}%
\begin{pgfscope}%
\pgfsys@transformshift{1.189424in}{0.499875in}%
\pgfsys@useobject{currentmarker}{}%
\end{pgfscope}%
\end{pgfscope}%
\begin{pgfscope}%
\pgftext[x=1.189424in,y=0.451264in,,top]{\rmfamily\fontsize{11.000000}{13.200000}\selectfont \(\displaystyle 22\)}%
\end{pgfscope}%
\begin{pgfscope}%
\pgfsetbuttcap%
\pgfsetroundjoin%
\definecolor{currentfill}{rgb}{0.000000,0.000000,0.000000}%
\pgfsetfillcolor{currentfill}%
\pgfsetlinewidth{0.501875pt}%
\definecolor{currentstroke}{rgb}{0.000000,0.000000,0.000000}%
\pgfsetstrokecolor{currentstroke}%
\pgfsetdash{}{0pt}%
\pgfsys@defobject{currentmarker}{\pgfqpoint{0.000000in}{0.000000in}}{\pgfqpoint{0.000000in}{0.048611in}}{%
\pgfpathmoveto{\pgfqpoint{0.000000in}{0.000000in}}%
\pgfpathlineto{\pgfqpoint{0.000000in}{0.048611in}}%
\pgfusepath{stroke,fill}%
}%
\begin{pgfscope}%
\pgfsys@transformshift{1.536624in}{0.499875in}%
\pgfsys@useobject{currentmarker}{}%
\end{pgfscope}%
\end{pgfscope}%
\begin{pgfscope}%
\pgftext[x=1.536624in,y=0.451264in,,top]{\rmfamily\fontsize{11.000000}{13.200000}\selectfont \(\displaystyle 23\)}%
\end{pgfscope}%
\begin{pgfscope}%
\pgfsetbuttcap%
\pgfsetroundjoin%
\definecolor{currentfill}{rgb}{0.000000,0.000000,0.000000}%
\pgfsetfillcolor{currentfill}%
\pgfsetlinewidth{0.501875pt}%
\definecolor{currentstroke}{rgb}{0.000000,0.000000,0.000000}%
\pgfsetstrokecolor{currentstroke}%
\pgfsetdash{}{0pt}%
\pgfsys@defobject{currentmarker}{\pgfqpoint{0.000000in}{0.000000in}}{\pgfqpoint{0.000000in}{0.048611in}}{%
\pgfpathmoveto{\pgfqpoint{0.000000in}{0.000000in}}%
\pgfpathlineto{\pgfqpoint{0.000000in}{0.048611in}}%
\pgfusepath{stroke,fill}%
}%
\begin{pgfscope}%
\pgfsys@transformshift{1.883824in}{0.499875in}%
\pgfsys@useobject{currentmarker}{}%
\end{pgfscope}%
\end{pgfscope}%
\begin{pgfscope}%
\pgftext[x=1.883824in,y=0.451264in,,top]{\rmfamily\fontsize{11.000000}{13.200000}\selectfont \(\displaystyle 24\)}%
\end{pgfscope}%
\begin{pgfscope}%
\pgfsetbuttcap%
\pgfsetroundjoin%
\definecolor{currentfill}{rgb}{0.000000,0.000000,0.000000}%
\pgfsetfillcolor{currentfill}%
\pgfsetlinewidth{0.501875pt}%
\definecolor{currentstroke}{rgb}{0.000000,0.000000,0.000000}%
\pgfsetstrokecolor{currentstroke}%
\pgfsetdash{}{0pt}%
\pgfsys@defobject{currentmarker}{\pgfqpoint{0.000000in}{0.000000in}}{\pgfqpoint{0.000000in}{0.048611in}}{%
\pgfpathmoveto{\pgfqpoint{0.000000in}{0.000000in}}%
\pgfpathlineto{\pgfqpoint{0.000000in}{0.048611in}}%
\pgfusepath{stroke,fill}%
}%
\begin{pgfscope}%
\pgfsys@transformshift{2.231024in}{0.499875in}%
\pgfsys@useobject{currentmarker}{}%
\end{pgfscope}%
\end{pgfscope}%
\begin{pgfscope}%
\pgftext[x=2.231024in,y=0.451264in,,top]{\rmfamily\fontsize{11.000000}{13.200000}\selectfont \(\displaystyle 25\)}%
\end{pgfscope}%
\begin{pgfscope}%
\pgfsetbuttcap%
\pgfsetroundjoin%
\definecolor{currentfill}{rgb}{0.000000,0.000000,0.000000}%
\pgfsetfillcolor{currentfill}%
\pgfsetlinewidth{0.501875pt}%
\definecolor{currentstroke}{rgb}{0.000000,0.000000,0.000000}%
\pgfsetstrokecolor{currentstroke}%
\pgfsetdash{}{0pt}%
\pgfsys@defobject{currentmarker}{\pgfqpoint{0.000000in}{0.000000in}}{\pgfqpoint{0.000000in}{0.048611in}}{%
\pgfpathmoveto{\pgfqpoint{0.000000in}{0.000000in}}%
\pgfpathlineto{\pgfqpoint{0.000000in}{0.048611in}}%
\pgfusepath{stroke,fill}%
}%
\begin{pgfscope}%
\pgfsys@transformshift{2.578224in}{0.499875in}%
\pgfsys@useobject{currentmarker}{}%
\end{pgfscope}%
\end{pgfscope}%
\begin{pgfscope}%
\pgftext[x=2.578224in,y=0.451264in,,top]{\rmfamily\fontsize{11.000000}{13.200000}\selectfont \(\displaystyle 26\)}%
\end{pgfscope}%
\begin{pgfscope}%
\pgfsetbuttcap%
\pgfsetroundjoin%
\definecolor{currentfill}{rgb}{0.000000,0.000000,0.000000}%
\pgfsetfillcolor{currentfill}%
\pgfsetlinewidth{0.501875pt}%
\definecolor{currentstroke}{rgb}{0.000000,0.000000,0.000000}%
\pgfsetstrokecolor{currentstroke}%
\pgfsetdash{}{0pt}%
\pgfsys@defobject{currentmarker}{\pgfqpoint{0.000000in}{0.000000in}}{\pgfqpoint{0.000000in}{0.048611in}}{%
\pgfpathmoveto{\pgfqpoint{0.000000in}{0.000000in}}%
\pgfpathlineto{\pgfqpoint{0.000000in}{0.048611in}}%
\pgfusepath{stroke,fill}%
}%
\begin{pgfscope}%
\pgfsys@transformshift{2.925424in}{0.499875in}%
\pgfsys@useobject{currentmarker}{}%
\end{pgfscope}%
\end{pgfscope}%
\begin{pgfscope}%
\pgftext[x=2.925424in,y=0.451264in,,top]{\rmfamily\fontsize{11.000000}{13.200000}\selectfont \(\displaystyle 27\)}%
\end{pgfscope}%
\begin{pgfscope}%
\pgfsetbuttcap%
\pgfsetroundjoin%
\definecolor{currentfill}{rgb}{0.000000,0.000000,0.000000}%
\pgfsetfillcolor{currentfill}%
\pgfsetlinewidth{0.501875pt}%
\definecolor{currentstroke}{rgb}{0.000000,0.000000,0.000000}%
\pgfsetstrokecolor{currentstroke}%
\pgfsetdash{}{0pt}%
\pgfsys@defobject{currentmarker}{\pgfqpoint{0.000000in}{0.000000in}}{\pgfqpoint{0.000000in}{0.048611in}}{%
\pgfpathmoveto{\pgfqpoint{0.000000in}{0.000000in}}%
\pgfpathlineto{\pgfqpoint{0.000000in}{0.048611in}}%
\pgfusepath{stroke,fill}%
}%
\begin{pgfscope}%
\pgfsys@transformshift{3.272624in}{0.499875in}%
\pgfsys@useobject{currentmarker}{}%
\end{pgfscope}%
\end{pgfscope}%
\begin{pgfscope}%
\pgftext[x=3.272624in,y=0.451264in,,top]{\rmfamily\fontsize{11.000000}{13.200000}\selectfont \(\displaystyle 28\)}%
\end{pgfscope}%
\begin{pgfscope}%
\pgfsetbuttcap%
\pgfsetroundjoin%
\definecolor{currentfill}{rgb}{0.000000,0.000000,0.000000}%
\pgfsetfillcolor{currentfill}%
\pgfsetlinewidth{0.501875pt}%
\definecolor{currentstroke}{rgb}{0.000000,0.000000,0.000000}%
\pgfsetstrokecolor{currentstroke}%
\pgfsetdash{}{0pt}%
\pgfsys@defobject{currentmarker}{\pgfqpoint{0.000000in}{0.000000in}}{\pgfqpoint{0.000000in}{0.048611in}}{%
\pgfpathmoveto{\pgfqpoint{0.000000in}{0.000000in}}%
\pgfpathlineto{\pgfqpoint{0.000000in}{0.048611in}}%
\pgfusepath{stroke,fill}%
}%
\begin{pgfscope}%
\pgfsys@transformshift{3.619824in}{0.499875in}%
\pgfsys@useobject{currentmarker}{}%
\end{pgfscope}%
\end{pgfscope}%
\begin{pgfscope}%
\pgftext[x=3.619824in,y=0.451264in,,top]{\rmfamily\fontsize{11.000000}{13.200000}\selectfont \(\displaystyle 29\)}%
\end{pgfscope}%
\begin{pgfscope}%
\pgfsetbuttcap%
\pgfsetroundjoin%
\definecolor{currentfill}{rgb}{0.000000,0.000000,0.000000}%
\pgfsetfillcolor{currentfill}%
\pgfsetlinewidth{0.501875pt}%
\definecolor{currentstroke}{rgb}{0.000000,0.000000,0.000000}%
\pgfsetstrokecolor{currentstroke}%
\pgfsetdash{}{0pt}%
\pgfsys@defobject{currentmarker}{\pgfqpoint{0.000000in}{0.000000in}}{\pgfqpoint{0.000000in}{0.048611in}}{%
\pgfpathmoveto{\pgfqpoint{0.000000in}{0.000000in}}%
\pgfpathlineto{\pgfqpoint{0.000000in}{0.048611in}}%
\pgfusepath{stroke,fill}%
}%
\begin{pgfscope}%
\pgfsys@transformshift{3.967024in}{0.499875in}%
\pgfsys@useobject{currentmarker}{}%
\end{pgfscope}%
\end{pgfscope}%
\begin{pgfscope}%
\pgftext[x=3.967024in,y=0.451264in,,top]{\rmfamily\fontsize{11.000000}{13.200000}\selectfont \(\displaystyle 30\)}%
\end{pgfscope}%
\begin{pgfscope}%
\pgftext[x=2.231024in,y=0.247854in,,top]{\rmfamily\fontsize{11.000000}{13.200000}\selectfont \(\displaystyle m\)}%
\end{pgfscope}%
\begin{pgfscope}%
\pgfsetbuttcap%
\pgfsetroundjoin%
\definecolor{currentfill}{rgb}{0.000000,0.000000,0.000000}%
\pgfsetfillcolor{currentfill}%
\pgfsetlinewidth{0.501875pt}%
\definecolor{currentstroke}{rgb}{0.000000,0.000000,0.000000}%
\pgfsetstrokecolor{currentstroke}%
\pgfsetdash{}{0pt}%
\pgfsys@defobject{currentmarker}{\pgfqpoint{0.000000in}{0.000000in}}{\pgfqpoint{0.048611in}{0.000000in}}{%
\pgfpathmoveto{\pgfqpoint{0.000000in}{0.000000in}}%
\pgfpathlineto{\pgfqpoint{0.048611in}{0.000000in}}%
\pgfusepath{stroke,fill}%
}%
\begin{pgfscope}%
\pgfsys@transformshift{0.495024in}{0.499875in}%
\pgfsys@useobject{currentmarker}{}%
\end{pgfscope}%
\end{pgfscope}%
\begin{pgfscope}%
\pgftext[x=0.100000in,y=0.441837in,left,base]{\rmfamily\fontsize{11.000000}{13.200000}\selectfont \(\displaystyle 0.400\)}%
\end{pgfscope}%
\begin{pgfscope}%
\pgfsetbuttcap%
\pgfsetroundjoin%
\definecolor{currentfill}{rgb}{0.000000,0.000000,0.000000}%
\pgfsetfillcolor{currentfill}%
\pgfsetlinewidth{0.501875pt}%
\definecolor{currentstroke}{rgb}{0.000000,0.000000,0.000000}%
\pgfsetstrokecolor{currentstroke}%
\pgfsetdash{}{0pt}%
\pgfsys@defobject{currentmarker}{\pgfqpoint{0.000000in}{0.000000in}}{\pgfqpoint{0.048611in}{0.000000in}}{%
\pgfpathmoveto{\pgfqpoint{0.000000in}{0.000000in}}%
\pgfpathlineto{\pgfqpoint{0.048611in}{0.000000in}}%
\pgfusepath{stroke,fill}%
}%
\begin{pgfscope}%
\pgfsys@transformshift{0.495024in}{0.823275in}%
\pgfsys@useobject{currentmarker}{}%
\end{pgfscope}%
\end{pgfscope}%
\begin{pgfscope}%
\pgftext[x=0.100000in,y=0.765237in,left,base]{\rmfamily\fontsize{11.000000}{13.200000}\selectfont \(\displaystyle 0.425\)}%
\end{pgfscope}%
\begin{pgfscope}%
\pgfsetbuttcap%
\pgfsetroundjoin%
\definecolor{currentfill}{rgb}{0.000000,0.000000,0.000000}%
\pgfsetfillcolor{currentfill}%
\pgfsetlinewidth{0.501875pt}%
\definecolor{currentstroke}{rgb}{0.000000,0.000000,0.000000}%
\pgfsetstrokecolor{currentstroke}%
\pgfsetdash{}{0pt}%
\pgfsys@defobject{currentmarker}{\pgfqpoint{0.000000in}{0.000000in}}{\pgfqpoint{0.048611in}{0.000000in}}{%
\pgfpathmoveto{\pgfqpoint{0.000000in}{0.000000in}}%
\pgfpathlineto{\pgfqpoint{0.048611in}{0.000000in}}%
\pgfusepath{stroke,fill}%
}%
\begin{pgfscope}%
\pgfsys@transformshift{0.495024in}{1.146675in}%
\pgfsys@useobject{currentmarker}{}%
\end{pgfscope}%
\end{pgfscope}%
\begin{pgfscope}%
\pgftext[x=0.100000in,y=1.088637in,left,base]{\rmfamily\fontsize{11.000000}{13.200000}\selectfont \(\displaystyle 0.450\)}%
\end{pgfscope}%
\begin{pgfscope}%
\pgfsetbuttcap%
\pgfsetroundjoin%
\definecolor{currentfill}{rgb}{0.000000,0.000000,0.000000}%
\pgfsetfillcolor{currentfill}%
\pgfsetlinewidth{0.501875pt}%
\definecolor{currentstroke}{rgb}{0.000000,0.000000,0.000000}%
\pgfsetstrokecolor{currentstroke}%
\pgfsetdash{}{0pt}%
\pgfsys@defobject{currentmarker}{\pgfqpoint{0.000000in}{0.000000in}}{\pgfqpoint{0.048611in}{0.000000in}}{%
\pgfpathmoveto{\pgfqpoint{0.000000in}{0.000000in}}%
\pgfpathlineto{\pgfqpoint{0.048611in}{0.000000in}}%
\pgfusepath{stroke,fill}%
}%
\begin{pgfscope}%
\pgfsys@transformshift{0.495024in}{1.470075in}%
\pgfsys@useobject{currentmarker}{}%
\end{pgfscope}%
\end{pgfscope}%
\begin{pgfscope}%
\pgftext[x=0.100000in,y=1.412037in,left,base]{\rmfamily\fontsize{11.000000}{13.200000}\selectfont \(\displaystyle 0.475\)}%
\end{pgfscope}%
\begin{pgfscope}%
\pgfsetbuttcap%
\pgfsetroundjoin%
\definecolor{currentfill}{rgb}{0.000000,0.000000,0.000000}%
\pgfsetfillcolor{currentfill}%
\pgfsetlinewidth{0.501875pt}%
\definecolor{currentstroke}{rgb}{0.000000,0.000000,0.000000}%
\pgfsetstrokecolor{currentstroke}%
\pgfsetdash{}{0pt}%
\pgfsys@defobject{currentmarker}{\pgfqpoint{0.000000in}{0.000000in}}{\pgfqpoint{0.048611in}{0.000000in}}{%
\pgfpathmoveto{\pgfqpoint{0.000000in}{0.000000in}}%
\pgfpathlineto{\pgfqpoint{0.048611in}{0.000000in}}%
\pgfusepath{stroke,fill}%
}%
\begin{pgfscope}%
\pgfsys@transformshift{0.495024in}{1.793475in}%
\pgfsys@useobject{currentmarker}{}%
\end{pgfscope}%
\end{pgfscope}%
\begin{pgfscope}%
\pgftext[x=0.100000in,y=1.735437in,left,base]{\rmfamily\fontsize{11.000000}{13.200000}\selectfont \(\displaystyle 0.500\)}%
\end{pgfscope}%
\begin{pgfscope}%
\pgfsetbuttcap%
\pgfsetroundjoin%
\definecolor{currentfill}{rgb}{0.000000,0.000000,0.000000}%
\pgfsetfillcolor{currentfill}%
\pgfsetlinewidth{0.501875pt}%
\definecolor{currentstroke}{rgb}{0.000000,0.000000,0.000000}%
\pgfsetstrokecolor{currentstroke}%
\pgfsetdash{}{0pt}%
\pgfsys@defobject{currentmarker}{\pgfqpoint{0.000000in}{0.000000in}}{\pgfqpoint{0.048611in}{0.000000in}}{%
\pgfpathmoveto{\pgfqpoint{0.000000in}{0.000000in}}%
\pgfpathlineto{\pgfqpoint{0.048611in}{0.000000in}}%
\pgfusepath{stroke,fill}%
}%
\begin{pgfscope}%
\pgfsys@transformshift{0.495024in}{2.116875in}%
\pgfsys@useobject{currentmarker}{}%
\end{pgfscope}%
\end{pgfscope}%
\begin{pgfscope}%
\pgftext[x=0.100000in,y=2.058837in,left,base]{\rmfamily\fontsize{11.000000}{13.200000}\selectfont \(\displaystyle 0.525\)}%
\end{pgfscope}%
\begin{pgfscope}%
\pgfsetbuttcap%
\pgfsetroundjoin%
\definecolor{currentfill}{rgb}{0.000000,0.000000,0.000000}%
\pgfsetfillcolor{currentfill}%
\pgfsetlinewidth{0.501875pt}%
\definecolor{currentstroke}{rgb}{0.000000,0.000000,0.000000}%
\pgfsetstrokecolor{currentstroke}%
\pgfsetdash{}{0pt}%
\pgfsys@defobject{currentmarker}{\pgfqpoint{0.000000in}{0.000000in}}{\pgfqpoint{0.048611in}{0.000000in}}{%
\pgfpathmoveto{\pgfqpoint{0.000000in}{0.000000in}}%
\pgfpathlineto{\pgfqpoint{0.048611in}{0.000000in}}%
\pgfusepath{stroke,fill}%
}%
\begin{pgfscope}%
\pgfsys@transformshift{0.495024in}{2.440275in}%
\pgfsys@useobject{currentmarker}{}%
\end{pgfscope}%
\end{pgfscope}%
\begin{pgfscope}%
\pgftext[x=0.100000in,y=2.382237in,left,base]{\rmfamily\fontsize{11.000000}{13.200000}\selectfont \(\displaystyle 0.550\)}%
\end{pgfscope}%
\begin{pgfscope}%
\pgfsetbuttcap%
\pgfsetroundjoin%
\definecolor{currentfill}{rgb}{0.000000,0.000000,0.000000}%
\pgfsetfillcolor{currentfill}%
\pgfsetlinewidth{0.501875pt}%
\definecolor{currentstroke}{rgb}{0.000000,0.000000,0.000000}%
\pgfsetstrokecolor{currentstroke}%
\pgfsetdash{}{0pt}%
\pgfsys@defobject{currentmarker}{\pgfqpoint{0.000000in}{0.000000in}}{\pgfqpoint{0.048611in}{0.000000in}}{%
\pgfpathmoveto{\pgfqpoint{0.000000in}{0.000000in}}%
\pgfpathlineto{\pgfqpoint{0.048611in}{0.000000in}}%
\pgfusepath{stroke,fill}%
}%
\begin{pgfscope}%
\pgfsys@transformshift{0.495024in}{2.763675in}%
\pgfsys@useobject{currentmarker}{}%
\end{pgfscope}%
\end{pgfscope}%
\begin{pgfscope}%
\pgftext[x=0.100000in,y=2.705637in,left,base]{\rmfamily\fontsize{11.000000}{13.200000}\selectfont \(\displaystyle 0.575\)}%
\end{pgfscope}%
\begin{pgfscope}%
\pgfsetbuttcap%
\pgfsetroundjoin%
\definecolor{currentfill}{rgb}{0.000000,0.000000,0.000000}%
\pgfsetfillcolor{currentfill}%
\pgfsetlinewidth{0.501875pt}%
\definecolor{currentstroke}{rgb}{0.000000,0.000000,0.000000}%
\pgfsetstrokecolor{currentstroke}%
\pgfsetdash{}{0pt}%
\pgfsys@defobject{currentmarker}{\pgfqpoint{0.000000in}{0.000000in}}{\pgfqpoint{0.048611in}{0.000000in}}{%
\pgfpathmoveto{\pgfqpoint{0.000000in}{0.000000in}}%
\pgfpathlineto{\pgfqpoint{0.048611in}{0.000000in}}%
\pgfusepath{stroke,fill}%
}%
\begin{pgfscope}%
\pgfsys@transformshift{0.495024in}{3.087075in}%
\pgfsys@useobject{currentmarker}{}%
\end{pgfscope}%
\end{pgfscope}%
\begin{pgfscope}%
\pgftext[x=0.100000in,y=3.029037in,left,base]{\rmfamily\fontsize{11.000000}{13.200000}\selectfont \(\displaystyle 0.600\)}%
\end{pgfscope}%
\begin{pgfscope}%
\pgfpathrectangle{\pgfqpoint{0.495024in}{0.499875in}}{\pgfqpoint{3.472000in}{2.587200in}} %
\pgfusepath{clip}%
\pgfsetrectcap%
\pgfsetroundjoin%
\pgfsetlinewidth{0.501875pt}%
\definecolor{currentstroke}{rgb}{0.000000,0.000000,0.000000}%
\pgfsetstrokecolor{currentstroke}%
\pgfsetdash{}{0pt}%
\pgfpathmoveto{\pgfqpoint{0.495024in}{2.606363in}}%
\pgfpathlineto{\pgfqpoint{0.530094in}{2.587894in}}%
\pgfpathlineto{\pgfqpoint{0.565165in}{2.569565in}}%
\pgfpathlineto{\pgfqpoint{0.600236in}{2.551374in}}%
\pgfpathlineto{\pgfqpoint{0.635306in}{2.533320in}}%
\pgfpathlineto{\pgfqpoint{0.670377in}{2.515400in}}%
\pgfpathlineto{\pgfqpoint{0.705448in}{2.497613in}}%
\pgfpathlineto{\pgfqpoint{0.740519in}{2.479957in}}%
\pgfpathlineto{\pgfqpoint{0.775589in}{2.462432in}}%
\pgfpathlineto{\pgfqpoint{0.810660in}{2.445034in}}%
\pgfpathlineto{\pgfqpoint{0.845731in}{2.427763in}}%
\pgfpathlineto{\pgfqpoint{0.880801in}{2.410617in}}%
\pgfpathlineto{\pgfqpoint{0.915872in}{2.393595in}}%
\pgfpathlineto{\pgfqpoint{0.950943in}{2.376695in}}%
\pgfpathlineto{\pgfqpoint{0.986014in}{2.359915in}}%
\pgfpathlineto{\pgfqpoint{1.021084in}{2.343255in}}%
\pgfpathlineto{\pgfqpoint{1.056155in}{2.326712in}}%
\pgfpathlineto{\pgfqpoint{1.091226in}{2.310286in}}%
\pgfpathlineto{\pgfqpoint{1.126296in}{2.293975in}}%
\pgfpathlineto{\pgfqpoint{1.161367in}{2.277777in}}%
\pgfpathlineto{\pgfqpoint{1.196438in}{2.261693in}}%
\pgfpathlineto{\pgfqpoint{1.231508in}{2.245719in}}%
\pgfpathlineto{\pgfqpoint{1.266579in}{2.229855in}}%
\pgfpathlineto{\pgfqpoint{1.301650in}{2.214099in}}%
\pgfpathlineto{\pgfqpoint{1.336721in}{2.198451in}}%
\pgfpathlineto{\pgfqpoint{1.371791in}{2.182910in}}%
\pgfpathlineto{\pgfqpoint{1.406862in}{2.167473in}}%
\pgfpathlineto{\pgfqpoint{1.441933in}{2.152140in}}%
\pgfpathlineto{\pgfqpoint{1.477003in}{2.136909in}}%
\pgfpathlineto{\pgfqpoint{1.512074in}{2.121780in}}%
\pgfpathlineto{\pgfqpoint{1.547145in}{2.106752in}}%
\pgfpathlineto{\pgfqpoint{1.582216in}{2.091822in}}%
\pgfpathlineto{\pgfqpoint{1.617286in}{2.076991in}}%
\pgfpathlineto{\pgfqpoint{1.652357in}{2.062257in}}%
\pgfpathlineto{\pgfqpoint{1.687428in}{2.047619in}}%
\pgfpathlineto{\pgfqpoint{1.722498in}{2.033076in}}%
\pgfpathlineto{\pgfqpoint{1.757569in}{2.018627in}}%
\pgfpathlineto{\pgfqpoint{1.792640in}{2.004271in}}%
\pgfpathlineto{\pgfqpoint{1.827710in}{1.990006in}}%
\pgfpathlineto{\pgfqpoint{1.862781in}{1.975833in}}%
\pgfpathlineto{\pgfqpoint{1.897852in}{1.961750in}}%
\pgfpathlineto{\pgfqpoint{1.932923in}{1.947756in}}%
\pgfpathlineto{\pgfqpoint{1.967993in}{1.933850in}}%
\pgfpathlineto{\pgfqpoint{2.003064in}{1.920032in}}%
\pgfpathlineto{\pgfqpoint{2.038135in}{1.906299in}}%
\pgfpathlineto{\pgfqpoint{2.073205in}{1.892653in}}%
\pgfpathlineto{\pgfqpoint{2.108276in}{1.879090in}}%
\pgfpathlineto{\pgfqpoint{2.143347in}{1.865612in}}%
\pgfpathlineto{\pgfqpoint{2.178418in}{1.852216in}}%
\pgfpathlineto{\pgfqpoint{2.213488in}{1.838902in}}%
\pgfpathlineto{\pgfqpoint{2.248559in}{1.825669in}}%
\pgfpathlineto{\pgfqpoint{2.283630in}{1.812517in}}%
\pgfpathlineto{\pgfqpoint{2.318700in}{1.799444in}}%
\pgfpathlineto{\pgfqpoint{2.353771in}{1.786450in}}%
\pgfpathlineto{\pgfqpoint{2.388842in}{1.773534in}}%
\pgfpathlineto{\pgfqpoint{2.423913in}{1.760694in}}%
\pgfpathlineto{\pgfqpoint{2.458983in}{1.747932in}}%
\pgfpathlineto{\pgfqpoint{2.494054in}{1.735245in}}%
\pgfpathlineto{\pgfqpoint{2.529125in}{1.722633in}}%
\pgfpathlineto{\pgfqpoint{2.564195in}{1.710095in}}%
\pgfpathlineto{\pgfqpoint{2.599266in}{1.697630in}}%
\pgfpathlineto{\pgfqpoint{2.634337in}{1.685238in}}%
\pgfpathlineto{\pgfqpoint{2.669407in}{1.672918in}}%
\pgfpathlineto{\pgfqpoint{2.704478in}{1.660670in}}%
\pgfpathlineto{\pgfqpoint{2.739549in}{1.648492in}}%
\pgfpathlineto{\pgfqpoint{2.774620in}{1.636384in}}%
\pgfpathlineto{\pgfqpoint{2.809690in}{1.624345in}}%
\pgfpathlineto{\pgfqpoint{2.844761in}{1.612375in}}%
\pgfpathlineto{\pgfqpoint{2.879832in}{1.600473in}}%
\pgfpathlineto{\pgfqpoint{2.914902in}{1.588638in}}%
\pgfpathlineto{\pgfqpoint{2.949973in}{1.576870in}}%
\pgfpathlineto{\pgfqpoint{2.985044in}{1.565169in}}%
\pgfpathlineto{\pgfqpoint{3.020115in}{1.553532in}}%
\pgfpathlineto{\pgfqpoint{3.055185in}{1.541961in}}%
\pgfpathlineto{\pgfqpoint{3.090256in}{1.530453in}}%
\pgfpathlineto{\pgfqpoint{3.125327in}{1.519010in}}%
\pgfpathlineto{\pgfqpoint{3.160397in}{1.507629in}}%
\pgfpathlineto{\pgfqpoint{3.195468in}{1.496311in}}%
\pgfpathlineto{\pgfqpoint{3.230539in}{1.485055in}}%
\pgfpathlineto{\pgfqpoint{3.265609in}{1.473860in}}%
\pgfpathlineto{\pgfqpoint{3.300680in}{1.462726in}}%
\pgfpathlineto{\pgfqpoint{3.335751in}{1.451652in}}%
\pgfpathlineto{\pgfqpoint{3.370822in}{1.440638in}}%
\pgfpathlineto{\pgfqpoint{3.405892in}{1.429684in}}%
\pgfpathlineto{\pgfqpoint{3.440963in}{1.418788in}}%
\pgfpathlineto{\pgfqpoint{3.476034in}{1.407950in}}%
\pgfpathlineto{\pgfqpoint{3.511104in}{1.397169in}}%
\pgfpathlineto{\pgfqpoint{3.546175in}{1.386446in}}%
\pgfpathlineto{\pgfqpoint{3.581246in}{1.375779in}}%
\pgfpathlineto{\pgfqpoint{3.616317in}{1.365169in}}%
\pgfpathlineto{\pgfqpoint{3.651387in}{1.354614in}}%
\pgfpathlineto{\pgfqpoint{3.686458in}{1.344114in}}%
\pgfpathlineto{\pgfqpoint{3.721529in}{1.333669in}}%
\pgfpathlineto{\pgfqpoint{3.756599in}{1.323278in}}%
\pgfpathlineto{\pgfqpoint{3.791670in}{1.312941in}}%
\pgfpathlineto{\pgfqpoint{3.826741in}{1.302657in}}%
\pgfpathlineto{\pgfqpoint{3.861811in}{1.292426in}}%
\pgfpathlineto{\pgfqpoint{3.896882in}{1.282248in}}%
\pgfpathlineto{\pgfqpoint{3.931953in}{1.272121in}}%
\pgfpathlineto{\pgfqpoint{3.967024in}{1.262045in}}%
\pgfusepath{stroke}%
\end{pgfscope}%
\begin{pgfscope}%
\pgfpathrectangle{\pgfqpoint{0.495024in}{0.499875in}}{\pgfqpoint{3.472000in}{2.587200in}} %
\pgfusepath{clip}%
\pgfsetbuttcap%
\pgfsetroundjoin%
\pgfsetlinewidth{0.501875pt}%
\definecolor{currentstroke}{rgb}{0.000000,0.000000,0.000000}%
\pgfsetstrokecolor{currentstroke}%
\pgfsetdash{{1.850000pt}{0.800000pt}}{0.000000pt}%
\pgfpathmoveto{\pgfqpoint{0.495024in}{1.892435in}}%
\pgfpathlineto{\pgfqpoint{0.530094in}{1.892435in}}%
\pgfpathlineto{\pgfqpoint{0.565165in}{1.892435in}}%
\pgfpathlineto{\pgfqpoint{0.600236in}{1.892435in}}%
\pgfpathlineto{\pgfqpoint{0.635306in}{1.892435in}}%
\pgfpathlineto{\pgfqpoint{0.670377in}{1.892435in}}%
\pgfpathlineto{\pgfqpoint{0.705448in}{1.892435in}}%
\pgfpathlineto{\pgfqpoint{0.740519in}{1.892435in}}%
\pgfpathlineto{\pgfqpoint{0.775589in}{1.892435in}}%
\pgfpathlineto{\pgfqpoint{0.810660in}{1.892435in}}%
\pgfpathlineto{\pgfqpoint{0.845731in}{1.892435in}}%
\pgfpathlineto{\pgfqpoint{0.880801in}{1.892435in}}%
\pgfpathlineto{\pgfqpoint{0.915872in}{1.892435in}}%
\pgfpathlineto{\pgfqpoint{0.950943in}{1.892435in}}%
\pgfpathlineto{\pgfqpoint{0.986014in}{1.892435in}}%
\pgfpathlineto{\pgfqpoint{1.021084in}{1.892435in}}%
\pgfpathlineto{\pgfqpoint{1.056155in}{1.892435in}}%
\pgfpathlineto{\pgfqpoint{1.091226in}{1.892435in}}%
\pgfpathlineto{\pgfqpoint{1.126296in}{1.892435in}}%
\pgfpathlineto{\pgfqpoint{1.161367in}{1.892435in}}%
\pgfpathlineto{\pgfqpoint{1.196438in}{1.892435in}}%
\pgfpathlineto{\pgfqpoint{1.231508in}{1.892435in}}%
\pgfpathlineto{\pgfqpoint{1.266579in}{1.892435in}}%
\pgfpathlineto{\pgfqpoint{1.301650in}{1.892435in}}%
\pgfpathlineto{\pgfqpoint{1.336721in}{1.892435in}}%
\pgfpathlineto{\pgfqpoint{1.371791in}{1.892435in}}%
\pgfpathlineto{\pgfqpoint{1.406862in}{1.892435in}}%
\pgfpathlineto{\pgfqpoint{1.441933in}{1.892435in}}%
\pgfpathlineto{\pgfqpoint{1.477003in}{1.892435in}}%
\pgfpathlineto{\pgfqpoint{1.512074in}{1.892435in}}%
\pgfpathlineto{\pgfqpoint{1.547145in}{1.892435in}}%
\pgfpathlineto{\pgfqpoint{1.582216in}{1.892435in}}%
\pgfpathlineto{\pgfqpoint{1.617286in}{1.892435in}}%
\pgfpathlineto{\pgfqpoint{1.652357in}{1.892435in}}%
\pgfpathlineto{\pgfqpoint{1.687428in}{1.892435in}}%
\pgfpathlineto{\pgfqpoint{1.722498in}{1.892435in}}%
\pgfpathlineto{\pgfqpoint{1.757569in}{1.892435in}}%
\pgfpathlineto{\pgfqpoint{1.792640in}{1.892435in}}%
\pgfpathlineto{\pgfqpoint{1.827710in}{1.892435in}}%
\pgfpathlineto{\pgfqpoint{1.862781in}{1.892435in}}%
\pgfpathlineto{\pgfqpoint{1.897852in}{1.892435in}}%
\pgfpathlineto{\pgfqpoint{1.932923in}{1.892435in}}%
\pgfpathlineto{\pgfqpoint{1.967993in}{1.892435in}}%
\pgfpathlineto{\pgfqpoint{2.003064in}{1.892435in}}%
\pgfpathlineto{\pgfqpoint{2.038135in}{1.892435in}}%
\pgfpathlineto{\pgfqpoint{2.073205in}{1.892435in}}%
\pgfpathlineto{\pgfqpoint{2.108276in}{1.892435in}}%
\pgfpathlineto{\pgfqpoint{2.143347in}{1.892435in}}%
\pgfpathlineto{\pgfqpoint{2.178418in}{1.892435in}}%
\pgfpathlineto{\pgfqpoint{2.213488in}{1.892435in}}%
\pgfpathlineto{\pgfqpoint{2.248559in}{1.892435in}}%
\pgfpathlineto{\pgfqpoint{2.283630in}{1.892435in}}%
\pgfpathlineto{\pgfqpoint{2.318700in}{1.892435in}}%
\pgfpathlineto{\pgfqpoint{2.353771in}{1.892435in}}%
\pgfpathlineto{\pgfqpoint{2.388842in}{1.892435in}}%
\pgfpathlineto{\pgfqpoint{2.423913in}{1.892435in}}%
\pgfpathlineto{\pgfqpoint{2.458983in}{1.892435in}}%
\pgfpathlineto{\pgfqpoint{2.494054in}{1.892435in}}%
\pgfpathlineto{\pgfqpoint{2.529125in}{1.892435in}}%
\pgfpathlineto{\pgfqpoint{2.564195in}{1.892435in}}%
\pgfpathlineto{\pgfqpoint{2.599266in}{1.892435in}}%
\pgfpathlineto{\pgfqpoint{2.634337in}{1.892435in}}%
\pgfpathlineto{\pgfqpoint{2.669407in}{1.892435in}}%
\pgfpathlineto{\pgfqpoint{2.704478in}{1.892435in}}%
\pgfpathlineto{\pgfqpoint{2.739549in}{1.892435in}}%
\pgfpathlineto{\pgfqpoint{2.774620in}{1.892435in}}%
\pgfpathlineto{\pgfqpoint{2.809690in}{1.892435in}}%
\pgfpathlineto{\pgfqpoint{2.844761in}{1.892435in}}%
\pgfpathlineto{\pgfqpoint{2.879832in}{1.892435in}}%
\pgfpathlineto{\pgfqpoint{2.914902in}{1.892435in}}%
\pgfpathlineto{\pgfqpoint{2.949973in}{1.892435in}}%
\pgfpathlineto{\pgfqpoint{2.985044in}{1.892435in}}%
\pgfpathlineto{\pgfqpoint{3.020115in}{1.892435in}}%
\pgfpathlineto{\pgfqpoint{3.055185in}{1.892435in}}%
\pgfpathlineto{\pgfqpoint{3.090256in}{1.892435in}}%
\pgfpathlineto{\pgfqpoint{3.125327in}{1.892435in}}%
\pgfpathlineto{\pgfqpoint{3.160397in}{1.892435in}}%
\pgfpathlineto{\pgfqpoint{3.195468in}{1.892435in}}%
\pgfpathlineto{\pgfqpoint{3.230539in}{1.892435in}}%
\pgfpathlineto{\pgfqpoint{3.265609in}{1.892435in}}%
\pgfpathlineto{\pgfqpoint{3.300680in}{1.892435in}}%
\pgfpathlineto{\pgfqpoint{3.335751in}{1.892435in}}%
\pgfpathlineto{\pgfqpoint{3.370822in}{1.892435in}}%
\pgfpathlineto{\pgfqpoint{3.405892in}{1.892435in}}%
\pgfpathlineto{\pgfqpoint{3.440963in}{1.892435in}}%
\pgfpathlineto{\pgfqpoint{3.476034in}{1.892435in}}%
\pgfpathlineto{\pgfqpoint{3.511104in}{1.892435in}}%
\pgfpathlineto{\pgfqpoint{3.546175in}{1.892435in}}%
\pgfpathlineto{\pgfqpoint{3.581246in}{1.892435in}}%
\pgfpathlineto{\pgfqpoint{3.616317in}{1.892435in}}%
\pgfpathlineto{\pgfqpoint{3.651387in}{1.892435in}}%
\pgfpathlineto{\pgfqpoint{3.686458in}{1.892435in}}%
\pgfpathlineto{\pgfqpoint{3.721529in}{1.892435in}}%
\pgfpathlineto{\pgfqpoint{3.756599in}{1.892435in}}%
\pgfpathlineto{\pgfqpoint{3.791670in}{1.892435in}}%
\pgfpathlineto{\pgfqpoint{3.826741in}{1.892435in}}%
\pgfpathlineto{\pgfqpoint{3.861811in}{1.892435in}}%
\pgfpathlineto{\pgfqpoint{3.896882in}{1.892435in}}%
\pgfpathlineto{\pgfqpoint{3.931953in}{1.892435in}}%
\pgfpathlineto{\pgfqpoint{3.967024in}{1.892435in}}%
\pgfusepath{stroke}%
\end{pgfscope}%
\begin{pgfscope}%
\pgfpathrectangle{\pgfqpoint{0.495024in}{0.499875in}}{\pgfqpoint{3.472000in}{2.587200in}} %
\pgfusepath{clip}%
\pgfsetbuttcap%
\pgfsetroundjoin%
\definecolor{currentfill}{rgb}{0.000000,0.000000,0.000000}%
\pgfsetfillcolor{currentfill}%
\pgfsetlinewidth{0.000000pt}%
\definecolor{currentstroke}{rgb}{0.000000,0.000000,0.000000}%
\pgfsetstrokecolor{currentstroke}%
\pgfsetdash{}{0pt}%
\pgfsys@defobject{currentmarker}{\pgfqpoint{-0.041667in}{-0.041667in}}{\pgfqpoint{0.041667in}{0.041667in}}{%
\pgfpathmoveto{\pgfqpoint{0.000000in}{-0.041667in}}%
\pgfpathcurveto{\pgfqpoint{0.011050in}{-0.041667in}}{\pgfqpoint{0.021649in}{-0.037276in}}{\pgfqpoint{0.029463in}{-0.029463in}}%
\pgfpathcurveto{\pgfqpoint{0.037276in}{-0.021649in}}{\pgfqpoint{0.041667in}{-0.011050in}}{\pgfqpoint{0.041667in}{0.000000in}}%
\pgfpathcurveto{\pgfqpoint{0.041667in}{0.011050in}}{\pgfqpoint{0.037276in}{0.021649in}}{\pgfqpoint{0.029463in}{0.029463in}}%
\pgfpathcurveto{\pgfqpoint{0.021649in}{0.037276in}}{\pgfqpoint{0.011050in}{0.041667in}}{\pgfqpoint{0.000000in}{0.041667in}}%
\pgfpathcurveto{\pgfqpoint{-0.011050in}{0.041667in}}{\pgfqpoint{-0.021649in}{0.037276in}}{\pgfqpoint{-0.029463in}{0.029463in}}%
\pgfpathcurveto{\pgfqpoint{-0.037276in}{0.021649in}}{\pgfqpoint{-0.041667in}{0.011050in}}{\pgfqpoint{-0.041667in}{0.000000in}}%
\pgfpathcurveto{\pgfqpoint{-0.041667in}{-0.011050in}}{\pgfqpoint{-0.037276in}{-0.021649in}}{\pgfqpoint{-0.029463in}{-0.029463in}}%
\pgfpathcurveto{\pgfqpoint{-0.021649in}{-0.037276in}}{\pgfqpoint{-0.011050in}{-0.041667in}}{\pgfqpoint{0.000000in}{-0.041667in}}%
\pgfpathclose%
\pgfusepath{fill}%
}%
\begin{pgfscope}%
\pgfsys@transformshift{2.073205in}{1.892653in}%
\pgfsys@useobject{currentmarker}{}%
\end{pgfscope}%
\end{pgfscope}%
\begin{pgfscope}%
\pgfpathrectangle{\pgfqpoint{0.495024in}{0.499875in}}{\pgfqpoint{3.472000in}{2.587200in}} %
\pgfusepath{clip}%
\pgfsetbuttcap%
\pgfsetroundjoin%
\pgfsetlinewidth{0.501875pt}%
\definecolor{currentstroke}{rgb}{0.000000,0.000000,0.000000}%
\pgfsetstrokecolor{currentstroke}%
\pgfsetdash{{3.200000pt}{0.800000pt}{0.500000pt}{0.800000pt}}{0.000000pt}%
\pgfpathmoveto{\pgfqpoint{2.073205in}{0.489875in}}%
\pgfpathlineto{\pgfqpoint{2.073205in}{1.892653in}}%
\pgfusepath{stroke}%
\end{pgfscope}%
\begin{pgfscope}%
\pgfsetrectcap%
\pgfsetmiterjoin%
\pgfsetlinewidth{0.501875pt}%
\definecolor{currentstroke}{rgb}{0.000000,0.000000,0.000000}%
\pgfsetstrokecolor{currentstroke}%
\pgfsetdash{}{0pt}%
\pgfpathmoveto{\pgfqpoint{0.495024in}{0.499875in}}%
\pgfpathlineto{\pgfqpoint{0.495024in}{3.087075in}}%
\pgfusepath{stroke}%
\end{pgfscope}%
\begin{pgfscope}%
\pgfsetrectcap%
\pgfsetmiterjoin%
\pgfsetlinewidth{0.501875pt}%
\definecolor{currentstroke}{rgb}{0.000000,0.000000,0.000000}%
\pgfsetstrokecolor{currentstroke}%
\pgfsetdash{}{0pt}%
\pgfpathmoveto{\pgfqpoint{3.967024in}{0.499875in}}%
\pgfpathlineto{\pgfqpoint{3.967024in}{3.087075in}}%
\pgfusepath{stroke}%
\end{pgfscope}%
\begin{pgfscope}%
\pgfsetrectcap%
\pgfsetmiterjoin%
\pgfsetlinewidth{0.501875pt}%
\definecolor{currentstroke}{rgb}{0.000000,0.000000,0.000000}%
\pgfsetstrokecolor{currentstroke}%
\pgfsetdash{}{0pt}%
\pgfpathmoveto{\pgfqpoint{0.495024in}{0.499875in}}%
\pgfpathlineto{\pgfqpoint{3.967024in}{0.499875in}}%
\pgfusepath{stroke}%
\end{pgfscope}%
\begin{pgfscope}%
\pgfsetrectcap%
\pgfsetmiterjoin%
\pgfsetlinewidth{0.501875pt}%
\definecolor{currentstroke}{rgb}{0.000000,0.000000,0.000000}%
\pgfsetstrokecolor{currentstroke}%
\pgfsetdash{}{0pt}%
\pgfpathmoveto{\pgfqpoint{0.495024in}{3.087075in}}%
\pgfpathlineto{\pgfqpoint{3.967024in}{3.087075in}}%
\pgfusepath{stroke}%
\end{pgfscope}%
\begin{pgfscope}%
\pgfsetbuttcap%
\pgfsetmiterjoin%
\definecolor{currentfill}{rgb}{1.000000,1.000000,1.000000}%
\pgfsetfillcolor{currentfill}%
\pgfsetfillopacity{0.800000}%
\pgfsetlinewidth{1.003750pt}%
\definecolor{currentstroke}{rgb}{0.000000,0.000000,0.000000}%
\pgfsetstrokecolor{currentstroke}%
\pgfsetstrokeopacity{0.800000}%
\pgfsetdash{}{0pt}%
\pgfpathmoveto{\pgfqpoint{2.709767in}{2.351401in}}%
\pgfpathlineto{\pgfqpoint{3.890635in}{2.351401in}}%
\pgfpathlineto{\pgfqpoint{3.890635in}{3.010686in}}%
\pgfpathlineto{\pgfqpoint{2.709767in}{3.010686in}}%
\pgfpathclose%
\pgfusepath{stroke,fill}%
\end{pgfscope}%
\begin{pgfscope}%
\pgfsetrectcap%
\pgfsetroundjoin%
\pgfsetlinewidth{0.501875pt}%
\definecolor{currentstroke}{rgb}{0.000000,0.000000,0.000000}%
\pgfsetstrokecolor{currentstroke}%
\pgfsetdash{}{0pt}%
\pgfpathmoveto{\pgfqpoint{2.770878in}{2.794548in}}%
\pgfpathlineto{\pgfqpoint{3.076433in}{2.794548in}}%
\pgfusepath{stroke}%
\end{pgfscope}%
\begin{pgfscope}%
\pgftext[x=3.198656in,y=2.741076in,left,base]{\rmfamily\fontsize{11.000000}{13.200000}\selectfont \(\displaystyle \displaystyle\frac{m!}{m^{m+1}}\mathrm{e}^m\)}%
\end{pgfscope}%
\begin{pgfscope}%
\pgfsetbuttcap%
\pgfsetroundjoin%
\pgfsetlinewidth{0.501875pt}%
\definecolor{currentstroke}{rgb}{0.000000,0.000000,0.000000}%
\pgfsetstrokecolor{currentstroke}%
\pgfsetdash{{1.850000pt}{0.800000pt}}{0.000000pt}%
\pgfpathmoveto{\pgfqpoint{2.770878in}{2.497763in}}%
\pgfpathlineto{\pgfqpoint{3.076433in}{2.497763in}}%
\pgfusepath{stroke}%
\end{pgfscope}%
\begin{pgfscope}%
\pgftext[x=3.198656in,y=2.444290in,left,base]{\rmfamily\fontsize{11.000000}{13.200000}\selectfont \(\displaystyle A\cdot B\)}%
\end{pgfscope}%
\end{pgfpicture}%
\makeatother%
\endgroup%

    \label{fig:graph_solve}
\end{figure}

On remarque l'intersection entre les deux courbes est entre $m=24$ et
$m=25$. L'intersection donnée par le calcul du graphique est d'environ
$m\approx 24.54$. Par conséquent, on peut déduire que $m=25$ est le
nombre de composants qui reflète le mieux l'estimation de l'ingénieur.

\subsection{b)}
Soit $T$ la durée de vie d'un composant $k$ avec le taux de défaillance 
donnée par
\begin{equation*}
    r(t)=Ct,
\end{equation*}
où $C$ est une constante. On obtient la fonction de fiabilité avec
\begin{equation*}
    R_k(t)
    =\exp\left\{-\int_0^t r(s)\d{s}\right\}
    =\exp\left\{-C\int_0^t s\d{s}\right\}
    =\e^{-Ct^2/2}.
\end{equation*}

La figure \ref{fig:sys_parallel} montre un système contenant $n$ de ces
composants en parallèle.
\begin{figure}[H]
    \centering
    \documentclass[11pt]{article}
\usepackage[french]{babel}
\usepackage[utf8]{inputenc}
\usepackage[T1]{fontenc}
\usepackage{standalone}
\usepackage{tikz}

\usetikzlibrary{babel}
\usetikzlibrary{calc}

\begin{document}
\begin{tikzpicture}[
    every node/.style={
        draw,
        minimum width=2cm,
        minimum height=1cm
    }
]

\node            (C1) at (0,0) {$1$};
\node            (C2) at (3,0) {$2$};
\node[draw=none] (CD) at (6,0) {$\cdots$};
\node            (CN) at (9,0) {$n$};

\coordinate (T)  at (4.5, 1);
\coordinate (B)  at (4.5,-1);

\draw let
    \p1 = (T),
    \p2 = (B),
    \p3 = (C2)
in (C1.north) |- (T) -| (CN.north)
   (C2.north) -- (\x3,\y1)
   (C1.south) |- (B) -| (CN.south)
   (C2.south) -- (\x3,\y2)
   (T) -- (\x1,\y1+5mm)
   (B) -- (\x1,\y2-5mm);


\end{tikzpicture}
\end{document}




    \caption{système en parallèle à $n$ composants}
    \label{fig:sys_parallel}
\end{figure}

Soit $X$ la durée de vie du système en entier donc la fonction de
fiabilité est donnée par
\begin{equation*}
    R(x)
    =1-\prod_{k=1}^n\bigg[1-R_k(x)\bigg]
    =1-\bigg[1-\e^{-Cx^2/2}\bigg]^n,
\end{equation*}
pour $x\geq 0$, car les composants sont indépendant.

On obtient la fonction de répartition de $X$ avec
\begin{equation*}
    R(x)=1-F_X(x)\Leftrightarrow
    F_X(x)=1-R(x)=\bigg[1-\e^{-Cx^2/2}\bigg]^n
\end{equation*}
de sorte que la médianne de $X$ est la valeur de $x$ tel que
\begin{equation*}
    \frac{1}{2}=\bigg[1-\e^{-Cx^2/2}\bigg]^n.
\end{equation*}

En prenant le logarithme naturel, on obtient
\begin{equation*}
    -\ln{2}=n\ln\bigg[1-\e^{-Cx^2/2}\bigg],
\end{equation*}
de sorte à isoler $n$, soit
\begin{equation*}
    n=-\frac{\ln{2}}{\ln\big[1-\e^{-Cx^2/2}\big]}.
\end{equation*}

L'ingénieur estime la médiane à $x=D$, donc
\begin{equation*}
    n
    \approx-\frac{\ln{2}}{\ln\big[1-\e^{-CD^2/2}\big]}
    \approx{28.511},
\end{equation*}
avec $C=2.950$ et $D=1.590$. Par conséquent, on peut déduire que
$n=29$ est le nombre de composants qui reflète le mieux l'estimation
de l'ingénieur.

\pagebreak
\section*{Question 2}
Soit une expérience aléatoire dont le résultat est modélisé par
la variable aléatoire
\begin{equation*}
    X\sim\Norm{E}{F^2},
\end{equation*}
où $E$ et $F$ sont des constantes.

\subsection*{a)}
On s'intérèsse au nombre d'essaies $M$ de l'expérience jusqu'à obtenir
un premier résultat supérieur à une constante $G$. Par conséquent, $M$
suit une loi géométrique tel que
\begin{equation*}
    M\sim\Geo{p},
\end{equation*}
où $p=\P{X>G}$, la probabilité que le résultat soit supérieur à $G$.

La probabilité $m$ que $M$ soit une valeur impaire est donnée par
\begin{equation*}
    m
    =\sum_{k=1}^\infty\P{M=2k-1}
    =\sum_{k=1}^\infty q^{2k-1-1}p
    =\sum_{k=1}^\infty q^{2(k-1)}p
    =q^2\sum_{k=1}^\infty q^{k-1}p,
\end{equation*}
où $q=1-p$. On remarque que la somme est effectuée sur tout l'espace
échantillion d'une distribution géométrique. Par conséquent,
\begin{equation*}
    m=q^2=(1-p)^2.
\end{equation*}

On calcule $p$ avec la distribution normale centrée réduite, c'est-à-dire
\begin{equation*}
    p
    =1-\P{X\leq G}
    =1-\Phi\left(\frac{G-E}{F^2}\right).
\end{equation*}
Avec $E=0.770$, $F=1.780$ et $G=1.535$, on obtient
\begin{equation*}
    p
    \approx 1-\Phi\left(0.42978\right)
    \approx 1-0.6628
    =0.3372,
\end{equation*}
de sorte que la probabilité que $M$ prenne une valeur impaire est
$m\approx 0.4393$.

\subsection{b)}
On répète l'expérience $H$ fois, où $H$ est une constante. Soit $N$ le
nombre de fois où le carré du résultat est supérieur à une constante $I$.
Par conséquent, $N$ suit une loi binomiale tel que
\begin{equation*}
    N\sim\Bin{H}{p},
\end{equation*}
où $p=\P{X^2>I}=\P{-\sqrt{I}<X<\sqrt{I}}$, la probabilité que le carré
du résultat soit supérieur à $I$.

On évalue à l'aide de la distribution normale centrée réduite, alors
\begin{equation*}
    p
    =\Phi\bigg(\frac{ \sqrt{I}-E}{F^2}\bigg)
    -\Phi\bigg(\frac{-\sqrt{I}-E}{F^2}\bigg)
    =\Phi\bigg(\frac{ \sqrt{I}-E}{F^2}\bigg)
    +\Phi\bigg(\frac{ \sqrt{I}+E}{F^2}\bigg)-1.
\end{equation*}
Avec $E=0.770$, $F=1.780$ et $I=1.160$, on obtient
\begin{equation*}
    p
    \approx\Phi\left(0.0969\right)+\Phi\left(0.5830\right)-1
    \approx 0.5359+0.7190-1
    =0.2549.
\end{equation*}

La probabilité $n$ que $N$ soit une valeur paire est donnée par
\begin{equation*}
    n
    =\sum_{k=0}^{\left\lfloor H/2\right\rfloor}\P{N=2k}
    =\sum_{k=0}^{\left\lfloor H/2\right\rfloor}\binom{2k}{H}p^{2k}q^{H-2k}
\end{equation*}
Par évaluation numérique, avec $H=59$, on obtient $n\approx 0.500$.

\pagebreak
\section*{Question 3}
Soit $(X,Y)$ un vecteur aléatoire continu dont la fonction de densité
conjointe est donnée par
\begin{equation*}
    f_{X,Y}(x,y)=\left\{
        \begin{matrix}
            c\e^{-Jy} & \text{si} & K\leq x\leq L, y\geq 0,\\
            0         & \text{sinon},
        \end{matrix}
    \right.
\end{equation*}
où $J$, $K$ et $L$ sont des constantes connues et $c$ une constante inconnue.

\subsection{a)}
On obtient la fonction de densité de $X$ en intégrant une première fois, soit
\begin{equation*}
    f_X(x)
    =\int_{-\infty}^\infty f_{X,Y}(x,y)\d{y}
    =c\int_0^\infty\e^{-Jy}\d{y}
    =-\frac{c}{J}\e^{-Jy}\bigg\rvert_0^\infty
    =\frac{c}{J},
\end{equation*}
et celle de répartition en intégrant une deuxième fois, soit
\begin{equation*}
    F_X(x)
    =\int_{-\infty}^x f_X(t)\d{t}
    =\frac{c}{J}\int_K^x\d{t}
    =\frac{c}{J}(x-K),
\end{equation*}
car $K\leq x\leq L$.

D'une manière similaire, on obtient la fonction de densité de $Y$ en intégrant
une première fois, soit
\begin{equation*}
    f_Y(y)
    =\int_{-\infty}^\infty f_{X,Y}(x,y)\d{x}
    =c\e^{-Jy}\int_K^L\d{x}
    =(L-K)c\e^{-Jy},
\end{equation*}
et celle de répartition en intégrant une deuxième fois, soit
\begin{equation*}
    F_Y(y)
    =\int_{-\infty}^y f_Y(t)\d{t}
    =(L-K)c\int_0^y\e^{-Jt}\d{t}
    =-\frac{c}{J}(L-K)\e^{-Jt}\bigg\rvert_0^y
    =\frac{c}{J}(L-K)(1-\e^{-Jy}),
\end{equation*}
car $y\geq 0$.

Soit $T$ une variable aléatoire définie par $T=g(X)$, où $g(x)$ est une fonction
quelconque. On cherche la fonction $g(X)$ tel que $T=Y$, c'est-à-dire
\begin{equation*}
    \P{T\leq t}
    =\P{g(X)\leq t}
    =\P{Y\leq t},
\end{equation*}
de sorte que les fonctions de répartition de $T$ et $Y$ soient identiques.

Avec l'inverse de $g(x)$, on peut isoler $X$ de sorte à obtenir
\begin{equation*}
    \P{g^{-1}(g(X))\leq g^{-1}(t)}
    =\P{X\leq g^{-1}(t)}
    =\P{Y\leq t}.
\end{equation*}
Par conséquent, on a
\begin{equation*}
    F_X\left(g^{-1}(t)\right)
    =F_Y(t)
    \Leftrightarrow
    \frac{c}{J}\left(g^{-1}(t)-K\right)
    =\frac{c}{J}(L-K)(1-\e^{-Jt}),
\end{equation*}
et en isolant, on obtient
\begin{equation*}
    g^{-1}(t)
    =(L-K)(1-\e^{-Jt})+K.
\end{equation*}

On a donc
\begin{equation*}
    \P{X\leq (K-L)(1-\e^{-Jt})+K}
    =\P{\frac{X-K}{L-K}\leq 1-\e^{-Jt}}
    =\P{1-\frac{X-K}{L-K}\geq \e^{-Jt}},
\end{equation*}
car $L-K>0$, de sorte qu'en appliquant le logarithme, on obtient
\begin{equation*}
    \P{1-\frac{X-K}{L-K}\geq \e^{-Jt}}
    =\P{-\frac{1}{J}\ln\left(1-\frac{X-K}{L-K}\right)\leq t}.
\end{equation*}
Le terme à droite est la transformation pour obtenir $T$. Par conséquent,
\begin{equation*}
    g(x)=-\frac{1}{J}\ln\left(1-\frac{x-K}{L-K}\right).
\end{equation*}

\subsection{b)}
Soit $Z$ la variable aléatoire définie par Z=\Max{X}{Y}. Cette fonction
peut se définir par partie, soit
\begin{equation*}
    \Max{x}{y}=\left\{
        \begin{matrix}
            x & \text{si} & x\geq y,\\
            y & \text{si} & x  <  y.\\
        \end{matrix}
    \right.
\end{equation*}
de sorte que la fonction de répartition de $Z$ est
\begin{equation*}
    F_Z(z)
    =\P{Z\leq z}
    =\P{\Max{X}{Y}\leq z}
    =\P{\left\{X\leq z\right\}\cap\left\{Y\leq z\right\}}.
\end{equation*}

Hors, c'est la fonction de répartition du vecteur aléatoire $(X,Y)$ évaluée
en $(z,z)$, soit
\begin{equation*}
    F_Z(z)
    =\int_{-\infty}^z\int_{-\infty}^zf_{X,Y}(x,y)\d{y}\d{x}
    =\int_K^z\int_0^zc\e^{-Jy}\d{y}\d{x}
    =c\int_K^z\d{x}\int_0^z\e^{-Jy}\d{y},
\end{equation*}
de sorte à obtenir
\begin{equation*}
    F_Z(z)
    =\frac{c}{J}(z-K)(1+\e^{-Jz}),
\end{equation*}
pour $K\leq z\leq L$, car $K\leq x\leq L$.

Lorsque $z>L$, on a
\begin{equation*}
    F_Z(z)
    =\int_{-\infty}^z\int_{-\infty}^zf_{X,Y}(x,y)\d{y}\d{x}
    =\int_K^L\int_0^zc\e^{-Jy}\d{y}\d{x}
    =c\int_K^L\d{x}\int_0^z\e^{-Jy}\d{y},
\end{equation*}
de sorte à obtenir
\begin{equation*}
    F_Z(z)
    =\frac{c}{J}(L-K)(1-\e^{-Jz}).
\end{equation*}

Par conséquent, la fonction de répartition de $Z$ est
\begin{equation*}
    F_Z(z)=\left\{
        \begin{matrix}
            0                             & \text{si} &       z  <  K,\\[6pt]
            \dfrac{c}{J}(z-K)(1+\e^{-Jz}) & \text{si} & K\leq z\leq L,\\[6pt]
            \dfrac{c}{J}(L-K)(1-\e^{-Jz}) & \text{si} & L  <  z.      \\[6pt]
        \end{matrix}
    \right.
\end{equation*}

\pagebreak
\section*{Question 4}
Soit $(X,Y)$ un vecteur aléatoire discret dont la fonction de masse conjointe
est donnée par
\begin{equation*}
    p_{X,Y}(j,k)=\left\{
        \begin{matrix}
            \e^{-M} & \text{si} & (j,k)=(0,0),\\
            \dfrac{\e^{-M}M^j}{j!}\cdot\dfrac{\e^{-(Nj)}(Nj)^k}{k!}
                & \text{si} & (j,k)\in\left\{1,2,\dots\right\}
                \times \left\{0,1,\dots\right\},\\
        \end{matrix}
    \right.
\end{equation*}
où $M$ et $N$ sont des constantes connues.

\subsection{a)}
La fonction de masse de $Y$ est donnée par
\begin{equation*}
    p_Y(k)
    =\sum_{j=0}^\infty p_{X,Y}(j,k).
\end{equation*}

\subsubsection*{Cas où \boldmath $k=0$}
Il faut séparer la somme en deux, soit
\begin{equation*}
    p_Y(0)
    =\e^{-M}+\sum_{j=1}^\infty\frac{\e^{-M}M^j}{j!}
        \cdot\frac{\e^{-(Nj)}(Nj)^0}{0!}
    =\e^{-M}+\e^{-M}\sum_{j=1}^\infty\frac{M^j}{j!}
        \cdot\e^{-(Nj)},
\end{equation*}
car $p_{X,Y}(j,k)$ est définie par partie. On peut écrire la somme comme
\begin{equation*}
    \sum_{j=1}^\infty\frac{M^j}{j!}\cdot\e^{-(Nj)}
    =\sum_{j=1}^\infty\frac{\left(M\e^{-N}\right)^j}{j!}
    =\sum_{j=0}^\infty\frac{\left(M\e^{-N}\right)^j}{j!}
        -\frac{\left(M\e^{-N}\right)^0}{0!}
    =\sum_{j=0}^\infty\frac{\left(M\e^{-N}\right)^j}{j!}-1.
\end{equation*}
Hors, on remarque que la somme est la série de MacLaurin de la fonction
exponentielle, alors
\begin{equation*}
    \sum_{j=0}^\infty\frac{\left(M\e^{-N}\right)^j}{j!}-1
    =\e^{M\e^{-N}}-1,
\end{equation*}
de sorte à obtenir
\begin{equation*}
    p_Y(0)
    =\e^{-M}+\e^{-M}\left(\e^{M\e^{-N}}-1\right)
    =\e^{-M}\e^{M\e^{-N}}
    =\e^{M\e^{-N}-M}.
\end{equation*}

\subsubsection*{Cas où \boldmath $k=1$}
Puisque $(j,k)\neq (0,0)$, on a
\begin{equation*}
    p_Y(1)
    =\sum_{j=0}^\infty\frac{\e^{-M}M^j}{j!}\cdot\frac{\e^{-(Nj)}(Nj)^1}{1!}
    =N\e^{-M}\sum_{j=0}^\infty\frac{\left(M\e^{-N}\right)^j}{j!}j
\end{equation*}
Le terme de la somme à $j=0$ est nul, alors on peut écrire
\begin{equation}\label{eq:part1}
    \sum_{j=0}^\infty\frac{\left(M\e^{-N}\right)^j}{j!}j
    =\sum_{j=1}^\infty\frac{\left(M\e^{-N}\right)^j}{j!}j
    =\sum_{j=1}^\infty\frac{\left(M\e^{-N}\right)^j}{(j-1)!},
\end{equation}
de sorte qu'avec une translation de la série, on obtient
\begin{equation}\label{eq:part2}
    \sum_{j=1}^\infty\frac{\left(M\e^{-N}\right)^j}{(j-1)!}
    =\sum_{j=0}^\infty\frac{\left(M\e^{-N}\right)^{j+1}}{j!}
    =M\e^{-N}\sum_{j=0}^\infty\frac{\left(M\e^{-N}\right)^j}{j!}
    =M\e^{-N}\e^{M\e^{-N}},
\end{equation}
car la somme est le développement en série de la fonction exponentielle. Par
conséquent, on a
\begin{equation*}
    p_Y(1)
    =N\e^{-M}M\e^{-N}\e^{M\e^{-N}}
    =NM\e^{-N}\e^{M\e^{-N}-M}.
\end{equation*}

\subsubsection*{Cas où \boldmath $k=2$}
Puisque $(j,k)\neq (0,0)$, on a
\begin{equation*}
    p_Y(2)
    =\sum_{j=0}^\infty\frac{\e^{-M}M^j}{j!}\cdot\frac{\e^{-(Nj)}(Nj)^2}{2!}
    =\frac{N^2}{2}\e^{-M}\sum_{j=0}^\infty\frac{\left(M\e^{-N}\right)^j}{j!}j^2.
\end{equation*}
Le terme de la somme à $j=0$ est nul, alors on peut écrire
\begin{equation*}
    \sum_{j=0}^\infty\frac{\left(M\e^{-N}\right)^j}{j!}j^2
    =\sum_{j=1}^\infty\frac{\left(M\e^{-N}\right)^j}{j!}j^2
    =\sum_{j=1}^\infty\frac{\left(M\e^{-N}\right)^j}{(j-1)!}j,
\end{equation*}
de sorte qu'avec une translation de la série, on obtient
\begin{equation*}
    \sum_{j=1}^\infty\frac{\left(M\e^{-N}\right)^j}{(j-1)!}j
    =\sum_{j=0}^\infty\frac{\left(M\e^{-N}\right)^{j+1}}{j!}(j+1)
    =M\e^{-N}\left[
        \sum_{j=0}^\infty\frac{\left(M\e^{-N}\right)^j}{j!}j+
        \sum_{j=0}^\infty\frac{\left(M\e^{-N}\right)^j}{j!}
    \right]
\end{equation*}

On remarque que la première somme est celle développée dans le cas où $k=1$
dans les équations \eqref{eq:part1} et \eqref{eq:part2}, et que la deuxième somme
est le développement en série de la fonction exponentielle, comme dans le cas
$k=0$. Par conséquent, la somme est
\begin{equation*}
    \sum_{j=0}^\infty\frac{\left(M\e^{-N}\right)^j}{j!}j^2
    =M\e^{-N}\left[M\e^{-N}\e^{M\e^{-N}}+\e^{M\e^{-N}}\right]
    =M\e^{-N}(M\e^{-N}+1)\e^{M\e^{-N}}
\end{equation*}
de sorte à obtenir
\begin{equation*}
    p_Y(2)
    =\frac{N^2}{2}\e^{-M}M\e^{-N}(M\e^{-N}+1)\e^{M\e^{-N}}
    =\frac{N^2}{2}M\e^{-N}(M\e^{-N}+1)\e^{M\e^{-N}-M}.
\end{equation*}

\subsubsection*{Cas où \boldmath $k>0$}
Puisque $(j,k)\neq (0,0)$, on a
\begin{equation}\label{eq:thisispoisson}
    p_Y(k)
    =\sum_{j=0}^\infty\frac{\e^{-M}M^j}{j!}\cdot\frac{\e^{-(Nj)}(Nj)^k}{k!}
    =\frac{N^k}{k!}\e^{-M}\sum_{j=0}^\infty\frac{\left(M\e^{-N}\right)^j}{j!}j^k
\end{equation}
Hors, soit $T\sim\Poi{\alpha}$ une variable aléatoire suivant une loi de Poisson
de paramètre $\alpha$. Le $k$-ième moment de $T$ est donné par
\begin{equation*}
    \Esp{T^k}
    =\sum_{j=0}^\infty\frac{\alpha^j}{j!}\e^{-\alpha}j^k
    =\e^{-\alpha}\sum_{j=0}^\infty\frac{\alpha^j}{j!}j^k.
\end{equation*}
On remarque que la somme dans l'équation \eqref{eq:thisispoisson} est le
$k$-ième moment d'une distribution de poisson avec $\alpha=M\e^{-N}$. Par
conséquent, on a
\begin{equation*}
    \sum_{j=0}^\infty\frac{\left(M\e^{-N}\right)^j}{j!}j^k
    =\e^{M\e^{-N}}\Esp{T^k},
\end{equation*}
où $T\sim\Poi{M\e^{-N}}$, et
\begin{equation*}
    p_Y(k)
    =\frac{N^k}{k!}\e^{M\e^{-N}-M}\Esp{T^k}.
\end{equation*}
On remarque que ce résultat est aussi valide avec $k=0$. La table \ref{tb:moment}
montre les moments de $T$ ainsi que les $p_Y(k)$ reliées à ceux-ci.

\begin{table}[H]
    \centering
    \caption{Liste des moments de $T$}
    \vspace{5mm}
    \begin{tabular}{ccc}
        \toprule
        $k$ & $\Esp{T^k}$ & $p_Y(k)$\\
        \midrule
        $0$ 
            & $1$
            & $\e^{M\e^{-N}-M}$\\
        $1$ 
            & $M\e^{-N}$
            & $N\e^{M\e^{-N}-M}M\e^{-N}$\\
        $2$ 
            & $M\left(M\e^{-N}+1\right)\e^{-N}$
            & $\dfrac{N^2}{2}\e^{M\e^{-N}-M}M
              \left(M\e^{-N}+1\right)\e^{-N}$\\
        $n$
            & $\dfrac{\d[n]{}}{\D[n]{t}}\exp\left\{
              {M\e^{-N}\left(\e^t-1\right)}\right\}
              \bigg\rvert_{t=0}$
            & $\cdots$\\
        \bottomrule
    \end{tabular}
    \label{tb:moment}
\end{table}

\subsection{b)}
Soit $W=g(X)$, où $g(X)$ est le meilleur estimateur de $Y$ en fonction de $X$.
Par conséquent, on a
\begin{equation}\label{eq:estimateur}
    g(j)
    =\Espg{Y}{X=j}
    =\sum_{k=0}^\infty kp_{Y|X}(k|X=j)
    =\sum_{k=0}^\infty k\frac{p_{X,Y}(j,k)}{p_X(j)}.
\end{equation}

\subsubsection*{Fonction de masse de \boldmath $X$}
Par définition, on a
\begin{equation*}
    p_X(j)
    =\sum_{k=0}^\infty p_{X,Y}(j,k).
\end{equation*}

Si $j\neq 0$, alors on a
\begin{equation*}
    p_X(j)
    =\sum_{k=0}^\infty\frac{\e^{-M}M^j}{j!}
        \cdot\frac{\e^{-(Nj)}(Nj)^k}{k!}
    =\frac{M^j}{j!}\e^{-M}\e^{-(Nj)}\sum_{k=0}^\infty\frac{(Nj)^k}{k!}.
\end{equation*}
Hors, la somme est le développement en série de la fonction exponentielle,
alors on a
\begin{equation*}
    p_X(j)
    =\frac{M^j}{j!}\e^{-M}\e^{-(Nj)}\e^{(Nj)}
    =\frac{M^j}{j!}\e^{-M}.
\end{equation*}

Dans le cas où $j=0$, on a
\begin{equation*}
    p_X(j)
    =\e^{-M}+\underbrace{\sum_{k=1}^\infty\frac{\e^{-M}M^j}{j!}
        \cdot\frac{\e^{-(Nj)}(Nj)^k}{k!}}_{0}
    =\e^{-M}.
\end{equation*}
de sorte que
\begin{equation*}
    p_X(j)
    =\frac{M^j}{j!}\e^{-M}
\end{equation*}
est valide pour tout $j\geq 0$. On remarque alors que $X\sim\Poi{M}$.

\subsubsection*{Estimateur de \boldmath $Y$}
En continuant l'équation \eqref{eq:estimateur} pour tout $j\geq 0$, on a
\begin{equation*}
    g(j)
    =\sum_{k=0}^\infty k\frac{p_{X,Y}(j,k)}{p_X(j)}
    =\sum_{k=1}^\infty k\frac{p_{X,Y}(j,k)}{p_X(j)}
\end{equation*}
et donc
\begin{equation*}
    g(j)
    =\sum_{k=1}^\infty k\bigg[
        \frac{\e^{-M}M^j}{j!}\cdot\frac{\e^{-(Nj)}(Nj)^k}{k!}
    \bigg]\bigg[
        \frac{M^j}{j!}\e^{-M}
    \bigg]^{-1}
    =\sum_{k=1}^\infty k
        \frac{(Nj)^k}{k!}\e^{-(Nj)}
    =Nj,
\end{equation*}
car on remarque que c'est l'espérance d'une loi de Poisson de paramètre
$\alpha=Nj$.

\subsubsection*{Fonction de masse de \boldmath $W$}
Par définition, on a
\begin{equation*}
    p_W(k)
    =\P{W=k}
    =\P{g(X)=k}
    =\P{NX=k}
    =\P{X=\frac{k}{N}}.
\end{equation*}
Puisque $X\sim\Poi{M}$, alors
\begin{equation*}
    p_W(k)
    =\frac{M^{K/N}}{(K/N)!}\e^{-M}.
\end{equation*}

\pagebreak
\section*{Question 5}
\subsection{a)}
Soit $X$ une variable aléatoire discrète telle que
\begin{equation}\label{eq:eq1}
    \P{X=2}=O\Var{X}\P{X=1},
\end{equation}
où $O$ est une constante rationelle donnée par
\begin{equation*}
    O=2\frac{(9+a)}{(10+a)},\quad a=1,2,\dots,9,
\end{equation*}
et
\begin{equation}\label{eq:eq2}
    \phi_X(P\pi)=0,
\end{equation}
où $P$ une constante entière, positive et impaire.

Puisque deux équations sont fournies, il est fort probable que la distribution
de $X$ ait deux paramètres. Supposons que $X\sim\Bin{n}{p}$, où $n$ est un
nombre entier positif et $p$ une probabilité telle que $0\leq p\leq 1$.

Par conséquent, on a
\begin{equation*}
    \phi_X(\omega)
    =\left(p\e^{i\omega}+q\right)^n
    =\left(p\e^{i\omega}+1-p\right)^n
    =\left[p\left(\e^{i\omega}-1\right)+1\right]^n,
\end{equation*}
où $q=1-p$, de sorte qu'avec l'équation \eqref{eq:eq2}, on a
\begin{equation*}
    \phi_X(P\pi)
    =\left[p\left(\e^{iP\pi}-1\right)+1\right]^n
    =0.
\end{equation*}

Hors, $\e^{iP\pi}=\cos{P\pi}+i\sin{P\pi}=-1$ pour tout $P$ impair de sorte que
\begin{equation*}
    -2p+1=0\Leftrightarrow
    p=\frac{1}{2},
\end{equation*}
ce qui en accord avec l'hypothèse $0\leq p\leq 1$.

De plus, la variance et la fonction de masse d'une loi Binomial sont
\begin{equation*}
    \Var{X}=np(1-p)\quad\text{et}\quad
    p_X(k)=\binom{n}{k}p^kq^{n-k}
\end{equation*}
de sorte que l'équation \eqref{eq:eq1} devient
\begin{equation*}
    \binom{n}{2}p^2q^{n-2}
    =O\cdot npq\cdot \binom{n}{1}p^1q^{n-1}.
\end{equation*}

En simplifiant et en appliquant la définition d'un binôme, on a
\begin{equation*}
    \frac{n!}{2!(n-2)!}q^{-2}
    =O\cdot n\cdot\frac{n!}{1!(n-1)!}
\end{equation*}
et
\begin{equation*}
    n(n-1)\frac{1}{2q^2}=O\cdot n\cdot n
    \Rightarrow
    n-1=2q^2On
\end{equation*}
En isolant $n$, on obtient finalement
\begin{equation*}
    n
    =\frac{1}{1-2q^2O}.
\end{equation*}

Au dénominateur, on a
\begin{equation*}
    1-2q^2O
    =1-2\left(\frac{1}{2}\right)^2\left(2\frac{(9+a)}{(10+a)}\right)
    =1-\frac{(9+a)}{(10+a)}
    =\frac{(10+a)-(9+a)}{(10+a)}
    =\frac{1}{10+a}
\end{equation*}
de sorte qu'on a
\begin{equation*}
    n=10+a,
\end{equation*}
ce qui est en accord avec l'hypothèse que $n$ est une nombre entier
et positif. Avec $a=6$, ou $O=\nicefrac{15}{8}$, on a que
\begin{equation*}
    X\sim\Bin{\frac{1}{2}}{16}.
\end{equation*}

\subsection{b)}
Soit $Y$ une variable aléatoire continue telle que
\begin{equation}\label{eq:eq1b}
    \Espg{Y}{Y\geq Q}=R,
\end{equation}
et
\begin{equation}\label{eq:eq2b}
    \Pg{Y\leq S}{Y\leq T}=U,\quad S<T,
\end{equation}
où $Q$, $R$, $S$, $T$ et $U$ sont des constantes.

Puisque deux équations sont fournies, il est fort probable que la
distribution de $Y$ ait deux paramètres. Supposons que $Y$ suit la
distribution la plus simple à deux paramètres, soit $Y\sim\Uni{a}{b}$,
où $a$ et $b$ sont des constantes telles que $a<b$.

Par définition, on a que l'équation \eqref{eq:eq2b} devient
\begin{equation*}
    \Pg{Y\leq S}{Y\leq T}
    =\frac{\P{\left\{Y\leq S\right\}\cap\left\{Y\leq T\right\}}}{\P{Y\leq T}}
    =\frac{\P{Y\leq S}}{\P{Y\leq T}}
    =\frac{F_Y(S)}{F_Y(T)}
\end{equation*}
car $S<T$. On suppose que $a\leq S<T\leq b$, alors
\begin{equation*}
    \Pg{Y\leq S}{Y\leq T}
    =\frac{S-a}{b-a}\cdot\frac{b-a}{T-a}
    =\frac{S-a}{T-a}=U
\end{equation*}
de sorte à obtenir
\begin{equation*}
    a
    =\frac{UT-S}{U-1}
    =1.640,
\end{equation*}
avec $S=1.870$, $T=2.640$ et $U=0.230$.

Pour développer l'équation \eqref{eq:eq1b}, il faut déterminer la fonction de
densité. Premièrement, on a que
\begin{equation*}
    F_Y(y|Y\geq Q)
    =\frac{\P{\left\{Y\leq y\right\}\cap\left\{Y\geq Q\right\}}}{\P{Y\geq Q}}
    =\frac{\P{Q\leq Y\leq y}}{1-\P{Y\leq Q}}
    =\frac{F_Y(y)-F_Y(Q)}{1-F_Y(Q)},
\end{equation*}
si $Q\leq y$. Par conséquent, on a
\begin{equation*}
    f_Y(y|Y\geq Q)
    =\frac{\d{}}{\d{y}}F_Y(y|Y\geq Q)
    =\frac{f_Y(y)}{1-F_Y(Q)}.
\end{equation*}
Dans le cas où $y<Q$, on a $F_Y(y|Y\geq Q)=f_Y(y|Y\geq Q)=0$.

À la lumière de ce résultat, l'équation \eqref{eq:eq1b} devient
\begin{equation*}
    \Espg{Y}{Y\geq Q}
    =\int_{-\infty}^\infty yf_Y(y|Y\geq Q)\d{y}
    =\frac{1}{1-F_Y(Q)}\int_Q^\infty yf_Y(y)\d{y}.
\end{equation*}
Puisque $1.640=a<Q=4.760$ et $a\leq y\leq b$, on a
\begin{equation*}
    \Espg{Y}{Y\geq Q}
    =\frac{1}{1-F_Y(Q)}\int_Q^by\cdot\frac{1}{b-a}\d{y}
    =\frac{1}{1-F_Y(Q)}\frac{1}{b-a}\int_Q^by\d{y}.
\end{equation*}
Le facteur en avant de l'intégrale est
\begin{equation*}
    \frac{1}{1-F_Y(Q)}\cdot\frac{1}{b-a}
    =\frac{1}{1-\dfrac{Q-a}{b-a}}\cdot\frac{1}{b-a}
    =\frac{1}{b-Q}
\end{equation*}
de sorte que
\begin{equation*}
    \Espg{Y}{Y\geq Q}
    =\frac{1}{b-Q}\cdot\frac{y^2}{2}\bigg\rvert_Q^b
    =\frac{b^2-Q^2}{2(b-Q)}
    =\frac{(b-Q)(b+Q)}{2(b-Q)}
    =\frac{b+Q}{2}
    =R.
\end{equation*}
On obtient alors que
\begin{equation*}
    b
    =2R-Q
    =6.130,
\end{equation*}
ce qui est en accord avec l'hypothèse que $a<b$. Par conséquent,
\begin{equation*}
    Y\sim\Uni{1.640}{6.130}
\end{equation*}


\pagebreak
\section*{Annexe}
\begin{theoreme}\label{th:integrale}
    $\displaystyle\int_0^\infty x^m\e^{-xm}\d{x}=\dfrac{m!}{m^{m+1}}$.
\end{theoreme}

\begin{proof}
    Soit l'intégrale suivante :
    \begin{equation*}
        I=\int_0^\infty x^m\e^{-xm}\d{x},
    \end{equation*}
    où $m$ est un nombre entier positif. En posant
    \begin{equation*}
        u=x^m\Leftrightarrow\d{u}=mx^{m-1}\d{x}\quad\text{ et }\quad
        \d{v}=\e^{-mx}\d{x}\Leftrightarrow v=-\frac{1}{m}\e^{-mx},
    \end{equation*}
    on obtient par intégration par partie :
    \begin{equation*}
        I
        =-\frac{1}{m}x^m\e^{-mx}\bigg\rvert_0^\infty+
            \frac{m}{m}\int_0^\infty x^{m-1}\e^{-xm}\d{x}
        =\frac{m}{m}\int_0^\infty x^{m-1}\e^{-xm}\d{x}.
    \end{equation*}
    
    En appliquant une intégration par partie à nouveau avec
        \begin{equation*}
        u=x^{m-1}\Leftrightarrow\d{u}=(m-1)x^{m-2}\d{x}\quad\text{ et }\quad
        \d{v}=\e^{-mx}\d{x}\Leftrightarrow v=-\frac{1}{m}\e^{-mx},
    \end{equation*}
    on obtient
    \begin{equation*}
        I
        =-\frac{1}{m}x^{m-1}\e^{-mx}\bigg\rvert_0^\infty+
            \frac{m(m-1)}{m^2}\int_0^\infty x^{m-2}\e^{-xm}\d{x}
        =\frac{m(m-1)}{m^2}\int_0^\infty x^{m-2}\e^{-xm}\d{x}.
    \end{equation*}
    
    Après $n$ intégrations par partie, on remarque que
    \begin{equation*}
        I=\frac{m(m-1)\cdots(m-[n-1])}{m^n}\int_0^\infty x^{m-n}\e^{-xm}\d{x},
    \end{equation*}
    de sorte qu'en appliquant $n=m$ intégrations par partie, on obtient
    \begin{equation*}
        I=\frac{m(m-1)\cdots(2)(1)}{m^m}\int_0^\infty\e^{-xm}\d{x}.
    \end{equation*}

    Hors, le numérateur est la définition du factoriel de $m$ et l'intégrale
    à droite est
    \begin{equation*}
        \int_0^\infty\e^{-xm}\d{x}
        =-\frac{1}{m}\e^{-xm}\bigg\rvert_0^\infty
        =-\frac{1}{m}\left[\lim_{x\rightarrow\infty}\e^{-mx}-\e^0\right]
        =\frac{1}{m}.
    \end{equation*}
    de sorte que
    \begin{equation*}
        \int_0^\infty x^m\e^{-xm}\d{x}=\frac{m!}{m^{m+1}}.\qedhere
    \end{equation*}
\end{proof}

\begin{table}[H]
    \centering
	\caption{table des valeurs}
	\vspace{5mm}
    \documentclass[11pt]{article}
\usepackage[french]{babel}
\usepackage[utf8]{inputenc}
\usepackage[T1]{fontenc}
\usepackage{siunitx}
\usepackage{booktabs}

\begin{document}
\begin{tabular}{cS[table-format=5.3,table-auto-round]ccS[table-format=5.3,table-auto-round]}
    \toprule
        lettre & {nombre} && lettre & {nombre} \\
    \midrule
        A	&	0.71000004	&&	L	&	3.37\\
        B	&	0.71500003	&&	M	&	2.77\\
        C	&	2.95	&&	N	&	0.75\\
        D	&	1.59	&&	O	&	1.875\\
        E	&	0.77	&&	P	&	43.0\\
        F	&	1.78	&&	Q	&	4.67\\
        G	&	1.535	&&	R	&	5.4\\
        H	&	59.0	&&	S	&	1.87\\
        I	&	1.16	&&	T	&	2.6399999\\
        J	&	0.265	&&	U	&	0.22999999\\
        K	&	1.76	&&		&	\\
    \bottomrule
\end{tabular}
\end{document}

\end{table}
\end{document}


