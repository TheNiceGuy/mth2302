\documentclass[11pt]{article}
\usepackage[francais]{babel}
\usepackage[utf8]{inputenc}
\usepackage[T1]{fontenc}
\usepackage{standalone}
\usepackage{amsmath}
\usepackage{amssymb}
\usepackage{amsthm}
\usepackage{geometry}
\usepackage{siunitx}
\usepackage{gphys}
\usepackage{thmtools}
\usepackage{mdframed}
\usepackage{float}
\usepackage{tikz}
\usepackage{contour}
\usepackage{scrextend}
\usepackage{subcaption}
\usepackage{booktabs}
\usepackage{nicefrac}
\usepackage{pgfplots}

\usetikzlibrary{babel}
\usetikzlibrary{calc}
\usetikzlibrary{patterns}
\usetikzlibrary{decorations.pathreplacing}
\usetikzlibrary{decorations.markings}

%%%%%%%%%%%%%%%%%%
% Configurations %
%%%%%%%%%%%%%%%%%%

\geometry{
	top=34mm,
	bottom=34mm
}

%%%%%%%%%%
% Macros %
%%%%%%%%%%

\makeatletter
\pgfmathdeclarefunction{erf}{1}{%
    \begingroup
        \pgfmathparse{#1 > 0 ? 1 : -1}%
        \edef\sign{\pgfmathresult}%
        \pgfmathparse{abs(#1)}%
        \edef\x{\pgfmathresult}%
        \pgfmathparse{1/(1+0.3275911*\x)}%
        \edef\t{\pgfmathresult}%
        \pgfmathparse{%
            1-(((((1.061405429*\t-1.453152027)*\t)+1.421413741)*\t%
            -0.284496736)*\t+0.254829592)*\t*exp(-(\x*\x))}%
        \edef\y{\pgfmathresult}%
        \pgfmathparse{(\sign)*\y}%
        \pgfmath@smuggleone\pgfmathresult%
    \endgroup
}
\makeatother

\newcommand\card{%
	\ensuremath{%
		\mathrm{card}\,%
	}%
}%

\newcommand\comb[2]{%
	\ensuremath{%
		\mathcal{C}_{#2}^{#1}%
	}%
}%

\newcommand\perm[2]{%
	\ensuremath{%
		\mathcal{P}_{#2}^{#1}%
	}%
}%

\renewcommand\P{%
	\ensuremath{%
		\mathrm{P}
	}%
}%

\newtheorem{axiome}{Axiome}
\newtheorem{theoreme}{Théoreme}[section]

\declaretheoremstyle[
	notefont=\bfseries,
	notebraces={}{},
	bodyfont=\normalfont,
	postheadspace=0.5em,
	numbered=no,
]{basicstyle}
\declaretheorem[name=Définition,style=basicstyle]{definition}

\newmdtheoremenv[
	linecolor=black,
	backgroundcolor=gray!40,
	ntheorem
]{exemple}{Exemple}[section]

%%%%%%%%%%%%%
% Documents %
%%%%%%%%%%%%%

\begin{document}
\tableofcontents
\pagebreak

\section{Élèments de probabilités}
\begin{definition}
	Une expérience est \textit{aléatoire} si un observateur peut la répéter
	dans les mêmes conditions, mais sans pouvoir en prédire le résultat.
\end{definition}

\begin{definition}
	Un \textit{espace échantilion} est un ensemble $S$ des résultats possibles.
\end{definition}

Un espace d'échantilion peut être \textit{qualitatif} ou \textit{quantitatif},
ainsi que \textit{discret}, \textit{continu} ou \textit{mixte}. Il peut aussi
être \textit{dénombrable} ou \textit{non-dénombrable}.

\begin{definition}
	Un \textit{évènement} $A$ est un sous-ensemble de $S$ d'intêret à
	l'observateur.
\end{definition}

\begin{definition}
	Un \textit{évènement élémentaire} $A$ est un résultat particulier,
	c'est-à-dire, un élèment de $S$.
\end{definition}

La différence entre les deux dernières définitions est que la $\card(A)=1$ pour
un évènement élémentaire tandis que $\card(A)\geq 1$ pour un évènement.

\begin{exemple}
	On observe le résultat du lancer de deux pièces de monnaie. On note $P$
	comme un lancer pile et $F$ comme un lancer face. L'ensemble est donc
	\begin{equation*}
		S=\left\{PP, FF, PF, FP\right\}
	\end{equation*}
	avec chaque résultat ayant \SI{25}{\percent} d'arriver. L'ensemble $S$ est
	qualitatif, soit pile ou face, et discret.
\end{exemple}

\begin{exemple}
	On observer la somme obtenue lors du lancer de deux dés à 6 faces.
	L'ensemble de résultat possible est
	\begin{equation*}
		S=\left\{2,3,4,5,6,7,8,9,10,11,12\right\}.
	\end{equation*}
	Il est quantitatif et discret.
\end{exemple}

\begin{exemple}
	On compte le nombre de lancers d'une pièce pour obtenir une première fois
	un pile. L'espace échantilion est
	\begin{equation*}
		S=\left\{1,2,3,4,5,\dots\right\},
	\end{equation*}
	car il est possible qu'une grande quantité de lancer est effectuée avant
	d'obtenir un pile. L'ensemble $S$ est quantitatif, discret et
	infini dénombrable.
\end{exemple}

\pagebreak
\begin{exemple}
	On mesure le temps d'attente à l'arrêt d'autobus. L'espace échantilion est
	\begin{equation*}
		S=\left[0,\infty\right[.
	\end{equation*}
	L'ensemble $S$ est quantitatif, continu et infini non-dénombrable.
\end{exemple}

\begin{exemple}
	On mesure le temps d'attente à l'arrêt d'autobus ainsi que le nombre de
	personnes en file à l'arrivée de l'autobus. L'espace échantilion est
	\begin{equation*}
		S=T\times U,
	\end{equation*}
	avec
	\begin{equation*}
		T=\left[0,\infty\right[
	\end{equation*}
	et
	\begin{equation*}
		U=\left\{1,2,3,4,5,\dots\right\}.
	\end{equation*}
	L'ensemble $S$ est quantitatif et mixte. Un exemple d'évènement élèmentaire
	peut être un couple tel que $\left(\SI{4.25}{\second}, 4\right)$.
\end{exemple}

\subsection{Lien entre l'expérience aléatoire et son modèle}
\begin{definition}
	La \textit{fréquence relative} $f_A$ d'un évènement $A$ est le rapport
	entre le nombre d'observations $n_A$ de l'évèmenent et le nombre $n$ de
	répétition de l'expérience, c'est-à-dire
	\begin{equation*}
		f_A=\frac{n_A}{n}.
	\end{equation*}
	La limite lorsque l'expérience est répétée infiniment est la probabilité de
	l'évènement $A$, dénotée
	\begin{equation*}
		\P(A)=\lim_{n\rightarrow\infty}f_A.
	\end{equation*}
\end{definition}

\subsection{Opérations sur les ensembles}
Soit deux ensembles $A$ et $B$ tel que $A,B\subset S$. La figure
\ref{fig:venn_intersection} montre une intersection tandis que la figure
\ref{fig:venn_union} montre une union entre $A$ et $B$. La figure
\ref{fig:venn_complement} montre le complémenet de $A$ et la figure
\ref{fig:venn_exclusion} montre l'exclusion de deux ensembles.

\begin{figure}[H]
	\centering
	\begin{subfigure}[H]{0.21\textwidth}
		\centering
		\documentclass[11pt]{article}
\usepackage[french]{babel}
\usepackage[utf8]{inputenc}
\usepackage[T1]{fontenc}
\usepackage{tikz}
\usepackage{standalone}
\usepackage{contour}

\usetikzlibrary{patterns}
\usetikzlibrary{babel}

\begin{document}
\contourlength{1.5pt}

\begin{tikzpicture}
\draw (0,0) -- (3,0) -- (3,2) -- (0,2) -- cycle;
\draw (1,1) coordinate (A) circle (8mm);
\draw (2,1) coordinate (B) circle (8mm);

\begin{scope}
    \clip (A) circle (8mm);
    \path[fill,pattern=north west lines]
          (B) circle (8mm);
\end{scope}

\node at (A) {\contour{white}{$A$}};
\node at (B) {\contour{white}{$B$}};
\node at (0.2,0.2) {\contour{white}{$S$}};
\end{tikzpicture}
\end{document}
		\caption{$A\cap B$}
		\label{fig:venn_intersection}
	\end{subfigure}
	~
	\begin{subfigure}[H]{0.21\textwidth}
		\centering
		\documentclass[11pt]{article}
\usepackage[french]{babel}
\usepackage[utf8]{inputenc}
\usepackage[T1]{fontenc}
\usepackage{tikz}
\usepackage{standalone}
\usepackage{contour}

\usetikzlibrary{patterns}
\usetikzlibrary{babel}

\begin{document}
\contourlength{1.5pt}

\begin{tikzpicture}
\begin{scope}
    \path[postaction={fill,pattern=north west lines}]
          (1,1) coordinate (A) circle (8mm)
          (2,1) coordinate (B) circle (8mm);
\end{scope}

\draw (0,0) -- (3,0) -- (3,2) -- (0,2) -- cycle;
\draw (A) circle (8mm);
\draw (B) circle (8mm);

\node at (A) {\contour{white}{$A$}};
\node at (B) {\contour{white}{$B$}};
\node at (0.2,0.2) {\contour{white}{$S$}};
\end{tikzpicture}
\end{document}


		\caption{$A\cup B$}
		\label{fig:venn_union}
	\end{subfigure}
	~
	\begin{subfigure}[H]{0.21\textwidth}
		\centering
		\documentclass[11pt]{article}
\usepackage[french]{babel}
\usepackage[utf8]{inputenc}
\usepackage[T1]{fontenc}
\usepackage{tikz}
\usepackage{standalone}
\usepackage{contour}

\usetikzlibrary{patterns}
\usetikzlibrary{babel}

\begin{document}
\contourlength{1.5pt}

\begin{tikzpicture}
\draw[even odd rule, fill,pattern=north west lines]
      (0,0) -- (3,0) -- (3,2) -- (0,2) -- cycle
      (1,1) coordinate (A) circle (8mm);
\draw (2,1) coordinate (B) circle (8mm);

\node at (A) {\contour{white}{$A$}};
\node at (B) {\contour{white}{$B$}};
\node at (0.2,0.2) {\contour{white}{$S$}};
\end{tikzpicture}
\end{document}


		\caption{$A^c$}
		\label{fig:venn_complement}
	\end{subfigure}
	~
	\begin{subfigure}[H]{0.21\textwidth}
		\centering
		\documentclass[11pt]{article}
\usepackage[french]{babel}
\usepackage[utf8]{inputenc}
\usepackage[T1]{fontenc}
\usepackage{tikz}
\usepackage{standalone}
\usepackage{contour}

\usetikzlibrary{patterns}
\usetikzlibrary{babel}

\begin{document}
\contourlength{1.5pt}

\begin{tikzpicture}
\draw (0,0) -- (3,0) -- (3,2) -- (0,2) -- cycle;
\draw (1,1) coordinate (A) circle (8mm);
\draw (2,1) coordinate (B) circle (8mm);

\path[even odd rule,fill,pattern=north west lines]
      (A) circle (8mm)
      (B) circle (8mm);

\node at (A) {\contour{white}{$A$}};
\node at (B) {\contour{white}{$B$}};
\node at (0.2,0.2) {\contour{white}{$S$}};
\end{tikzpicture}
\end{document}


		\caption{$\left(A\cup B\right)\setminus\left(A\cap B\right)$}
		\label{fig:venn_exclusion}
	\end{subfigure}
	\caption{Opérations d'ensembles démontrées sur des diagrammes de Venn}
\end{figure}

\subsection{Axiomes fondamentales de la probabilité}
\begin{axiome}
	La probabilité d'un évèment $A$ est plus grand ou égal à 0, c'est-à-dire
	\begin{equation*}
		\P(A)\geq 0,
	\end{equation*}
	pour tout $A\in S$.
\end{axiome}

\begin{axiome}
	La probabilité de l'espace d'échantilion $S$ est 1, c'est-à-dire
	\begin{equation*}
		\P\left(S\right)=1.
	\end{equation*}
\end{axiome}

\begin{axiome}
	La probabilité d'un évènement $A$ ou d'un évènement $B$ est équivalent à la
	somme de leur probabilité, c'est-à-dire
	\begin{equation*}
		\P(A\cup B)=\P(A)+\P(B),
	\end{equation*}
	si $A\cup B=\emptyset$.

	En général, 
	\begin{equation*}
		\P\left(\bigcup_{k=1}^\infty A_k\right)=\sum_{k=1}^\infty \P\left(A_k\right),
	\end{equation*}
	si $A_i\cup A_j=\emptyset$, $\forall i,j$.
\end{axiome}

\begin{theoreme}
	$\P(A^c)=1-\P(A)$.
\end{theoreme}

\begin{proof}
	On sait que $A\cup A^c=S$ et $A\cap A^c=\emptyset$. Hors,
	$\P(A\cup A^c)=\P(S)\Leftrightarrow \P(A)+\P(A^c)=1$, car
	$A\cup A^c=\emptyset$.
	En réarrangeant, on obtient que $\P(A^c)=1-\P(A).$
\end{proof}

\begin{theoreme}
	$\P(A)\leq 1$.
\end{theoreme}

\begin{proof}
	On sait que $\P(A^c)\geq 0$ et $\P(A^c)=1-\P(A)$. En réarrangeant, on
	obtient que $\P(A)\leq 1$.
\end{proof}

\begin{theoreme}
	$\P(\emptyset)=0$.
\end{theoreme}

\begin{proof}
	On sait que $S^c=\emptyset$. Par conséquent, $\P(\emptyset)=1-\P(S)=1-1=0$.
\end{proof}

\begin{theoreme}
	$A\subset B\Rightarrow \P(A)\leq \P(B)$.
\end{theoreme}

\begin{proof}
	La différence entre $A$ et $B$ est $A^c\cap B$ de sorte qu'on peut écrire
	$B=A\cup(A^c\cap B)$. Par conséquent, $\P(B)=\P(A)+\P(A^c\cap B)
	\Leftrightarrow \P(B)-\P(A)=\P(A^c\cap B)\geq 0$, car
	$A\cup(A^c\cap B)=\emptyset$. En réarrangeant, on obtient $\P(A)\leq \P(B)$.
\end{proof}

\begin{theoreme}
	$\P(A\cup B)=\P(A)+\P(B)-\P(A\cap B)$.
\end{theoreme}

\subsection{Principe d'équiprobabilité}
\subsubsection{Ensemble fini}
On suppose que $S$ est fini, soit $S=\left\{e_1,e_2,\dots,e_n\right\}$. On dit
que les résultats d'une expérience aléatoire sont \textit{équiprobables} si
\begin{equation*}
	\P(e_1)=\P(e_2)=\cdots=\P(e_n)=\frac{1}{n}.
\end{equation*}
Dans ce cas, on a que la probabilité d'un évènement $A\subset S$ est
\begin{equation*}
	\P(A)=\frac{n_A}{n},
\end{equation*}
où $n_A$ est le nombre d'élément dans $A$ et $n$ celui dans $S$.

\subsubsection{Ensemble infini dénombrable}
On suppose que $S$ est infini dénombrable, alors l'équiprobabilité est
impossible. Si la probabilité d'un évènement élémentaire est $\P(e_i)=\epsilon$,
on obtient que $\P(S)=\P(e_1)+\P(e_2)+\cdots=\infty$, Ce qui est en contradiction
avec les axiomes. D'une manière similaire, si $\P(e_i)=0$, alors $\P(S)=0$, ce
qui est aussi en contradiction avec les axiomes.

\subsubsection{Ensemble infini non-dénombrable}
On suppose que $S=[a,b]$ est infini non-dénombrable. Soit un évènement
$A=[c,d]\subset S$. La probabilité de l'évènement est
\begin{equation*}
	\P(A)=\frac{l_A}{l}=\frac{d-c}{b-a},
\end{equation*}
où $l_A$ est la longueur de $A$ et $l$ la longueur de $S$. Par conséquent, la
probabilité d'un évènement élémentaire $e\in S$ est
\begin{equation*}
	\P(e)=0,
\end{equation*}
car $l_A=0$.

\begin{exemple}
	On calcule la somme de deux dés lancés. Quelle est la probabilité d'obtenir
	chaque somme possible?
	
	L'espace échantilion est $S=L\times L$, où $L=\left\{1,2,3,4,5,6\right\}$
	est le résultat possible d'un dé. On peut écrire
	$S=\left\{(1,1),(1,2),\dots,(6,6)\right\}$. On suppose qu'il y a
	équiprobabilité de sorte que la probabilité d'obtenir un évènement
	élémentaire est $\nicefrac{1}{36}$.

	Hors, certaines des sommes sont dupliquées de sorte que la probabilité
	d'obtenir une somme particulière est donnée par la table suivante.
	\begin{table}[H]
		\centering
		\begin{tabular}{r|ccccccccccc}
			$\phantom{\P(}A\phantom{)}$ &
			$2$ & $3$ & $4$ & $5$ & $6$ & $7$ & $8$ & $9$ & $10$ & $11$ & $12$\\
			$\P(A)$ &
			$\frac{1}{36}$ & $\frac{2}{36}$ & $\frac{3}{36}$ & $\frac{4}{36}$ &
			$\frac{5}{36}$ & $\frac{6}{36}$ & $\frac{5}{36}$ & $\frac{4}{36}$ &
			$\frac{3}{36}$ & $\frac{2}{36}$ & $\frac{1}{36}$\\
		\end{tabular}
	\end{table}
\end{exemple}

\section{Analyse combinatoire}
\subsection{Diagramme en arbre}

\begin{exemple}
	On lance un dé, puis une pièce de monnaie. Combien de résultats est-il
	possible?
	
	On peut représenté la situation par l'arbre suivant.
	\begin{figure}[H]
		\centering
		\documentclass[11pt]{article}
\usepackage[french]{babel}
\usepackage[utf8]{inputenc}
\usepackage[T1]{fontenc}
\usepackage{tikz}
\usepackage{standalone}

\usetikzlibrary{babel}

\begin{document}
\begin{tikzpicture}[
    level distance=8mm,
    L1/.style ={sibling distance=4em},
    L2/.style ={sibling distance=2em}]
\draw[fill] circle (1pt);
\node[inner sep=0mm] {}
    child[L1] {
        node {1}
            child[L2] {node {P}}
            child[L2] {node {F}}
    }
    child[L1] {
        node {2}
            child[L2] {node {P}}
            child[L2] {node {F}}
    }
    child[L1] {
        node {3}
            child[L2] {node {P}}
            child[L2] {node {F}}
    }
    child[L1] {
        node {4}
            child[L2] {node {P}}
            child[L2] {node {F}}
    }
    child[L1] {
        node {5}
            child[L2] {node {P}}
            child[L2] {node {F}}
    }
    child[L1] {
        node {6}
            child[L2] {node {P}}
            child[L2] {node {F}}
    }
    ;
\end{tikzpicture}
\end{document}


	\end{figure}
	Par le principe de multiplication, il y a $6\cdot 2=12$ possibilités.
\end{exemple}

\subsection{Permutations}
\begin{definition}
	Une \textit{permutation} correspond à un choix de $k$ objets parmi $n$
	objets distincts. Le choix se fait sans remise et dans un ordre spécifique.
\end{definition}

La table \ref{tb:permutation} résume le principe d'une permutation à l'aide
d'une pige d'objet. À chaque pige, la quantité d'objets diminue de 1.

\begin{table}[H]
	\centering
	\caption{Pige d'objets}
	\begin{tabular}{r|cccc}
		\toprule
		Choix d'objet   &   1 &   2   & $\dots$ & k\\
		Objets restants & $n$ & $n-1$ & $\dots$ & $n-(n-k)$\\
		\bottomrule
	\end{tabular}
	\label{tb:permutation}
\end{table}

À l'aide du principe de multiplication, il y a $n\cdot(n-1)\cdots(n-k+1)$
combinaisons. On dénote le nombre de permutations sans remise de $n$ éléments
par
\begin{equation*}
	\perm{k}{n}=\frac{n!}{(n-k)!}.
\end{equation*}
Lorsqu'il y a remise, le nombre de permutations de $n$ éléments est $n^k$.

\begin{exemple}
	On dispose de 10 composantes dont 4 défectueuses. On pige 3 composantes au
	hasard et sans remise. Quelle est la probabilité d'obtenir uniquement des
	composantes non défectueuses?

	On suppose qu'il y a équiprobabilité du choix des 3 composantes. La
	probabilité d'obtenir 3 composantes non défecteuses $F$ est
	\begin{equation*}
		\P(F)
		=\frac{6\cdot 5\cdot 4}{10\cdot 9\cdot 8}
		=\frac{\perm{3}{6}}{\perm{3}{10}}
		=\frac{1}{6}.
	\end{equation*}
\end{exemple}

\subsection{Combinaisons}
\begin{definition}
	Une \textit{combinaison} correspond au choix de $k$ objets parmis $n$
	objets distincts. Le choix se fait sans remise et sans ordre spécifique.
\end{definition}

Soit $C_k^n$ le nombre de combinaisons. Le nombre de permutations est égal au
nombre de combinaisons multiplié par le nombre d'arrangement possible $k!$,
c'est-à-dire 
\begin{equation*}
	\perm{k}{n}=\comb{k}{n}\cdot k!,
\end{equation*}
et en réarrangeant, on obtient
\begin{equation*}
	\comb{k}{n}={{n}\choose{k}}=\frac{n!}{k!(n-k)!}.
\end{equation*}

\begin{exemple}
	Combien de codes de deux lettres peut-on former à partir du mot
	\texttt{OUI}?
	
	Avec ordre, il y a $\perm{2}{3}=6$ permutations et sans ordre, il y a
	$\comb{2}{3}=3$ combinaisons.
\end{exemple}

\begin{exemple}
	Quel est la probabilité de gagner le gros lot à la $6/49$?
	
	Sans ordre, il y a 1 seul cas favorable et $\comb{6}{49}$ cas possibles.
	Par conséquent, la probabilité de gagner $G$ est
	\begin{equation*}
		\P(G)=\frac{1}{\comb{6}{49}}=\frac{1}{\SI{13983816}{}}.
	\end{equation*}
	Avec ordre, il y a $6!$ cas favorables et $\perm{6}{49}$ cas possibles de
	sorte que la probabilité est
	\begin{equation*}
		\P(G)=\frac{6!}{\perm{6}{49}}=\frac{720}{\SI{10068347520}{}}.
	\end{equation*}
\end{exemple}

\subsubsection{Triangle de Pascal}
Une combinaison peut se représenter avec le triangle de Pascal comme le montre
la figure \ref{fig:pascal}.

\begin{figure}[H]
	\centering
	\documentclass[11pt]{article}
\usepackage[french]{babel}
\usepackage[utf8]{inputenc}
\usepackage[T1]{fontenc}
\usepackage{standalone}

\begin{document}
\begin{tabular}{ccccccccc}
    &&&& $\comb{0}{0}$\\
    &&&  $\comb{0}{1}$ && $\comb{1}{1}$\\
    &&   $\comb{0}{2}$ && $\comb{1}{2}$ && $\comb{2}{2}$\\
    &    $\comb{0}{3}$ && $\comb{1}{3}$ && $\comb{2}{3}$ && $\comb{3}{3}$\\
         $\comb{0}{4}$ && $\comb{1}{4}$ && $\comb{2}{4}$ && $\comb{3}{4}$ && $\comb{4}{4}$\\
\end{tabular}
\end{document}


	\caption{représentation du triangle de Pascal}
	\label{fig:pascal}
\end{figure}

\begin{theoreme}
	$\comb{k}{n}=\comb{k-1}{n-1}+\comb{k}{n-1}$
\end{theoreme}

\begin{proof}
	La démonstration est triviale et est laissée au lecteur.
\end{proof}

\subsubsection{Binôme de Newton}
La combinaison est souvent appliquée dans le cas de la puissance d'un binôme
tel que
\begin{equation*}
	\left(a+b\right)^n
	=\sum_{k=0}^n\comb{k}{n}\cdot a^k\cdot b^{n-k}
	=\sum_{k=0}^n{{k}\choose{n}}\cdot a^k\cdot b^{n-k},
\end{equation*}
mais elle est plus souvent utilisée avec la deuxième notation.

\subsection{Permutation d'objets semblables}
\begin{exemple}
	Combien y a-t-il d'ordres possibles des lettres <<ppfff>>?

	Soit 5 cases distinctes représentant l'ordre d'une pige dans les lettres.
	Il faut choisir les cases où mettre les <<p>>, soit
	\begin{equation*}
		\comb{2}{5}=\frac{5!}{2!3!}=10
	\end{equation*}
	ordres possibles.
	
	Si on échange un <<p>> et un <<f>> dans une séquence $A$, on obtient une
	séquence $B$ différente de $A$. Si on échange un <<p>> avec une autre <<p>>
	dans une séquence $A$, on retrouve la même séquence $A$.

	En analysant la formule, le facteur $5!$ représente le nombre d'ordres si
	toutes les lettres étaient différentes. Le facteur $2!$ représente les
	<<p>> s'ils étaient différents et le facteur $3!$ représente le nombres des
	<<f>> s'ils étaient différents.
\end{exemple}

En général, avec $n$ objets objets comprenant $n_1$ objets de classes $1$,
$n_2$ objets de classe $2$, $\dots$, $n_k$ objets à la classe $k$, on a
\begin{equation*}
	\frac{n!}{n_1!n_2!\cdots n_k!}
\end{equation*}
ordres possibles.

\begin{exemple}\label{ex:network}
	Soit le graphe $n\times n$ suivant.
	\begin{figure}[H]
		\vspace{-3mm}
		\centering
		\documentclass[11pt]{article}
\usepackage[french]{babel}
\usepackage[utf8]{inputenc}
\usepackage[T1]{fontenc}
\usepackage{tikz}
\usepackage{standalone}

\usetikzlibrary{babel}
\usetikzlibrary{calc}
\usetikzlibrary{decorations.pathreplacing}

\begin{document}

% better vertically centered dots
\makeatletter
\DeclareRobustCommand{\rvdots}{%
    \vbox{%
        \baselineskip4\p@\lineskiplimit\z@%
        \kern-\p@%
        \hbox{.}\hbox{.}\hbox{.}%
    }%
}%
\makeatother

% a very large square used to invert a path
\tikzstyle{reverseclip}=[insert path={
    (-100,-100) -- (-100, 100) -- ( 100, 100) -- ( 100,-100)}
]

\begin{tikzpicture}[scale=0.8]
\def\n{3}

% draw all the lines
\begin{scope}
    % temporary constants
    \def\t{\n-1}
    \def\p{\n+1}

    % draw the masks    
    \begin{pgfinterruptboundingbox}
        \clip ($(-1,\t)!0.2!(-1,\n)$) -- ($(\p,\t)!0.2!(\p,\n)$)
           -- ($(\p,\t)!0.8!(\p,\n)$) -- ($(-1,\t)!0.8!(-1,\n)$)
           -- cycle [reverseclip];

        \clip ($(\t,-1)!0.2!(\n,-1)$) -- ($(\t,-1)!0.8!(\n,-1)$)
           -- ($(\t,\p)!0.8!(\n,\p)$) -- ($(\t,\p)!0.2!(\n,\p)$)
           -- cycle [reverseclip];
    \end{pgfinterruptboundingbox}

    \draw (0,0) grid (\n,\n);
\end{scope}

% draw all the nodes
\foreach\x in {0,...,\n} {
    \foreach\y in {0,...,\n} {
        \draw[fill] (\x,\y) circle (2pt);
    }
}

% draw the dots
\foreach\i in {0,...,\n} {
    \path (\i,\n-1) -- (\i,\n) node[midway] {$\rvdots$};
    \path (\n-1,\i) -- (\n,\i) node[midway] {$\ldots$};
}

% draw the braces
\draw[decorate,decoration={brace,amplitude=10pt,raise=4pt}]
    (0, 0) -- ( 0,\n) node [black,midway,xshift=-0.8cm] {$n$};
\draw[decorate,decoration={brace,amplitude=10pt,raise=4pt}]
    (0,\n) -- (\n,\n) node [black,midway,yshift= 0.8cm] {$n$};

% name the node of interest
\node[xshift= 2mm,yshift= 2mm] at (0 ,0 ) {$A$};
\node[xshift=-2mm,yshift=-2mm] at (\n,\n) {$B$};

\end{tikzpicture}
\end{document}


		\vspace{-3mm}
	\end{figure}

	Combien de chemins existe entre $A$ et $B$ suivant, s'il est seulement
	permis d'aller vers la droite ou vert le haut?
\end{exemple}
\addtocounter{exemple}{-1}
\begin{exemple}[suite]
	On suppose que tous les chemins possibles sont équiprobables. Peu importe
	le chemin, il faut avancer $n-1$ vers la droite et $n-1$ vers le haut pour
	un total de $2n-2$ mouvement.
	
	Il suffit de calculer le nombre de permutations de ces mouvements sachant
	qu'il y des objets semblables. Le nombre chemins est
	\begin{equation*}
		\frac{\left(2n-2\right)!}{\left(n-1\right)!\left(n-1\right)!},
	\end{equation*}
	ce qui est équivalent à $\comb{n-1}{2n-2}$.
\end{exemple}


\subsection{Équivalence}
\begin{exemple}\label{ex:network_broken}
	Soit le graphe $n\times n$ suivant.
	\begin{figure}[H]
		\vspace{-3mm}
		\centering
		\documentclass[11pt]{article}
\usepackage[french]{babel}
\usepackage[utf8]{inputenc}
\usepackage[T1]{fontenc}
\usepackage{tikz}
\usepackage{standalone}

\usetikzlibrary{babel}
\usetikzlibrary{calc}
\usetikzlibrary{decorations.pathreplacing}

\begin{document}

% better vertically centered dots
\makeatletter
\DeclareRobustCommand{\rvdots}{%
    \vbox{%
        \baselineskip4\p@\lineskiplimit\z@%
        \kern-\p@%
        \hbox{.}\hbox{.}\hbox{.}%
    }%
}%
\makeatother

% a very large square used to invert a path
\tikzstyle{reverseclip}=[insert path={
    (-100,-100) -- (-100, 100) -- ( 100, 100) -- ( 100,-100)}
]

\begin{tikzpicture}[scale=0.8]
\def\n{3}
\draw[dashed] (0,1) -- (\n-1,\n);

% draw all the lines
\begin{scope}
    % temporary constants
    \def\t{\n-1}
    \def\p{\n+1}

    % draw the masks    
    \begin{pgfinterruptboundingbox}
        \clip ($(-1,\t)!0.2!(-1,\n)$) -- ($(\p,\t)!0.2!(\p,\n)$)
           -- ($(\p,\t)!0.8!(\p,\n)$) -- ($(-1,\t)!0.8!(-1,\n)$)
           -- cycle [reverseclip];

        \clip ($(\t,-1)!0.2!(\n,-1)$) -- ($(\t,-1)!0.8!(\n,-1)$)
           -- ($(\t,\p)!0.8!(\n,\p)$) -- ($(\t,\p)!0.2!(\n,\p)$)
           -- cycle [reverseclip];
    \end{pgfinterruptboundingbox}

    \draw (0,0) grid (\n,\n);
\end{scope}

% draw all the nodes
\foreach\x in {0,...,\n} {
    \foreach\y in {0,...,\n} {
        \draw[fill] (\x,\y) circle (2pt);
    }
}

% draw the dots
\foreach\i in {0,...,\n} {
    \path (\i,\n-1) -- (\i,\n) node[midway] {$\rvdots$};
    \path (\n-1,\i) -- (\n,\i) node[midway] {$\ldots$};
}

% draw the braces
\draw[decorate,decoration={brace,amplitude=10pt,raise=4pt}]
    (0, 0) -- ( 0,\n) node [black,midway,xshift=-0.8cm] {$n$};
\draw[decorate,decoration={brace,amplitude=10pt,raise=4pt}]
    (0,\n) -- (\n,\n) node [black,midway,yshift= 0.8cm] {$n$};

% name the node of interest
\node[xshift= 2mm,yshift= 2mm] at (0 ,0 ) {$A$};
\node[xshift=-2mm,yshift=-2mm] at (\n,\n) {$B$};

\end{tikzpicture}
\end{document}


		\vspace{-3mm}
	\end{figure}

	Quel est la probabilité qu'un chemin choisi au hasard, allant seulement
	vers la droite ou le haut, ne traverse pas les noeuds au-dessus de la
	diagonale entre $A$ et $B$?

	Les <<bons>> chemins ne passent pas par la ligne critique en pointillée et
	les <<mauvais>> chemins passent par la ligne critique. La probabilité d'un
	bon chemin $B$ peut s'écrire en fonction de son complément, soit
	\begin{equation*}
		\P(B)=1-\P(M)=1-\frac{n_m}{C_{n-1}^{2n-2}},
	\end{equation*}
	où $M$ est un mauvais chemin et $n_m$ est le nombre de mauvais chemins.

	Soit un mauvais chemin $M$. On définit le point $X$ comme étant le premier
	point au-dessus de la diagonale que $M$ atteint. On applique une
	transformation mirroir au chemin suivant $X$ comme la prochaine figure.

	\noindent
	\begin{minipage}{0.5\textwidth}
		\begin{figure}[H]
			\centering
			\documentclass[11pt]{article}
\usepackage[french]{babel}
\usepackage[utf8]{inputenc}
\usepackage[T1]{fontenc}
\usepackage{tikz}
\usepackage{standalone}

\usetikzlibrary{babel}

\begin{document}
\begin{tikzpicture}[scale=0.7]
\def\n{3}
\draw[dashed] (0,1) -- (\n-1,\n);

% draw all the nodes
\foreach\x in {0,...,\n} {
    \foreach\y in {0,...,\n} {
        \draw[fill] (\x,\y) circle (2pt);
    }
}

\draw (0,0) -- (1,0) -- (1,2) -- (3,2) -- (3,3);

% name the node of interest
\node[above right] at (0 ,0 ) {$A$};
\node[below left]  at (\n,\n) {$B$};
\node[left] at (1,2) {$X$};

\end{tikzpicture}
\end{document}


		\end{figure}
	\end{minipage}%
	\begin{minipage}{0.5\textwidth}
		\begin{figure}[H]
			\centering
			\documentclass[11pt]{article}
\usepackage[french]{babel}
\usepackage[utf8]{inputenc}
\usepackage[T1]{fontenc}
\usepackage{tikz}
\usepackage{standalone}

\usetikzlibrary{babel}

\begin{document}
\begin{tikzpicture}[scale=0.7]
\def\n{3}
\draw[dashed] (0,1) -- (\n,\n+1);

% draw all the nodes
\foreach\x in {0,...,2} {
    \foreach\y in {0,...,4} {
        \draw[fill] (\x,\y) circle (2pt);
    }
}

\draw (0,0) -- (1,0) -- (1,4) -- (2,4);
\draw[dotted] (1,2) -- (3,2) -- (3,3);

\foreach\y in {0,...,4} {
    \path[fill=gray] (3,\y) circle (2pt);
}

% name the node of interest
\node[above right] at (0 ,0 ) {$A$};
\node[below left]  at (\n,\n) {$B$};
\node[left] at (1,2) {$X$};

\end{tikzpicture}
\end{document}


		\end{figure}
	\end{minipage}
\end{exemple}
\addtocounter{exemple}{-1}
\begin{exemple}[suite]
	Il s'avère que tous les mauvais chemins dans le graphe $n\times n$
	correspond à un chemin unique dans le graphe $(n-1)\times(n+1)$.
	C'est-à-dire la transformation est bijective.
	
	Par conséquent, le problème est équivalent à trouver le nombre de chemins
	dans un graphe $(n-1)\times(n+1)$. D'une manière similaire à l'exemple
	\ref{ex:network}, ce nombre de chemins est donné par
	\begin{equation*}
		\frac{(2n-2)!}{(n-2)!n!},
	\end{equation*}
	ce qui est équivalent à $\comb{2n-2}{n-2}$.

	Par conséquent, la probabilité d'avoir un bon chemin au hasard est
	\begin{equation*}
		\P(B)
		=1-\frac{\comb{n-2}{2n-2}}{\comb{n-1}{2n-2}}
		=\frac{1}{n}.
	\end{equation*}
\end{exemple}

\subsection{Réccurence}
\begin{exemple}
	Il est possible de résoudre le problème à l'exemple \ref{ex:network_broken}
	à l'aide de la réccursion.
	
	En effet, le nombre de <<bons>> chemins $b_{i,j}$ à partir d'un noeud est
	la somme des <<bons>> chemins des noeuds à droite et en haut, c'est-à-dire
	\begin{equation*}
		b_{i,j}=b_{i-1,j}+b_{i,j-1},
	\end{equation*}
	où $i$ est le nombre de noeuds restants vers la droite et $j$ la nombre de
	noeuds restants vers le haut.

	On sait que $b_{0,0}=1$, car il ne reste plus de noeud à parcourir fois
	rendu à $B$. Aussi, $b_{0,j}=1$, car il est seulement possible de se rendre
	à $B$ en allant vers le haut. Sur la diagonale, on a $b_{i,i}=b_{i-1,j}$,
	car on ne peut pas aller vers le haut. Par conséquent, on obtient le
	système d'équations à reccurence suivant,
	\begin{equation*}
		b_{i,j}=\left\{
			\begin{matrix}
				0,                 &\text{si}&i=0,j=0\\
				1,                 &\text{si}&i=0\\
				b_{i-1,j},         &\text{si}&i=j\\
				b_{i-1,j}+b_{i,j-1}&\text{sinon}
			\end{matrix}
		\right.
	\end{equation*}

	Pour une grille $4\times 4$, on peut calculer le nombre de <<bons>> chemins
	en développant la récurrence afin d'obtenir $b_{3,3}=5$ chemins.
\end{exemple}

\section{Probabilité conditionnelles}
\begin{definition}
	Une \textit{probabilité conditionnelle} est la probabilité qu'un évènement
	$A$ se réalise, si $B$ s'est réalisé.
\end{definition}

Mathématiquement, on dénote une probabilité conditionnelle avec
\begin{equation*}
	\P(A|B)=\frac{\P(A\cap B)}{\P(B)}.
\end{equation*}
où $A$ et $B$ sont des évènements.

\begin{exemple}
	Soit le lancement d'un dé avec les évènements $A=\left\{5,6\right\}$ et
	$B=\left\{2,4,6\right\}$, alors $\P(A)=\nicefrac{1}{3}$ et 
	\begin{equation*}
		\P(A|B)=\frac{\P(\left\{6\right\})}{\P(B)}=\frac{1}{3}.
	\end{equation*}

	Si $A=\left\{6\right\}$, alors $\P(A)=\nicefrac{1}{6}$ et
	\begin{equation*}
		\P(A|B)=\frac{\P(\left\{6\right\})}{\P(B)}=\frac{1}{3}.
	\end{equation*}
\end{exemple}

\subsection{Propriétés}
\begin{theoreme}
	$\P(A\cap B)=\P(A|B)\cdot \P(B)$.
\end{theoreme}

\begin{theoreme}
	$\P(A|B)=\P(B|A)=0$, si $A\cap B=\emptyset$.
\end{theoreme}

\begin{theoreme}
	$\P(A|B)\neq \P(B|A)$, en général.
\end{theoreme}

\begin{theoreme}
	$\P(A|S)=\P(A)$, $A\in S$.
\end{theoreme}

Le dernier théorème résulte que toutes probabilités peut s'exprimer sous la
forme d'une probabilité conditionnelle.

\begin{exemple}
	On pige sans remise 3 composantes non défecteuses parmis 10 composantes, donc 4 sont
	défecteuses. Soit $A_i$ le $i^\text{e}$ composante non défecteuses.
	\begin{equation*}
		\P(A_1\cap A_2\cap A_3)=\P(A_3|A_1\cap A_2)\P(A_2|A_1)\P(A_1)
		=\frac{4}{8}\cdot\frac{5}{9}\cdot\frac{6}{10}
	\end{equation*}
\end{exemple}

\subsection{Probabilités totales}
\begin{definition}
	Les évènements $B_1,B_2,\dots,B_n$ forment une \textit{partition} si les
	évènements $B_i\cap B_j=\emptyset$, $\forall i\neq j$, et
	$B_1\cup B_2\cup\cdots\cup B_n=S$.
\end{definition}

Avec une partition, on a la règle de probabilités totales, soit
\begin{equation*}
	\P(A)=\sum_{i=1}^n\P(A\cap B_i)=\sum_{i=1}^n\P(A|B_i)\P(B_i).
\end{equation*}

\begin{exemple}
	Soit la partition $B_1$, $B_2$ et $B_3$ représenté dans l'arbre suivant.
	\begin{figure}[H]
		\centering
		\documentclass[11pt]{article}
\usepackage[french]{babel}
\usepackage[utf8]{inputenc}
\usepackage[T1]{fontenc}
\usepackage{tikz}
\usepackage{standalone}

\usetikzlibrary{babel}

\begin{document}
\begin{tikzpicture}[
    level distance=12mm,
    L1/.style ={sibling distance=12em},
    L2/.style ={sibling distance=6em}]
\draw[fill] circle (1pt);
\node[inner sep=0mm] {}
    child[L1] {
        node {$B_1$ $\left(\frac{3}{6}\right)$}
            child[L2] {node {$A$ $\left(\frac{1}{10}\right)$}}
            child[L2] {node {$A^c$ $\left(\frac{9}{10}\right)$}}
    }
    child[L1] {
        node {$B_2$ $\left(\frac{2}{6}\right)$}
            child[L2] {node {$A$ $\left(\frac{1}{10}\right)$}}
            child[L2] {node {$A^c$ $\left(\frac{9}{10}\right)$}}
    }
    child[L1] {
        node {$B_3$ $\left(\frac{1}{6}\right)$}
            child[L2] {node {$A$ $\left(\frac{1}{10}\right)$}}
            child[L2] {node {$A^c$ $\left(\frac{9}{10}\right)$}}
    }
    ;
\end{tikzpicture}
\end{document}


	\end{figure}

	À partir de l'arbre, il facile de déterminer les probabilités
	conditionnelles de $A$ et $A^c$. Par exemple,
	$\P(A|B_1)=\nicefrac{3}{6}\cdot\nicefrac{1}{10}=\nicefrac{1}{30}$ et
	$\P(A^c|B_2)=\nicefrac{2}{6}\cdot\nicefrac{9}{10}=\nicefrac{3}{10}$.
\end{exemple}

\begin{theoreme}[Règle d'inversion]
	$\P(B|A)=\dfrac{\P(A|B)\P(B)}{\P(A)}$
\end{theoreme}

\begin{theoreme}[Règles de Bayes]
	$\P(B_j|A)=\dfrac{\P(A|B_j)\P(B_j)}{\sum_{i=1}^n\P(A|B_i)\P(B_i)}$, où $B_i$ et
	$B_j$ sont des partitions.
\end{theoreme}

\begin{exemple}
	On dépiste le cancer du poumon dans une clinique. On sait que 
	\begin{itemize}
		\item\SI{25}{\percent} des individus sont fumeurs
		\item\SI{75}{\percent} des individus sont non fumeurs
		\item\SI{10}{\percent} des fumeurs développent un cancer
		\item\SI{1 }{\percent} des non fumeurs développent un cancer
	\end{itemize}

	On détecte un cancer des poumons chez un individu sélectionné au hasard
	pour dépistage. Quelle est la probabilité que ça soit un fumeur?

	Soit $B$ les fumeurs, $B^c$ les non fumeurs, $A$ la présence de cancer du
	poumon et $A^c$ son absence. Par conséquent, on cherche $\P(B|A)$. Avec les
	données, on sait que $\P(A|B)=\SI{0.10}{}$, $\P(A|B^c)=\SI{0.01}{}$,
	$\P(B)=\SI{0.25}{}$ et $\P(B^c)=\SI{0.75}{}$, de sorte que le théorème de
	Bayes nous donne
	\begin{equation*}
		\P(B|A)=\frac{\P(A|B)\P(B)}{\P(A|B)\P(B)+\P(A|B^c)\P(B^c)}\approx\SI{0.769}{}.
	\end{equation*}
\end{exemple}

\subsection{Notion d'indépendance}
\begin{definition}
	On dit que les évènements $A$ et $B$ sont \textit{indépendants} si la
	réalisation d'une n'affecte pas l'autrea.
\end{definition}

Mathématiquement parlant, des évènements indépendants $A$ et $B$ sont tels que
$\P(A|B)=\P(A)$ et $\P(B|A)=\P(B)$.

\begin{theoreme}
	$\P(A\cap B)=\P(A)\P(B)\Leftrightarrow \P(A|B)=\P(A)\wedge \P(B|A)=\P(B)$
\end{theoreme}

\begin{theoreme}
	$\P(A\cap B)=\P(A)\P(B)\Leftrightarrow \P(A\cap B^c)=\P(A)\P(B^c)$.
\end{theoreme}

\begin{exemple}
	Soit un système en série ayant $n$ composantes comme à la figure suivante.
	\begin{figure}[H]
		\centering
		\documentclass[11pt]{article}
\usepackage[french]{babel}
\usepackage[utf8]{inputenc}
\usepackage[T1]{fontenc}
\usepackage{tikz}
\usepackage{standalone}

\usetikzlibrary{babel}
\usetikzlibrary{decorations.pathreplacing}

\begin{document}
\begin{tikzpicture}
\node       (S)  at (0,0) {};
\node[draw] (A1) at (1,0) {$A_1$};
\node[draw] (A2) at (3,0) {$A_2$};
\node       (AD) at (5,0) {$\cdots$};
\node[draw] (AN) at (7,0) {$A_n$};
\node       (E)  at (8,0) {};

\draw (S) -- (A1) -- (A2) -- (AD) -- (AN) -- (E);

\draw[decorate,decoration={brace,amplitude=10pt,raise=10pt}]
    (A1.west) -- (AN.east) node[midway,above,yshift=6mm] {$A_S$};

\end{tikzpicture}
\end{document}


	\end{figure}
	Un système en série fonction si et seulement si tous ses composantes
	fonctionnent. Quelle est la probabilité que le système fonctionne si chaque
	composante a une probabilité de \SI{75}{\percent} de fonctionner?

	Soit l'ensemble échantillion $S=\left\{F,D\right\}$ avec $F$ une composante
	qui fonctionne et $D$ une composante défectueuse. De plus, on définit
	$A_i=\left\{F\right\}\in S$ comme étant la composante $i$ qui fonctionne et
	$A_i^c=\left\{D\right\}\in S$ comme étant la composante $i$ qui est
	défecteuse.

	On suppose que les composantes fonctionnent et tombent en panne
	indépendamment les uns des autres. Le système fonctionne si
	\begin{equation*}
		A_S=\bigcap_{i=1}^nA_i,
	\end{equation*}
	de sorte que
	\begin{equation*}
		\P(A_S)
		=\P\left(\bigcap_{i=1}^nA_i\right)
		=\prod_{i=1}^n\P(A_i)
		=\SI{0.75}{}^n.
	\end{equation*}
\end{exemple}

\begin{exemple}
	On génère un point au hasard dans le carré $S=[0,10]\times[0,10]$. On
	définit $A$ comme ayant l'abscisse du point entre 2 et 4, et $B$ comme
	ayant l'ordonnée du point entre 3 et 6. Est-ce que $A$ et $B$ sont
	indépendants?
	\begin{figure}[H]
		\centering
		\documentclass[11pt]{article}
\usepackage[french]{babel}
\usepackage[utf8]{inputenc}
\usepackage[T1]{fontenc}
\usepackage{tikz}
\usepackage{standalone}

\usetikzlibrary{babel}
\usetikzlibrary{patterns}

\begin{document}

\begin{tikzpicture}[scale=0.4]
\draw (0,0) rectangle (10,10);

\path[fill,pattern=north west lines] (2,0) rectangle ( 4,10);
\path[fill,pattern=north east lines] (0,3) rectangle (10,6);

\draw ( 2, 0) -- ( 2,10) ( 4, 0) -- ( 4,10);
\draw ( 0, 3) -- (10, 3) ( 0, 6) -- (10, 6);

\node at (3,8.0) {$A$};
\node at (7,4.5) {$B$};

\node[left] at ( 0,3) {$3$};
\node[left] at ( 0,6) {$6$};
\node[right] at (10,3) {\phantom{$3$}};
\node[right] at (10,6) {\phantom{$6$}};

\node[below] at (2,0) {$2$};
\node[below] at (4,0) {$4$};

\end{tikzpicture}
\end{document}



	\end{figure}
\end{exemple}
\addtocounter{exemple}{-1}
\begin{exemple}[suite]
	Puisqu'il y a équiprobabilité continue, alors les probabilités sont données
	par le rapport entre l'aire de $A$ ou $B$ sur $S$, alors
	$\P(A)=\nicefrac{2}{10}$ et $\P(B)=\nicefrac{3}{10}$. De plus,
	$\P(A\cap B)=\nicefrac{6}{10}$. Puisque $\P(A\cap B)=\P(A)\P(B)$, alors les
	évènements $A$ et $B$ sont indépendants.
\end{exemple}

\section{Variables Aléatoires}
\begin{definition}
	Une \textit{variable aléatoire} correspond à une expérience aléatoire dont
	les résultats sont quantitatifs.
\end{definition}

On peut décrire une variable aléatoire par sa
\begin{itemize}
	\item fonction de répartition
	\item fonction de masse (variable aléatoire discrète)
	\item fonction de densité (variable aléatoire continue)
\end{itemize}

\begin{exemple}
	Soit $X$ le nombre de défaults de soudure d'un transistor à 3 pattes. Quel
	est l'espace échantillion de $X$?

	L'espace échantillion est $S_X=\left\{0,1,2,3\right\}$, alors $X$ est une
	variable aléatoire discrète.
\end{exemple}

\begin{exemple}
	Soit $X$ un nombre réel choisie au hasard dans l'intervalle $[0,2]$. Quel
	est l'espace échantillion de $X$?

	L'espace échantillion est $S_X=[0,2]$, alors $X$ est une variable aléatoire
	continue.
\end{exemple}

\begin{exemple}
	Soit $X$ le temps d'attente d'un client au guichet automatique. Quel est
	l'espace d'échantillion de $X$?

	L'espace échantillion est $S_X={0}\cup [0,\infty[$. Il y a une partie
	discrète et continue, alors $X$ est une variable aléatoire mixte.
\end{exemple}

\subsection{Fonction de répartition}
\begin{definition}
	Un \textit{fonction de répartition} $F_X(x)$ est égal à la probabilité
	qu'une variable aléatoire $X$ soit plus petite qu'une valeur $x$,
	c'est-à-dire $F_X(x)=\P(X\leq x)$, $\forall x\in\mathbb{R}$.
\end{definition}

\begin{theoreme}
	$0\leq F_X(x)\leq 1$.
\end{theoreme}

\begin{theoreme}
	$\displaystyle\lim_{X\rightarrow-\infty}F_X(x)=0$.
\end{theoreme}

\begin{theoreme}
	$\displaystyle\lim_{X\rightarrow \infty}F_X(x)=1$.
\end{theoreme}

\begin{theoreme}[non décroissance]
	$x_0<x_1\Leftrightarrow F_X(x_0)<F_X(x_1)$.
\end{theoreme}

\begin{theoreme}[continuité à droite]
	$F_X(x^+)=F_X(x)$.
\end{theoreme}

Il en résulte que toutes fonctions de répartition $F_X(x)$ sont croissantes,
mais peuvent contenir des discontinuités. Elles peuvent représenté des
variables aléatoires discrètes, continues ou mixtes. La figure
\ref{fig:maxwell} est une exemple de la fonction de répartition d'une
distribution de Maxwell-Boltzmann pour certains paramètres différents.

\begin{figure}[H]
	\centering
	\caption{$F_X(x)$ de certaines distributions de Maxwell-Boltzmann}
	\documentclass[11pt]{article}
\usepackage[french]{babel}
\usepackage[utf8]{inputenc}
\usepackage[T1]{fontenc}
\usepackage{tikz}
\usepackage{pgfplots}
\usepackage{standalone}
\usepackage{amsmath,pgfmath,pgffor}

\usetikzlibrary{babel}

\makeatletter
\pgfmathdeclarefunction{erf}{1}{%
    \begingroup
        \pgfmathparse{#1 > 0 ? 1 : -1}%
        \edef\sign{\pgfmathresult}%
        \pgfmathparse{abs(#1)}%
        \edef\x{\pgfmathresult}%
        \pgfmathparse{1/(1+0.3275911*\x)}%
        \edef\t{\pgfmathresult}%
        \pgfmathparse{%
            1-(((((1.061405429*\t-1.453152027)*\t)+1.421413741)*\t%
            -0.284496736)*\t+0.254829592)*\t*exp(-(\x*\x))}%
        \edef\y{\pgfmathresult}%
        \pgfmathparse{(\sign)*\y}%
        \pgfmath@smuggleone\pgfmathresult%
    \endgroup
}
\makeatother

\begin{document}
\def\E{20}

\begin{tikzpicture}[scale=0.8]
    \begin{axis}[xlabel=$x$,
                 ylabel=$F_X(x)$,
                 xmin=0,
                 xmax=\E,
                 ymin=0,
                 ymax=1.1]
        \def\a{1}\addplot[dotted,samples=100,domain=0:\E]
            {erf(x/(\a*sqrt(2)))-sqrt(2/pi)*(x*exp(-x*x/(2*\a*\a)))/\a};
        \def\a{2}\addplot[dashed,samples=100,domain=0:\E]
            {erf(x/(\a*sqrt(2)))-sqrt(2/pi)*(x*exp(-x*x/(2*\a*\a)))/\a};
        \def\a{5}\addplot[samples=100,domain=0:\E]
            {erf(x/(\a*sqrt(2)))-sqrt(2/pi)*(x*exp(-x*x/(2*\a*\a)))/\a};
    \end{axis}
\end{tikzpicture}
\end{document}



	\label{fig:maxwell}
\end{figure}

\subsection{Fonction de masse}
\begin{definition}
	Une \textit{fonction de masse} $P_X(x_k)$ est égal à la probabilité qu'une
	variable aléatoire $X$ soit égal à une valeur discrète $x_k\in S_X$,
	c'est-à-dire $P_X(x_k)=\P(X=x_k)$ avec
	$S_X=\left\{x_1,x_2,\dots,x_k|x_1<x_2<\cdots<x_k\right\}$
\end{definition}

\begin{theoreme}
	$P_X(x_k)\geq 0$.
\end{theoreme}

\begin{theoreme}
	$\displaystyle\sum_{a< x_k\leq b}P_X(x_k)=\P(a<X\leq b)$.
\end{theoreme}

\begin{theoreme}
	$\displaystyle\sum_{k=1}^\infty P_X(x_k)=1$
\end{theoreme}

Une fonction de masse est nulle en tout point sauf aux valeurs discrètes
possibles. De plus, la somme de toutes les valeurs discrètes donne 1. La figure
\ref{fig:fonction_masse} montre un exemple d'une fonction de masse.

\begin{figure}[H]
	\centering
	\caption{exemple d'une fonction de masse}
	\documentclass[11pt]{article}
\usepackage[french]{babel}
\usepackage[utf8]{inputenc}
\usepackage[T1]{fontenc}
\usepackage{tikz}
\usepackage{pgfplots}
\usepackage{standalone}

\usetikzlibrary{babel}
\usetikzlibrary{calc}

\begin{document}
% control the number of points
\def\E{30}

\begin{tikzpicture}[scale=0.8]
    % the list containing the generated values
    \gdef\n{}
    
    % the sum of all generated values
    \def\s{0.0}
    
    \foreach\i in {2,...,\E}{
        % generate a value between 0 and 1
        \pgfmathparse{(0.5+rand*0.5)}
        \edef\c{\pgfmathresult}
        
        % add it to the sum
        \pgfmathparse{\s+\c}
        \xdef\s{\pgfmathresult}
        
        % add it to the list
        \xdef\n{\n\c,}
    }
    
    \begin{axis}[xlabel=$x$,
                 ylabel=$P_X(x)$,
                 xmin=0,
                 xmax=\E,
                 scaled ticks=false,
                 tick label style={/pgf/number format/fixed},
                 width=10cm,
                 height=5cm]
        % add the horizontal line defining P(x)=0 for non-integers
        \addplot[domain=0:\E]{0};
        
        \pgfmathparse{\E-2}
        \pgfplotsforeachungrouped\i in {0,...,\pgfmathresult}{
            % divide the random value by the sum so the sum of all the new values is 1
            \pgfmathparse{{\n}[\i]*(1/\s)}
            \edef\v{\pgfmathresult}

            % PGF hack to force macro expansion of the vertical lines
            \edef\temp{\noexpand\draw[fill] (axis cs:\i+1,0) -- (axis cs:\i+1,\v);}
            \temp
            
            % PGF hack to force macro expansion of the points
            \edef\temp{\noexpand\addplot[mark=*] coordinates {(\i+1,\v)};}
            \temp
        }
    \end{axis}
            
\end{tikzpicture}
\end{document}



	\label{fig:fonction_masse}
\end{figure}

\subsection{Fonction de densité}
\begin{definition}
	Une \textit{fonction de densité} $f_X(x)$ est égal à la probabilité de
	qu'une variable aléatoire $X$ soit autour d'une valeur $x$,
	c'est-à-dire
	\begin{equation*}
		f_X(x)=\lim_{\epsilon\rightarrow 0}\frac{1}{\epsilon}
		\P\left(x-\frac{\epsilon}{2}\leq X\leq x+\frac{\epsilon}{2}\right),
	\end{equation*}
	où $\epsilon$ est positif.
\end{definition}

\begin{theoreme}
	$f_X(x)\geq 0$
\end{theoreme}

\begin{theoreme}
	$\displaystyle\int_{-\infty}^\infty f_X(x)\d{x}=1$
\end{theoreme}

\subsection{Règles de calcul fondamentale}
\begin{theoreme}\label{th:calc_fond}
	$\P(a<X\leq b)=F_X(b)-F_X(a)$
\end{theoreme}

\begin{proof}
	Soit les ensembles $A=\left\{X\leq a\right\}$, $B=\left\{X\leq b\right\}$
	et $C=\left\{a< X\leq b\right\}$, avec $a<b$, comme à la figure
	\ref{fig:droite_num}.
	\begin{figure}[H]
		\centering
		\documentclass[11pt]{article}
\usepackage[french]{babel}
\usepackage[utf8]{inputenc}
\usepackage[T1]{fontenc}
\usepackage{tikz}
\usepackage{standalone}

\usetikzlibrary{babel}

\begin{document}
\begin{tikzpicture}[y=5mm]
\def\S{0}
\def\E{6}

\def\A{2}
\def\B{5}

\draw[->]   (\S,0) -- (\E,0) node[right] {$x$};
\draw[fill] (\S,1) -- (\A,1) circle (1pt) node[above] {$a$};
\draw[fill] (\S,2) -- (\B,2) circle (1pt) node[above] {$b$};
\draw[fill] (\A,3) -- (\B,3) circle (1pt) node[above] {$b$};

\draw[fill=white] (\A,3) circle (1pt) node[above] {$a$};

\def\space{0.2}
\node at (\S-\space,0) {$S$};
\node at (\S-\space,1) {$A$};
\node at (\S-\space,2) {$B$};
\node at (\S-\space,3) {$C$};

\end{tikzpicture}
\end{document}


		\caption{représentation sur une droite numérique}
		\label{fig:droite_num}
	\end{figure}
	
	On sait que $A\cap C=\emptyset$ et $A\cup C=B$. Par conséquent, 
	$\P(B)=\P(A)+\P(C)\Leftrightarrow \P(C)=\P(B)-\P(A)=F_X(b)-F_X(a)$.
\end{proof}

\begin{theoreme}
	$\P(X=x)=F_X(x)-F_X(x^-)$.
\end{theoreme}

\begin{proof}
	Soit $a=x-\epsilon$ et $b=x$, où $\epsilon$ est positif. Selon le théorème
	\ref{th:calc_fond}, on a
	\begin{equation*}
		\P(x-\epsilon<X\leq x)=F_X(x)-F_X(x-\epsilon).
	\end{equation*}
	En prenant la limite lorsque $\epsilon\rightarrow 0$, on obtient
	\begin{equation*}
		\lim_{\epsilon\rightarrow 0}\P(x-\epsilon<X\leq x)
		=F_X(x)-\lim_{\epsilon\rightarrow 0}F_X(x-\epsilon).
	\end{equation*}
	Avec $F_X(x^-)=\lim_{\epsilon\rightarrow 0}F_X(x-\epsilon)$ et
	$(x-\epsilon<X\leq x)\equiv (X=x)$ lorsque $\epsilon\rightarrow 0$, on
	obtient
	\begin{equation*}
		\P(X=x)=F_X(x)-F_X(x^-).
	\end{equation*}
\end{proof}

\subsection{Liens entre les différentes fonctions}
Toutes variables aléatoires peuvent être décrites par une fonction de
répartition. Lorsque la variable aléatoire est continue, alors elle peut aussi
être décrite pas une fonction de densité de probabilité. Si la variable
aléatoire est discrète, alors elle peut être décrite pas une fonction de masse.

\begin{theoreme}
	$f_X(x)=\dfrac{\d{}}{\d{x}}F_X(x)$ si $X$ est continue.
\end{theoreme}

\begin{theoreme}
	$F_X(x)=\displaystyle\int_{-\infty}^xf_X(t)\d{t}$ si $X$ est continue.
\end{theoreme}

\begin{theoreme}\label{th:masse_repartition}
	$F_X(x)=\displaystyle\sum_{x_k\leq x}P_X(x_k)$ si $X$ est discrète.
\end{theoreme}

\begin{theoreme}
	$P_X(x_k)=\left\{
		\begin{matrix}
			F_X(x_1), & k=1\\
			F_X(x_k)-F_X(x_{k-1}), & k\neq 1\\
		\end{matrix}
	\right.$.
\end{theoreme}

\begin{exemple}
	Soit $X$ le nombre de défaults de soudure d'un transistor à 3 pattes. La
	fonction de masse de $X$ est donnée par
	\begin{equation*}
		P_X(k)=\left\{
			\begin{matrix}
				\SI{0.7}{}, & k=0\\
				\SI{0.1}{}, & k=1\\
				\SI{0.1}{}, & k=2\\
				\SI{0.1}{}, & k=3\\
			\end{matrix}
		\right.
	\end{equation*}
	Quelle est la fonction de répartition de $P_X(k)$?

	On applique le théorème \ref{th:masse_repartition} pour obtenir la
	fonction de répartition. Lorsque $x<0$, alors $F_X(x)=0$. Lorsque
	$0\leq x<1$, alors $F_X(x)=\SI{0.7}{}$. Lorsque $1\leq x<2$, alors
	$F_X(x)=\SI{0.8}{}$. En continuant, on obtient
	\begin{equation*}
		F_X(k)=\left\{
			\begin{matrix}
				\SI{0.0}{}, & \phantom{0\;\leq}k<0\\
				\SI{0.7}{}, & 0\leq k<1\\
				\SI{0.8}{}, & 1\leq k<2\\
				\SI{0.9}{}, & 2\leq k<3\\
				\SI{1.0}{}, & 3\leq k\phantom{<0\;}\\
			\end{matrix}
		\right.
	\end{equation*}
\end{exemple}\pagebreak
\addtocounter{exemple}{-1}
\begin{exemple}[suite]
	Les fonctions de masse et de répartition sont montrées dans les figures
	suivantes.

	\noindent
	\begin{minipage}{0.5\textwidth}
		\begin{figure}[H]
			\centering
			\documentclass[11pt]{article}
\usepackage[french]{babel}
\usepackage[utf8]{inputenc}
\usepackage[T1]{fontenc}
\usepackage{tikz}
\usepackage{pgfplots}
\usepackage{standalone}

\usetikzlibrary{babel}

\begin{document}
\begin{tikzpicture}[scale=0.8]
    \begin{axis}[xlabel=$k$,
                 ylabel=$P_X(k)$,
                 xmin=-1,
                 xmax=4,
                 ymin=-0.1,
                 ymax= 1.1,
                 width=8cm,
                 height=4cm]
        \addplot[domain=-1:4]{0};
        \draw[fill] (axis cs:0,0) -- (axis cs:0,0.7) circle (2pt);
        \draw[fill] (axis cs:1,0) -- (axis cs:1,0.1) circle (2pt);
        \draw[fill] (axis cs:2,0) -- (axis cs:2,0.1) circle (2pt);
        \draw[fill] (axis cs:3,0) -- (axis cs:3,0.1) circle (2pt);
    \end{axis}
\end{tikzpicture}
\end{document}



		\end{figure}
	\end{minipage}%
	\begin{minipage}{0.5\textwidth}
		\begin{figure}[H]
			\centering
			\documentclass[11pt]{article}
\usepackage[french]{babel}
\usepackage[utf8]{inputenc}
\usepackage[T1]{fontenc}
\usepackage{tikz}
\usepackage{pgfplots}
\usepackage{standalone}

\usetikzlibrary{babel}
\usetikzlibrary{decorations.markings}

\begin{document}
\tikzset{
    o/.style={
        shorten >=#1,
        decoration={
            markings,
            mark={
                at position 1
                with {
                    \draw circle [radius=#1];
                }
            }
        },
        postaction=decorate
    },
    o/.default=2pt
}

\tikzset{
    */.style={
        shorten <=#1,
        decoration={
            markings,
            mark={
                at position 0
                with {
                    \draw[fill] circle [radius=#1];
                }
            }
        },
        postaction=decorate
    },
    */.default=2pt
}

\begin{tikzpicture}[scale=0.8]
    \begin{axis}[xlabel=$k$,
                 ylabel=$F_X(k)$,
                 xmin=-1,
                 xmax= 4,
                 ymin=-0.1,
                 ymax= 1.1,
                 width=8cm,
                 height=4cm]
        \draw[  o,fill] (axis cs:-1,0.0) -- (axis cs:0,0.0);
        \draw[*,o,fill] (axis cs: 0,0.7) -- (axis cs:1,0.7);
        \draw[*,o,fill] (axis cs: 1,0.8) -- (axis cs:2,0.8);
        \draw[*,o,fill] (axis cs: 2,0.9) -- (axis cs:3,0.9);
        \draw[*,  fill] (axis cs: 3,1.0) -- (axis cs:4,1.0);
    \end{axis}
\end{tikzpicture}
\end{document}



		\end{figure}
	\end{minipage}
\end{exemple}

\begin{exemple}
	Soit $X$ un nombrbe réel choisie au hasard dans l'intervalle $[0,2]$.
	Quelle est la fonction de répartition de $X$?

	On sait que $F_X(x)=\P(X\leq x)=\nicefrac{x}{2}$ lorsque $0\leq x\leq 2$,
	$F_X(x)=0$ si $x<0$ et $F_X(x)=1$ si $x>2$. On obtient donc la fonction de
	répartition
	\begin{equation*}
		F_X(k)=\left\{
			\begin{matrix}
				0,            & \phantom{0\;\leq}x<0\\
				\dfrac{x}{2}, & 0\leq x\leq 2\\
				1,            & 2\leq x\phantom{\leq 0\;}\\
			\end{matrix}
		\right.
	\end{equation*}
\end{exemple}

\begin{exemple}
	Soit $X$ le temps d'attente d'un client au guichet automatique. On suppose
	que \SI{10}{\percent} des visites sont sans attente. Quel est la fonction
	de répartition?
	
	La variable aléatoire est mixte. On sait que $F_X(0)=\nicefrac{1}{10}$ et
	que $F_X(x)=0$ si $x<0$. La fonction de répartition est
	\begin{equation*}
		F_X(k)=\left\{
			\begin{matrix}
				0, & x < 0\\
				\dfrac{x+1}{x+10}, & x \geq 0\\
			\end{matrix}
		\right.
	\end{equation*}
	Le graphique suivant montre la fonction de répartition $F_X(x)$.
	\begin{figure}[H]
		\centering
		\documentclass[11pt]{article}
\usepackage[french]{babel}
\usepackage[utf8]{inputenc}
\usepackage[T1]{fontenc}
\usepackage{tikz}
\usepackage{pgfplots}
\usepackage{standalone}

\usetikzlibrary{babel}

\begin{document}
\def\E{100}

\begin{tikzpicture}[scale=0.8]
    \begin{axis}[xlabel=$x$,
                 ylabel=$F_X(x)$,
                 xmin=-10,
                 xmax=\E,
                 ymin=-0.1,
                 ymax=1]
        \addplot[domain=0:\E]{(x+1)/(x+10)};
        \addplot[domain=-10:0-1pt]{0};
        \draw[fill] (axis cs:0,0.1) circle (2pt);
        \draw       (axis cs:0,0.0) circle (2pt);
    \end{axis}
\end{tikzpicture}
\end{document}



	\end{figure}
\end{exemple}

\subsection{Fonction conditionnelle}
\begin{definition}
	Une \textit{fonction conditionnelle} est la probabilité qu'une variable
	aléatoire $X$ prenne une valeur plus petit ou égal à  $x$ sachant un
	évènement $A$.
\end{definition}

\begin{theoreme}
	$F_X(x|A)=\dfrac{\P(\left\{X\leq x\right\}\cap A)}{\P(A)}$.
\end{theoreme}

\begin{theoreme}
	$f_X(x|A)=\dfrac{\d{}}{\d{x}}F_X(x|A)$ si $X$ est continue.
\end{theoreme}

\begin{theoreme}
	$P_X(x_k|A)=
	\left\{
		\begin{matrix}
			\dfrac{P_X(x_k)}{\P(A)}, & x_k\in A\\
			0, & x_k\notin A\\
		\end{matrix}
	\right.$
	si $X$ est discrète.
\end{theoreme}



\subsection{Médiane et quantile}
\begin{definition}
	La médiane d'une variable aléatoire $X$ continue est le nombre réel
	$x_{1/2}$ tel que $F_X(x_{1/2})=\nicefrac{1}{2}$.
\end{definition}

\begin{definition}
	Le \textit{quantile} d'ordre $p$ d'une variable aléatoire $X$ continue est
	le nombre réel $x_p$ tel que $F_X(x_p)=p$.
\end{definition}

\begin{exemple}
	Dans une certaine population, la taille $X$ d'un adulte choisi au hasard
	possède la fonction de répartition 
	\begin{equation*}
		F_X(x)=\left\{
			\begin{matrix}
				0,        & \phantom{1.2\leq\;}x<1.2\\
				1.5x-1.8, & 1.2\leq x<1.7\\
				0.5x-0.1, & 1.7\leq x<2.2\\
				1,        & 2.2\leq x\phantom{<2.2\;}\\
			\end{matrix}
		\right.
	\end{equation*}
	Calculer la médiane et le quantile d'ordre 95.

	Puisque $F_X(1.7)=0.75$, alors le quantile d'ordre 95 est dans la tranche
	$1.7\leq x<2.2$. Il suffit de résoudre $0.95=0.5x_{0.95}-0.1$ et on obtient
	que $x_{0.95}=2.1$.
\end{exemple}

\end{document}

